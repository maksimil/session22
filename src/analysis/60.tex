\documentclass{article}

\usepackage{defines}

\usepackage{tikz, pgfplots, caption}

\begin{document}

\tickettitle{60}{Точка перегиба функции. Теорема о точках перегиба $n$ раз дифференцируемой функции.}

\define{точки перегиба}

\begin{minipage}{0.7\linewidth}
	\raggedright
	$\llet y = f(x)$ дифференцируема в точке $ c$, тогда $ c$ --- точка перегиба графика $y=f(x)$, если в некоторой достаточно малой окрестности $c$ все точки
	при $x<c$ лежат по одну сторону от касательной, а $x>c$ --- по другую (\figref{60:recurve}).
	\begin{enumerate}
		\item{}$f$ имеет в $c$ точку перегиба $\land$ $\exists f'(x)$ в окрестности $c$\\
		$\Rarr$ $f'$ имеет в $c$ локальный экстремум.
		\item{}$f$ имеет в $c$ точку перегиба $\land$ $\exists f''(c)$ $\Rarr$ $f''(c)=0$.
        \item{} Если в $c \exists$ конечная $f''(x)$, то необходимо, чтобы $f''(x) = 0$, для перегиба в $с$. Это необходимое условие, но оно не является достаточным.
        \item{} Если в некоторой окрестности $c \exists f''(x)$, за исключением, может быть, самой $c$. При этом $f(x)$ ---непрерывна и имеет в $c$ касательную, тогда если $f''(x)$ имеет разные знаки слева и справа от $c$, тогда в $c$ есть перегиб.` 
	\end{enumerate}
\end{minipage}%
\begin{minipage}{0.3\linewidth}
	\centering
	\begin{tikzpicture}
		\begin{axis}[
				axis x line=center,
				axis y line=center,
				ticks=none,
				xmin=-1,
				xmax=3,
				ymin=-1,
				ymax=3,
				width=50mm,
				height=40mm,
				scale only axis]
			\addplot[mark=none,domain={-1:3},samples=60]{(x-1)^3-x+2};
			\addplot[mark=none,domain={-0.5:2.5},samples=2]{2-x};
		\end{axis}
	\end{tikzpicture}
	\captionof{figure}{Точка перегиба}\label{60:recurve}
\end{minipage}

\theorem

$f$ --- $n$ раз дифференцируема в окрестности $a$, и $f^{(n)}$ --- непрерывна в этой окрестности
\begin{align*}
	(\forall m=\overline{2,n-1})\;f^{(m)}(a)=0 \land f^{(n)}(a)\neq 0
	\Rarr\text{$c$ --- точка перегиба $f$, если $n$ --- нечётно}
\end{align*}

\proof

По теореме о стабилизации знака функции:
\begin{align*}
	f^{(n)}(c)\neq 0\Rarr \exists\eps>0:(\forall h\in(-\eps;\eps))\;f^{(n)}(c+h)\neq 0\land f^{(n)}(c+h)\text{ имеет тот же знак, что и $f^{(n)}(c)$}
\end{align*}

По формуле Тейлора:
\begin{align*}
	 & (\forall h\in(-\eps;\eps))\; f(c+h)=f(c)+hf'(c)+\sum_{k=2}^{n-1}\frac{h^{k}}{k!}f^{(k)}(c)+\frac{h^{n}}{n!}f^{(n)}(c+\theta h),\;\theta\in(0;1)\Rarr \\
	 & \Rarr f(c+h)=f(c)+hf'(c)+\frac{h^{n}}{n!}f^{(n)}(c+\theta h)
\end{align*}

$n=2m-1,\;m\in\N$
\begin{enumerate}
	\item{}$f^{(n)}(c)>0$
	\begin{align*}
		 & \begin{cases}
			   n=2m-1       & \Rarr \sgn(h^{n})=\sgn(h)   \\
			   f^{(n)}(c)>0 & \Rarr f^{(n)}(c+\theta h)>0
		   \end{cases}\Rarr sgn\left(\frac{h^{n}}{n!}f^{(n)}(c+\theta h)\right)=\sgn(h)\Rarr \\
		 & \Rarr \begin{cases}
			         f(c+h)>f(c)+hf'(c) & \text{ при $h>0$ } \\
			         f(c+h)<f(c)+hf'(c) & \text{ при $h<0$ } \\
		         \end{cases}\Rarr c \text{ --- точка перегиба }
	\end{align*}
	\item{}$f^{(n)}(c)<0$
	\begin{align*}
		 & \begin{cases}
			   n=2m-1       & \Rarr \sgn(h^{n})=\sgn(h)   \\
			   f^{(n)}(c)<0 & \Rarr f^{(n)}(c+\theta h)<0
		   \end{cases}\Rarr sgn\left(\frac{h^{n}}{n!}f^{(n)}(c+\theta h)\right)=-\sgn(h)\Rarr \\
		 & \Rarr \begin{cases}
			         f(c+h)>f(c)+hf'(c) & \text{ при $h<0$ } \\
			         f(c+h)<f(c)+hf'(c) & \text{ при $h>0$ } \\
		         \end{cases}\Rarr c \text{ --- точка перегиба }
	\end{align*}
 Если $f'(c) = 0$, то $c$ подозрительная на $extr$
\end{enumerate}
\end{document}