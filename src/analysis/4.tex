\documentclass{article}

\usepackage{defines}

\begin{document}

\tickettitle{4}{Эквивалентность множеств. Несчётность множества вещественных чисел. Понятие~мощности~множества.}

\define{эквивалентности множеств}

Множества $A$ и $B$	эквивалентны, если $\exists f:A\leftrightarrow B$.

$f:A\leftrightarrow B$ означает, что $f:A\to B$ взаимно-однозначно.

$A\sim B$ означает, что множества эквивалентны.

\define{истинной части}

$A$ --- истинная часть $B$, если $A\subset B\land A\sim B$

{\it Пример:}

$(0;1)$ --- истинная часть $\R$
\begin{align*}
	 & f:(0;1)\leftrightarrow\R            &  & (0;1)\subset\R \\
	 & f(x)=\tg\left(\frac{x\pi}{2}\right) &  & (0;1)\sim\R
\end{align*}

\theorem

$\R$ несчётно

\proof
\begin{enumerate}
	\item{}$\R\sim(0;1)$, значит, из $(0;1)\nsim\N$ будет следовать $\R\nsim\N$.
	\item{}Докажем $(0;1)\nsim\N$.

	Пусть существует перечень бесконечных десятичных дробей $S=\setdef{a_n}{\forall n\in\N}$:
	\begin{align*}
		 & a_1=0,a_{11}a_{12}a_{13}a_{14}... \\
		 & a_2=0,a_{21}a_{22}a_{23}a_{24}... \\
		 & ...                               \\
		 & a_n=0,a_{n1}a_{n2}a_{n3}a_{n4}... \\
		 & ...
	\end{align*}

	Построим тогда $\beta\in(0;1)\land\beta\notin S$:
	\begin{align*}
		 & \beta=0,\beta_1\beta_2\beta_3\beta_4\beta_5...                               \\
		 & \beta_n\neq a_{nn}\land \beta_n\neq 9\land\beta_n\neq 0\;\forall n\in\N\Rarr \\
		 & \Rarr \beta\neq a_n\;\forall n\in\N \land  0<\beta<1\Rarr                    \\
		 & \Rarr \beta\notin S\land \beta\in(0;1)
	\end{align*}

	В $(0;1)$ существуют числа вида $\frac{q}{10^{n}}\;(q\in\N,n\in\N)$, представимые двумя способами:
	$0,5=0,5(0)=0,4(9)$, поэтому в построении $\beta$ нельзя использовать $0$ и $9$.

	Таким образом, для любого перечня $(0;1)$ существует элемент в $(0;1)$, который не содержится в перечне $\Rarr(0;1)\nsim\N\Rarr\R\nsim\N\qed$

\end{enumerate}

\theorem[Кантора-Бернштейна]
\begin{align*}
	(\exists\text{инъекция }f:A\to B',\;B'\subset B)\land(\exists\text{инъекция }g:B\to A',\;A'\subset A)\Rarr A\sim B
\end{align*}

\pagebreak

\define{мощности множества}

Мощности множеств $A$ и $B$ равны, если $A\sim B$.

Мощность --- то общее, что есть у эквивалентных множеств.

$m(A)$ --- кардинальное число множества $A$.
$m(A)=m(B)$ означает, что мощности $A$ и $B$ равны.

\define{сравнения мощностей множеств}

$f$ и $g$ --- инъекции
\begin{enumerate}
	\item{}$(\exists f:A\to B',\;B'\subset B)\land(\exists g:B\to A,\;A'\subset A)\Rarr m(A)=m(B)$
	\item{}$(\exists f:A\to B',\;B'\subset B)\land(\nexists g:B\to A,\;A'\subset A)\Rarr m(A)<m(B)$
	\item{}$(\nexists f:A\to B',\;B'\subset B)\land(\nexists g:B\to A,\;A'\subset A)\Rarr$ $m(A)$ и $m(B)$ несравнимы,
\end{enumerate}


\end{document}
