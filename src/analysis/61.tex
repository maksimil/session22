\documentclass{article}

\usepackage{defines}

\begin{document}

\tickettitle{61}{Исследования на экстремум функции, $n$ раз дифференцируемой в стационарной точке.}

\define{стационарной точки}

$c$ --- стационарная точка $f$ означает $f'(c)=0$

\theorem

$f$ --- $n$ раз дифференцируема в окрестности $c$, и $f^{(n)}$ --- непрерывна в этой окрестности
\begin{align*}
	(\forall m=\overline{1,n-1})\;f^{(m)}(c)=0 \land f^{(n)}(c)\neq 0\Rarr
	\text{$c$ --- локальный экстремум $f$, если $n$ --- чётно}
\end{align*}

\proof

По теореме о стабилизации знака функции:
\begin{align*}
	f^{(n)}(c)\neq 0\Rarr \exists\eps>0:(\forall h\in(-\eps;\eps))\;f^{(n)}(c+h)\neq 0\land f^{(n)}(c+h)\text{ имеет тот же знак, что и $f^{(n)}(c)$}
\end{align*}

По формуле Тейлора:
\begin{align*}
	 & (\forall h\in(-\eps;\eps))\; f(c+h)=f(c)+\sum_{k=1}^{n-1}\frac{h^{k}}{k!}f^{(k)}(c)+\frac{h^{n}}{n!}f^{(n)}(c+\theta h),\;\theta\in(0;1)\Rarr \\
	 & \Rarr f(c+h)=f(c)+\frac{h^{n}}{n!}f^{(n)}(c+\theta h)
\end{align*}

$h\neq 0$
\begin{enumerate}
	\item{}$f^{(n)}(c)>0$
	\begin{align*}
		 & \begin{cases}
			   n\text{ --- чётно} & \Rarr h^{n}>0               \\
			   f^{(n)}(c)>0       & \Rarr f^{(n)}(c+\theta h)>0
		   \end{cases}\Rarr \frac{h^{n}}{n!}f^{(n)}(c+\theta h)>0\Rarr                                                  \\
		 & \Rarr (\forall h\in(-\eps;\eps)\setminus\{0\})\;f(c+h)>f(c)\Rarr c \text{ --- точка локального минимума}\qed
	\end{align*}
	\item{}$f^{(n)}(c)<0$
	\begin{align*}
		 & \begin{cases}
			   n\text{ --- чётно} & \Rarr h^{n}>0               \\
			   f^{(n)}(c)<0       & \Rarr f^{(n)}(c+\theta h)<0
		   \end{cases}\Rarr \frac{h^{n}}{n!}f^{(n)}(c+\theta h)<0\Rarr                                                    \\
		 & \Rarr (\forall h\in(-\eps;\eps)\setminus\{0\})\; f(c+h)<f(c)\Rarr c \text{ --- точка локального максимума}\qed
	\end{align*}
\end{enumerate}

\end{document}
