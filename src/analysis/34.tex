\documentclass{article}

\usepackage{defines}

\begin{document}

\tickettitle{34}{Определение равномерной непрерывности функции. Теорема Кантора}

\define{равномерной непрерывности функции}

Функция $f$  называется равномерно непрерывной на промежутке $X$, если
\begin{align*}
	\forall\eps>0\;\exists\delta>0:(\forall x_1,x_2\in X)\;|x_1-x_2|<\delta\Rarr |f(x_1)-f(x_2)|<\eps
\end{align*}

\theorem[Кантора]

$f$ --- непрерывна на $[a;b]$ $\Rarr$ $f$ --- равномерно непрерывна на $[a;b]$

\proof

Предположим обратное
\begin{align*}
	 & \lnot(\forall\eps>0\;\exists\delta>0:(\forall x_1,x_2\in [a;b])\;|x_1-x_2|<\delta\Rarr |f(x_1)-f(x_2)|<\eps)\Rarr \\
	 & \Rarr\exists\eps>0:\forall\delta>0\;\exists x_1,x_2\in [a;b]:|x_1-x_2|<\delta\land|f(x_1)-f(x_2)|\geq\eps
\end{align*}

Построим последовательности $\{x_{n}\}$ и  $\{x'_{n}\}$
\begin{align}
	\forall n\in\N\;\llet\delta=\frac{1}{n}\;\exists x_{n},x'_{n}\in [a;b]:|x_{n}-x'_{n}|<\delta=\frac{1}{n}\land|f(x_{n})-f(x'_{n})|\geq\eps\label{34:constr}
\end{align}

$\{x_{n}\}\subset[a;b]\Rarr\{x_{n}\}$ --- ограничена, тогда по принципу выбора Больцано-Вейерштрасса существует подпоследовательность $\{x_{k_{n}}\}:\exists\slimty x_{k_{n}}=:x_0$

Причём $\{x_{n}\}\subset[a;b]$ значит $x_0\in[a;b]$ $\Rarr$ $f$ --- непрерывна в $x_0$

По построению
\begin{align*}
	|x_{k_{n}}-x'_{k_{n}}|<\frac{1}{n}\Rarr\slimty|x_{k_{n}}-x'_{k_{n}}|=0\Rarr\slimty x_{k_{n}}-\slimty x'_{k_{n}}=0\Rarr\slimty x'_{k_{n}}=x_0
\end{align*}

По непрерывности $f$
\begin{align*}
	 & \slimty x_{k_{n}}=x_0\Rarr\slimty f(x_{k_{n}})=f(x_0)   \\
	 & \slimty x'_{k_{n}}=x_0\Rarr\slimty f(x'_{k_{n}})=f(x_0)
\end{align*}

Последовательности $\{f(x_{k_{n}})\}$ и $\{f(x'_{k_{n}})\}$ сходятся к одному пределу, что противоречит $\eqref{34:constr}$

Пришли к противоречию, значит $f$ --- равномерно непрерывна на $[a;b]\qed$

\end{document}
