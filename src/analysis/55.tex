\documentclass{article}

\usepackage{defines}

\begin{document}

\tickettitle{55}{Производные высших порядков, их свойства, формула Лейбница.}

\define{производных высших порядков}

Пусть $f$ имеет на $P$ производные $f'(x),f''(x),...,f^{(n-1)}(x)$.

Если в $x_0\in P$ $\exists f^{(n)}(x_0)$, то её называют производной $n$-го порядка в $x_0$
\begin{align*}
	 & f^{(n)}(x):=(f^{(n-1)}(x))' & f^{(0)}(x):=f(x)
\end{align*}

\define{$n$ раз дифференцируемой функции}

Если $(\forall x\in X)\;\exists f^{(m)}(x)\;\forall m\leq n$, то $f$ --- $n$ раз дифференцируема на $X$.

Если $(\forall x\in X, n\in\N)\;\exists f^{(n)}(x)$, то $f$ --- бесконечно дифференцируема на $X$.

\props{производных высших порядков}

Из определения следует:
\begin{align*}
	 & (cf(x))^{(n)}=cf^{(n)}(x),\; c = \const &  & (f(x)+g(x))^{(n)}=f^{(n)}(x)+g^{(n)}(x)
\end{align*}

\theorem[(формула Лейбница)]
\begin{align*}
	(uv)^{(n)}=\sum_{k=0}^{n}C_n^{k}u^{(k)}v^{(n-k)}
\end{align*}

\proof

{\it Индукция:} $P(n)$ --- верность формулы Лейбница для $n$
\begin{enumerate}
	\item{}$P(1)$
	\begin{align*}
		(uv)'=uv'+u'v=C_1^{0}uv'+C_1^{1}u'v=\sum_{k=0}^{1}C_1^{k}u^{(k)}v^{(1-k)}
	\end{align*}

	\item{}$P(n-1)\Rarr P(n)$
	\begin{align*}
		(uv)^{(n)}
		 & =((uv)^{(n-1)})'=\left(\sum_{k=0}^{n-1}C_{n-1}^{k}u^{(k)}v^{(n-1-k)}\right)'
		=\sum_{k=0}^{n-1}C_{n-1}^{k}u^{(k+1)}v^{(n-1-k)}+\sum_{k=0}^{n-1}C_{n-1}^{k}u^{(k)}v^{(n-k)}=                                                        \\
		 & =\sum_{k=1}^{n}C_{n-1}^{k-1}u^{(k)}v^{(n-k)}+\sum_{k=0}^{n-1}C_{n-1}^{k}u^{(k)}v^{(n-k)}=                                                         \\
		 & =C_{n-1}^{n-1}u^{(n)}v^{(0)}+\sum_{k=1}^{n-1}C_{n-1}^{k-1}u^{(k)}v^{(n-k)}+\sum_{k=1}^{n-1}C_{n-1}^{k}u^{(k)}v^{(n-k)}+C_{n-1}^{0}u^{(0)}v^{(n)}= \\
		 & =C_{n}^{0}u^{(0)}v^{(n)}+\sum_{k=1}^{n-1}(C_{n-1}^{k-1}+C_{n-1}^{k})u^{(k)}v^{(n-k)}+C_{n}^{n}u^{(n)}v^{(0)}=                                     \\
		 & =C_{n}^{0}u^{(0)}v^{(n)}+\sum_{k=1}^{n-1}C_n^{k}u^{(k)}v^{(n-k)}+C_{n}^{n}u^{(n)}v^{(0)}=\sum_{k=0}^{n}C_n^{k}u^{(k)}v^{(n-k)}\qed
	\end{align*}
\end{enumerate}

\end{document}
