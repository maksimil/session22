\documentclass{article}

\usepackage{defines}

\begin{document}

\tickettitle{68}{Иррациональное число e.}

\theorem
\begin{align*}
	e=\slimty\sum_{k=0}^{n}\frac{1}{k!}
\end{align*}

\proof

Найдём разложение $f(x)=e^{x}$ по формуле Маклорена

\begin{enumerate}
	\item{}Найдём $f^{(n)}(0)$:
	\begin{align*}
		f^{(n)}(x)=e^{x}\Rarr f^{(n)}(0)=1
	\end{align*}
	\item{}Найдём разложение:
	\begin{align*}
		e^{x} & =f(x_0)+\sum_{k=1}^{n}\frac{(x-x_0)^{k}}{k!}f^{(k)}(x_0)+\frac{(x-x_0)^{n+1}}{(n+1)!}f^{(n+1)}(x_0+\theta(x-x_0))= \\
		      & =\sum_{k=0}^{n}\frac{x^{k}}{k!}+\frac{x^{n+1}}{(n+1)!}e^{\theta x},\;\theta\in(0;1)
	\end{align*}
\end{enumerate}

\newcommand\ep{\widetilde{e}}

Тогда можно найти приближённое значение $e$:
\begin{align}
	 & \ep_{n}:=\sum_{k=0}^{n}\frac{1}{k!} &  & R_{n}:=\frac{e^{\theta}}{(n+1)!},\;\theta\in(0;1) &  & e=e^{1}=\ep_{n}+R_{n}\label{68:taylor_e}
\end{align}

И оценить погрешность:
\begin{align}
	 & |\ep_{n}-e|=|R_{n}|=\frac{|e^{\theta}|}{(n+1)!}<\frac{e}{(n+1)!}<\frac{3}{(n+1)!} &  & \frac{1}{(n+1)!}<R_{n}<\frac{e}{(n+1)!}<\frac{3}{(n+1)!}\label{68:rlims}
\end{align}
\begin{align*}
	 & \slimty|\ep_{n}-e|=\slimty\frac{3}{(n+1)!}=0                                                                                \\
	 & \slimty|\ep_{n}-e|=\left|\left(\slimty\ep_{n}\right)-e\right|=0\Rarr e=\slimty\ep_{n}=\slimty\sum_{k=0}^{n}\frac{1}{k!}\qed
\end{align*}

\pagebreak

\theorem

Число $e$ иррационально

\proof

Пусть $e$ --- рационально $\Rarr$ $\exists m,n\in\N:e=\frac{m}{n}$
\begin{align*}
	e=\frac{m}{n}\in\Q\Rarr n!e=(n-1)!m\in\N
\end{align*}

Воспользуется $\eqref{68:taylor_e}$:
\begin{align*}
	n!e=\sum_{k=0}^{n}\frac{n!}{k!}+n!R_{n}\in\N\land (\forall k\leq n)\;\frac{n!}{k!}\in\N\Rarr n!R_{n}\in\N
\end{align*}

Теперь воспользуемся оценкой $\eqref{68:rlims}$:
\begin{align*}
	 & \frac{1}{(n+1)!}<R_{n}<\frac{3}{(n+1)!}\Rarr\frac{n!}{(n+1)!}<n!R_{n}<\frac{3n!}{(n+1)!}\Rarr \frac{1}{n+1}<n!R_{n}<\frac{3}{n+1} \\
	 & \llet n=2\Rarr \frac{1}{3}<n!R_{n}<1\text{, но }n!R_{n}\in\N
\end{align*}

Пришли к противоречию, значит, таких $m$ и $n$ не существует и $e$ --- иррационально$\qed$

\end{document}