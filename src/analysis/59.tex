\documentclass{article}

\usepackage{defines}

\begin{document}

\tickettitle{59}{Построение графиков функций с помощью дифференциального исчисления: локально-выпуклые и локально-вогнутые функции {\it Билет не просмотрен Ксюшей, но проверен Артёмом}}

\define{} \\
$\sqsupset$ функция $y = f(x)$ определена и непрерывна на $X$. Говорят, что эта функция является локально выпуклой (выпуклой вниз), если $\forall x_1, x_2 \in X, \ x_1 < x_2: f(q_1x_1 + q_2x_2) \leq q_1f(x_1) + q_2f(x_2)$, $\forall q_1, q_2 > 0: \ q_1 + q_2 = 1$. \\
Если $f(q_1x_1 + q_2x_2) \geq q_1f(x_1) + q_2f(x_2)$ для $\forall x_1, x_2 \in X$, то $f(x)$ -- локально вогнутая (выпуклая вверх) на $X. \newline$ \\

$(\cdot) A: (x, f(q_1x_1 + q_2x_2))$ \\
$(\cdot) B: (x, q_1f(x_1) + q_2f(x_2)) \quad$ график!!! $\newline$ \\

Это определение не использует понятие производной. \\

\define{} \\
$y = f(x)$ -- дифференцирована в $(\cdot) c$. Кривая $y = f(x)$ -- локально выпуклая в $(\cdot) c$, \\ если $\exists$ окрестность $O\varepsilon$ этой точки $O\varepsilon = (c - \varepsilon, c + \varepsilon)$, такая, что для $\forall x \in O\varepsilon$ точки кривой лежат над касательной кривой в $(\cdot) c$. \\
Если же точки лежат под касательной прямой в окрестности $O\varepsilon$, то говорят, что кривая является локально вогнутой в $(\cdot) c. \newline$

два графика:
\begin{enumerate}
	\item локально выпуклая в $(\cdot) c$
	\item локально вогнутая в $(\cdot) c$
\end{enumerate}

Пусть $f$ --- определена на $(a,b)$ и $l(x) = \frac{f(x_2)(x-x_1) + f(x_1)(x_2-x)}{x_2-x_1}$ --- прямая, проходящая через точки $A(x_1, f(x_1))$ и $B(x_2, f(x_2))$

\theorem

$y = f(x)$ имеет непрерывную вторую производную в $(\cdot) c$, тогда, если $f''(c) > 0 \ (< 0)$, то кривая в $(\cdot) c$ -- локально выпуклая (локально вогнутая).

\proof

Воспользуемся теоремой о производной $n$-го порядка и точке перегиба: $\ \sqsupset f''(c) > 0$, $\sqsupset h > 0$ и $c + h > c$. Тогда по формуле Тейлора: $f(c+h) = f(c) + \frac{h}{1!} \cdot f'(c) + \frac{h^2}{2!} \cdot f''(c + \theta h)$ \\
По условию $f''(c) > 0$, тогда по непрерывности $f''(x)$ \\
$f''(c + \theta h) > 0$ для достаточно малых $h$. \\
$\Rightarrow \ f(c+h) - f(c) - h \cdot f'(c) = \frac{h^2}{2!} \cdot f''(c + \theta h) > 0$ \\
$\Rightarrow \ f(c+h) - f(c) > 0$ \\
$\Rightarrow \ f(x)$ в $(\cdot) c$ -- локально выпуклая.

Аналогично, для случая, когда $f''(c) < 0$.

\define{} \\
Если $y = f(x)$ является локально выпуклой в каждой точке из промежутка $X$ (локально вогнутой), то говорят, что $f(x)$ -- выпуклая (вогнутая) на промежутке $X$. \\


\end{document}
