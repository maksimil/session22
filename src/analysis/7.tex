\documentclass{article}

\usepackage{defines}

\begin{document}

\tickettitle{7}{Понятие функции (отображения). Числовая последовательность, определение~числовой~последовательности.}

\define{функции}

Пусть заданы непустые множества $X$ и $Y$

Соответствие, по которому каждому элементу $x \in X$ соответстует единственный элемент $y \in Y$, называется функцией,
заданной (определённой) на множестве $X$ со значениями в множестве $Y$ или отображением множества $X$ в множество $Y$

Функция (отображение) $f$ из $X$ в $Y$ обозначается $f:X\to Y$ и\\
любому $x\in X$ ставится в соответствие $y=f(x)\in Y$

Элемент $x\in X$ называется независимым переменным или аргументом,\\
а соответствующий элемент $y \in Y$ --- зависимым переменным

Множество $X$ называется множеством задания (определения) функции $f$, а множестов тех $y \in Y$, для которых $\exists x\in X:y=f(x)$ --- множеством значений функции $f$

Виды задания фунций:
\begin{enumerate}
	\item{}Явный: $x$ и $f(x)$ --- известны, $y=f(x)$
	\item{}Неявный: существует какая-то формула связывающая $x$ и $f(x)$
	\item{}Табличный
	\item{}Графический
\end{enumerate}

\define{естественного расширения функции}

Естественное расширение отображения $f: X\to Y$, это отображение $\widetilde{f}$, заданное на множестве подмножеств множества $X$ формулой
\begin{align*}
	\widetilde{f}(\alpha):=\setdef{y\in Y}{\exists x\in \alpha:y=f(x)},\;\alpha \subset X
\end{align*}

Обозначение: $\widetilde{f}: 2^X \to 2^Y$, где $2^{A}:=\setdef{\alpha}{\alpha\subset A}$

Если $z \in 2^X$, то $\widetilde{f}(z)$ --- образ множества $z$, а $z$ --- прообраз множества $\widetilde{f}(z)$

Обычно тильду опускают и $\widetilde{f}$ тоже обозначают через $f$

\define{числовой последовательности}

Пусть существует отображение $f:\N\to\R$ и $a_{n}=f(n)$

Тогда получим числовую последовательность $\{a_{n}\}_{n=1}^{\infty}$, где элементы $a_{n}$ расположены в пордяке возрастания $n$

Виды задания последовательности:
\begin{enumerate}
	\item{}Явный (в виде формулы): $a_n=f(n)$
	\item{}Неявный или рекурентный: $a_n=f(a_{n-1}, a_{n-2}, ...)$
	\item{}Существуют последовательности, которые нельзя задать какой-либо формулой. (например, последовательность простых чисел)
\end{enumerate}

\pagebreak

\sectitle{Классификация числовых последовательностей}

Пусть $\{a_n\}^{\infty}_{n=1}$ --- последовательрность вещественных чисел.
\begin{enumerate}
	\item{} Если $(\forall n\in\N)\;a_{n+1}>a_{n}$, то  $\{a_n\}^{\infty}_{n=1}$ строго возрастающая.
	\item{} Если $(\forall n\in\N)\;a_{n+1} \geq a_{n}$, то  $\{a_n\}^{\infty}_{n=1}$ возрастающая (неубывающая).
	\item{} Если $(\forall n\in\N)\;a_{n+1} \leq a_{n}$, то  $\{a_n\}^{\infty}_{n=1}$ убывающая (невозрастающая).
	\item{} Если $(\forall n\in\N)\;a_{n+1}<a_{n}$, то  $\{a_n\}^{\infty}_{n=1}$ строго убывающая.
	\item{} Если последовательность строго возрастающая или строго убывающая, то она строго монотонна
	\item{} Если последовательность возрастающая или убывающая, то она монотонна
\end{enumerate}

\end{document}
