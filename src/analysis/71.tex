\documentclass{article}

\usepackage{defines}

\begin{document}

\tickettitle{71}{Неопределенный интеграл. Интегрирование методом подстановки.}

\theorem

$f$ --- определена на $(a;b)$ $\land$ $F$ --- её первообразная

$\varphi$ --- дифференцируема на $(c;d)$ $\land$ $\varphi'$ --- непрерывна на $(c;d)$ $\land$ $\varphi((c;d))\subset(a;b)$

Тогда на $(c;d)$:
\begin{align*}
	\int f(\varphi(t))\varphi'(t)dt=F(\varphi(t))+c
\end{align*}

\proof

Продифференцируем правую часть:
\begin{align*}
	 & I(x):=	F(\varphi(t))+c                                             \\
	 & I'(x)=F'(\varphi(t))\varphi'(t)=f(\varphi(t))\varphi'(t)\Rarr      \\
	 & \Rarr I(x)\text{ --- первообразная $f(\varphi(t))\varphi'(t)$}\qed
\end{align*}

\sectitle{Интегрирование методом подстановки}

\hd{Способ 1: $x=\varphi(t)$}
\begin{align*}
	 & I=\int f(x)dx                                                             \\
	 & x=\varphi(t)\Rarr dx=\varphi'(t)dt                                        \\
	 & I=\int f(\varphi(t))\varphi'(t)dt=F(\varphi(t))+c                         \\
	 & g(t):=f(\varphi(t))\varphi'(t)\land G'(t)=g(t)                            \\
	 & I=\int g(t)dt=G(t)+c\Rarr F(\varphi(t))=G(t)\Rarr F(x)=G(\varphi^{-1}(x))
\end{align*}

{\it Пример}
\begin{align*}
	 & I=\int\frac{dx}{\sqrt{x}+1}                                                                 \\
	 & x=t^{2}\Rarr t=\sqrt{x} \land dx=2tdt                                                       \\
	 & \begin{aligned}
		   I
		    & =\int\frac{2t}{t+1}dt=2\int\frac{t+1-1}{t+1}dt=2\left(\int dt-\int\frac{dt}{t+1}\right)= \\
		    & =2(t+\ln|t+1|)+c=2(\sqrt{x}+\ln|\sqrt{x}+1|)+c
	   \end{aligned}
\end{align*}

\pagebreak

\hd{Способ 2: $t=\omega(x)$}
\begin{align*}
	 & I=\int f(x)dx                                              \\
	 & t=\omega(x)\Rarr dt=\omega'(x)dx                           \\
	 & f(x)=g(\omega(x))\omega'(x)\land G'(t)=g(t)                \\
	 & I=\int f(x)dx=\int g(\omega(x))\omega'(x)dx=G(\omega(x))+c
\end{align*}

{\it Пример}
\begin{align*}
	 & I=\int\sin^{2}(x)\cos(x)dx                                \\
	 & t=\sin(x)\Rarr dt=\cos(x)dx                               \\
	 & I=\int t^{2}dt=\frac{1}{3}t^{3}+c=\frac{\sin^{3}(x)}{3}+c
\end{align*}

{\it Пример}
\begin{align*}
	 & I =\int\frac{\varphi'(x)}{\varphi(x)}dx    \\
	 & t=\varphi(x)\Rarr dt=\varphi'(x)dx         \\
	 & I=\int \frac{dt}{t}=\ln|t|=\ln|\varphi(x)|
\end{align*}

\end{document}
