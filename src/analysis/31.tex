\documentclass{article}

\usepackage{defines}

\begin{document}

\tickettitle{31}{Число $e$ как предел функции. Замечательные пределы, неопределенности вида $1^{\infty}$, $0^{0}$, $\infty^{0}$.}

\theorem
\begin{align*}
	\slim{x\to 0}(1+x)^{\frac{1}{x}}=e
\end{align*}

\proof

Возьмём $\{n_{k}\}\subset\N:\slimty n_{k}=+\infty$
\begin{align*}
	 & E(n):=\left(1+\frac{1}{n}\right)^{n},\;\slimty E(n)=e                                           \\
	 & \forall r\;\exists k:(\forall m>k)\;n_{m}>r                                                     \\
	 & \forall\eps>0\;\exists r:(\forall n>r)\;|E(n)-e|<\eps\land\exists k:(\forall m>k)\;n_{m}>r      \\
	 & \Rarr\forall\eps>0\;\exists k:(\forall m>k)\;|E(n_{m})-e|<\eps\Rarr \slim{k\to\infty}E(n_{m})=e
\end{align*}

Возьмём теперь $\{x_{k}\}\subset(0;1):\slim{k\to\infty}x_{k}=0$

Тогда определим $\{n_{k}\}\subset\N$:
\begin{align*}
	 & n_{k}:=\left\lfloor\frac{1}{x_{k}}\right\rfloor\Rarr n_{k}\leq \frac{1}{x_{k}}<n_{k}+1        \\
	 & \forall r>0\;\exists k:(\forall m>k)\;0<x_{m}<\frac{1}{r}\Rarr r<\frac{1}{x_{m}}<n_{m}+1\Rarr \\
	 & \Rarr\slim{k\to\infty}n_{k}=+\infty\Rarr \slim{k\to\infty}E(n_{k})=e
\end{align*}
\begin{align*}
	n_{k}\leq\frac{1}{x_{k}}<n_{k}+1
	 & \Rarr \frac{1}{n_{k}+1}<x_{k}\leq\frac{1}{n_{k}}\Rarr 1+\frac{1}{n_{k}+1}<1+x_{k}\leq 1+\frac{1}{n_{k}}\Rarr        \\
	 & \Rarr \left(1+\frac{1}{n_{k}+1}\right)^{n_{k}}<(1+x_{k})^{\frac{1}{x_{k}}}<\left(1+\frac{1}{n_{k}}\right)^{n_{k}+1}
\end{align*}
\begin{align*}
	 & \slim{k\to\infty}\left(1+\frac{1}{n_{k}+1}\right)^{n_{k}}=\slim{k\to\infty}\left(1+\frac{1}{n_{k}+1}\right)^{n_{k}+1}\slim{k\to\infty}\left(1+\frac{1}{n_{k}+1}\right)^{-1}=e \\
	 & \slim{k\to\infty}\left(1+\frac{1}{n_{k}}\right)^{n_{k}+1}=\slim{k\to\infty}\left(1+\frac{1}{n_{k}}\right)^{n_{k}}\slim{k\to\infty}\left(1+\frac{1}{n_{k}}\right)=e
\end{align*}

Тогда по принципу сжатой переменной:
\begin{align*}
	\slim{k\to\infty}(1+x_{k})^{\frac{1}{x_{k}}}=e\Rarr \slim{x\to 0+0}(1+x)^{\frac{1}{x}}=e
\end{align*}

\pagebreak

Рассмотрим теперь предел для $y\to 0-0$ с помощью замены $x=-y$
\begin{align*}
	 & \slim{y\to 0-0}(1+y)^{\frac{1}{y}}=\slim{x\to 0+0}(1-x)^{-\frac{1}{x}}=
	\slim{x\to 0+0}\left(\frac{1}{1-x}\right)^{\frac{1}{x}}=\slim{x\to 0+0}\left(\frac{1-x+x}{1-x}\right)^{\frac{1}{x}}             \\
	 & =\slim{x\to 0+0}\left(1+\frac{x}{1-x}\right)^{\frac{1}{x}-1+1}=\slim{x\to 0+0}\left(1+\frac{x}{1-x}\right)^{\frac{1-x}{x}+1} \\
\end{align*}

Сделаем замену:
\begin{align*}
	 & z=\frac{x}{1-x}\Rarr\slim{x\to 0+0}z=0+0                                                         \\
	 & \slim{y\to 0-0}(1+y)^{\frac{1}{y}}=\slim{x\to 0+0}\left(1+\frac{x}{1-x}\right)^{\frac{1-x}{x}+1}
	=\slim{z\to 0+0}(1+z)^{\frac{1}{z}+1}=\slim{z\to 0+0}(1+z)^{\frac{1}{z}}\slim{z\to 0+0}(1+z)=e
\end{align*}

Тогда можно сделать вывод:
\begin{align*}
	\slim{x\to 0+0}(1+x)^{\frac{1}{x}}=e\land\slim{x\to 0-0}(1+x)^{\frac{1}{x}}=e\Rarr\slim{x\to 0}(1+x)^{\frac{1}{x}}=e\qed
\end{align*}

\sectitle{Замечательные пределы}
\begin{enumerate}
	\item{}
	\begin{enumerate}[label=\theenumi.\arabic*.]
		\setcounter{enumii}{-1}
		\item{}
		\begin{align*}
			\slim{x\to 0}\frac{\sin(x)}{x}=1\text{ (по билету 30) }
		\end{align*}
		\item{}
		\begin{align*}
			\slim{x\to 0}\frac{\tg(x)}{x}=\slim{x\to 0}\frac{\tg(x)}{x}=\slim{x\to 0}\frac{1}{\cos(x)}\slim{x\to 0}\frac{\sin(x)}{x}=1
		\end{align*}
		\item{}
		\begin{align*}
			 & L=\slim{x\to 0}\frac{\arcsin(x)}{x} \\
			 & x=\sin(y)\Rarr x\to 0\sim y\to 0    \\
			 & L=\slim{y\to 0}\frac{y}{\sin(y)}=1
		\end{align*}
		\item{}
		\begin{align*}
			 & L=\slim{x\to 0}\frac{\arctg(x)}{x} \\
			 & x=\tg(y)\Rarr x\to 0\sim y\to 0    \\
			 & L=\slim{y\to 0}\frac{y}{\tg(y)}=1
		\end{align*}
	\end{enumerate}

	\pagebreak

	\item{}
	\begin{enumerate}[label=\theenumi.\arabic*.]
		\setcounter{enumii}{-1}
		\item{}
		\begin{align*}
			\slim{x\to 0}(1+x)^{\frac{1}{x}}=e
		\end{align*}
		\item{}
		\begin{align*}
			 & L=\slim{x\to \infty}\left(1+\frac{1}{x}\right)^{x} \\
			 & z=\frac{1}{x}\Rarr z\to 0\sim x\to\infty           \\
			 & L=\slim{z\to 0}(1+z)^{\frac{1}{z}}=e
		\end{align*}
	\end{enumerate}

	\item{}
	\begin{enumerate}[label=\theenumi.\arabic*.]
		\setcounter{enumii}{-1}
		\item{}
		\begin{align*}
			\slim{x\to 0}\frac{\log_{a}(x+1)}{x}=\slim{x\to 0}\log_{a}\left[\left(1+x\right)^{\frac{1}{x}}\right]=\log_{a}e=\frac{1}{\ln a}
		\end{align*}
		\item{}
		\begin{align*}
			\slim{x\to 0}\frac{\ln(x+1)}{x}=\frac{1}{\ln e}=1
		\end{align*}
	\end{enumerate}

	\item{}
	\begin{enumerate}[label=\theenumi.\arabic*.]
		\setcounter{enumii}{-1}
		\item{}
		\begin{align*}
			 & L=\slim{x\to 0}\frac{a^{x}-1}{x}             \\
			 & y=a^{x}-1\Rarr y\to 0\sim x\to 0             \\
			 & L=\slim{y\to 0}\frac{y}{\log_{a}(y+1)}=\ln a
		\end{align*}
		\item{}
		\begin{align*}
			\slim{x\to 0}\frac{e^{x}-1}{x}=\ln e=1
		\end{align*}
	\end{enumerate}

	\item{}
	\begin{enumerate}[label=\theenumi.\arabic*.]
		\setcounter{enumii}{-1}
		\item{}
		\begin{align*}
			 & L=\slim{x\to 0}\frac{(1+x)^{\mu}-1}{x}                                                                \\
			 & y=(1+x)^{\mu}-1\Rarr y\to 0\sim x\to 0                                                                \\
			 & L=\slim{x\to 0}\frac{y}{x}=\slim{x\to 0}\frac{y}{\ln(y+1)}\frac{\ln(y+1)}{\ln(x+1)}\frac{\ln(x+1)}{x}
			=\slim{y\to 0}\frac{y}{\ln(y+1)}\slim{x\to 0}\frac{\ln(y+1)}{\ln(x+1)}\slim{x\to 0}\frac{\ln(x+1)}{x}=   \\
			 & =\slim{x\to 0}\frac{\ln(y+1)}{\ln(1+x)}=\slim{x\to 0}\frac{\ln((1+x)^{\mu}-1+1)}{\ln(1+x)}
			=\slim{x\to 0}\frac{\mu\ln(1+x)}{\ln(1+x)}=\mu
		\end{align*}
	\end{enumerate}
\end{enumerate}

\pagebreak

\sectitle{Раскрытие неопределенностей со степенями}

Раскрытие происходит через сведение к неопределенности вида $0\cdot\infty$
\begin{enumerate}
	\item{}$u\to 1$, $v\to \infty$
	\begin{align*}
		1^{\infty}=u^{v}=e^{v\ln u}=e^{\infty\cdot 0}
	\end{align*}
	\item{}$u\to 0$, $v\to 0$
	\begin{align*}
		0^{0}=u^{v}=e^{v\ln u}=e^{0\cdot\infty}
	\end{align*}
	\item{}$u\to\infty$, $v\to 0$
	\begin{align*}
		\infty^{0}=u^{v}=e^{v\ln u}=e^{0\cdot\infty}
	\end{align*}
\end{enumerate}

\end{document}
