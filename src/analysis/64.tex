\documentclass{article}

\usepackage{defines}

\begin{document}

\tickettitle{64}{Вычисление высших производных обратной функции с помощью дифференциалов.}

\theorem

$y=f(x)$

$g$ --- функция, обратная к $f$, $x=g(y)$
\begin{align}
	(\forall n\geq 2)\;d^{n}y=\left(R_{n}(y)+f'(x)g^{(n)}(y)\right)dy^{n}\label{64:form}
\end{align}

Причём $R_{n}(y)$ зависит только от производных $g$ порядка $<n$ и производных $f$ порядка $\leq n$

\proof

$g$ --- обратная к $f$ $\Rarr$ $y=f(x)=f(g(y))$

{\it Индукция:} $P(n)$ --- верность формулы $\eqref{64:form}$ для $n$
\begin{enumerate}
	\item{}$P(2)$
	\begin{align*}
		 & d^{2}y=d(dy)=d(f'(x)g'(y)dy)=\left[f^{(2)}(x)(g'(y))^{2}+f'(x)g^{(2)}(y)\right]dy^{2} &  & R_2(y)=f^{(2)}(x)(g'(y))^{2}
	\end{align*}
	\item{}$P(n)\Rarr P(n+1)$
	\begin{align*}
		d^{n+1}y & =d(d^{n}y)=d\left[R_{n}(y)+f'(x)g^{(n)}(y)\right]dy^{n-1}= \left[R'_{n}(y)+f''(x)g'(y)g^{(n)}(y)+f'(x)g^{(n+1)}(y)\right]dy^{n}
	\end{align*}

	Определим $R_{n+1}(y)$:
	\begin{align*}
		R_{n+1}(y):=R'_{n}(y)+f''(x)g'(y)g^{(n)}(y)
	\end{align*}

	Заметим, что $R_{n+1}(y)$ зависит только от производных $g$ порядка $<n+1$ и\\
	производных $f$ порядка $\leq n+1$

	Тогда получаем:
	\begin{align*}
		d^{n+1}y=\left(R_{n+1}(y)+f'(x)g^{(n+1)}(y)\right)dy^{n+1}\Rarr P(n+1)\qed
	\end{align*}
\end{enumerate}

\algorithm{вычисления высших производных обратной функции}

{\it Задача:} найти $g^{(n)}(y)$

\begin{enumerate}
	\item{}Вычислим $d^{(m)}y$ для $m=\overline{1,n}$ и приведём их к форме $\eqref{64:form}$

	$y$ --- независимая переменная $\Rarr$ $(\forall m\geq 2)\;d^{m}y=0$, тогда для $m\geq 2$
	\begin{align}
		R_{m}(y)+f'(x)g^{(m)}(y)=0\Rarr g^{(m)}(y)=\frac{-R_{m}(y)}{f'(g(y))}\label{64:gm}
	\end{align}

	\item{}Вычислим $g'(y)$ по формуле $g'(y)=\frac{1}{f'(g(y))}$, а $g^{(m)}(y)$ для $m=\overline{2,n}$ по формуле $\eqref{64:gm}$

	Для каждого из этих шагов необходимы только производные $g$ до порядка $m-1$\\
	и производные $f$ до порядка $m$
\end{enumerate}

\end{document}
