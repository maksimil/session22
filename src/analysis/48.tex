\documentclass{article}

\usepackage{defines}

\begin{document}

\tickettitle{48}{Теорема о производной обратной функции {\it Билет не просмотрен Ксюшей, но проверен Артёмом}}

\theorem

$y = f(x)$ --- взаимно-однозначная функция, дифференцируема на $[a,b]$ и $f([a, b]) = [c, d]$, тогда обратная функция $x = g(y)$ -- дифференцируема на [c, d] и $\frac{dx}{dy} = \frac{1}{\frac{dy}{dx}},$ если $\frac{dy}{dx} \ne 0$

\proof

$y = f(x)$ --- взаимно-однозначна и непрерывна (т.к. $f(x)$ --- дифференцируемая) $\Rarr$ обратная функция $x = g(y)$ --- непрерывной и взаимно-однозначна. $g(y)$ определена либо на сегменте $[f(a), f(b)]$ ($f$ - возрастающая), либо на $[f(b), f(a)]$ ($f$ - убывающая) \\

$\sqsupset$ для данного $x: \ \ k = f(x+h) - f(x) = f(x+h) - y \Rarr f(x+h) = k + y \Rarr x + h = g(k+y) \ \Rightarrow \ h = g(y+k) - g(y)$ \\
$\sqsupset h = h(k)$ \\

$g(y)$ -- непрерывна $\ \Rightarrow \ \lim_{k \to 0} h(k) = 0$ \\
$h \ne 0,$ если $k \ne 0$ (в силу взаимной-однозначности) \\

Рассмотрим $\frac{dx}{dy} = \frac{dg(y)}{dy} = \lim_{k \to 0} \frac{g(y+k) - g(y)}{k} = \lim_{k \to 0} \frac{h(k)}{f(x+h) - f(x)}$ (по теореме о пределе суперпозиции) $= \lim_{h \to 0} \frac{h}{f(x+h) - f(x)} = \frac{1}{\frac{dy}{dx}}\qed$
\end{document}
