\documentclass{article}

\usepackage{defines}

\begin{document}

\tickettitle{32}{Непрерывность функции в точке, непрерывность справа и слева в точке, непрерывность функции на множестве. Определение непрерывности на языке "$\eps-\delta$". Примеры.}

\define{непрерывности функции в точке}

$f$ ---  определена в окрестности $x_0\in\R$

$f$ называется непрерывной в $x_0$, если
\begin{align*}
	\slim{x \to x_0}f(x)=f(x_0)
\end{align*}

{\it Альтернативные определения}
\begin{align*}
	 & \slim{x\to x_0}f(x)-f(x_0)=0 &  & \slim{h\to 0}f(x+h)-f(x)=0
\end{align*}

\define{непрерывности справа и слева в точке}

$f$ ---  определена на $(a;x_0]$

$f$ называется непрерывной в $x_0$ слева, если
\begin{align*}
	\slim{x\to x_0-0}f(x)=f(x_0)
\end{align*}

$f$ --- определена на $[x_0;a)$

$f$ называется непрерывной в $x_0$ справа, если
\begin{align*}
	\slim{x\to x_0+0}f(x)=f(x_0)
\end{align*}

\define{непрерывность функции на множестве}

$f$ --- определена на промежутке $X$ с концами $a$ и $b$ ($a<b$)

$f$ называется непрерывной на этом промежутке, если она непрерывна в каждой его внутренней точке и
если точка $a \in X$, то под непрерывностью в ней понимают непрерывность справа,
если точка $b \in X$, то под непрерывностью в ней понимают непрерывность слева

\define{непрерывности на языке "$\eps-\delta$"\par}

$f$ --- непрерывна в $x_0\in\R$, если
\begin{align*}
	\forall\eps>0\;\exists\delta>0:(\forall x\in\R:|x-x_0|<\delta)\;|f(x)-f(x_0)|<\eps
\end{align*}

\theorem

$f$, $g$ --- непрерывны на $X$

$f\pm g$, $fg$ --- непрерывны на $X$ и $f/g$ --- непрерывна на $X\setminus\setdef{x\in X}{g(x)=0}$

\proof

По алгебраическим свойствам предела$\qed$

\pagebreak

\sectitle{Примеры непрерывных функций}

\begin{enumerate}
	\item{}Многочлены

	\theorem

	$f(x)=x^{n},\;n\in\N$ --- непрерывна на $\R$

	\proof

	{\it Индукция:} $T(n)$ --- верность теоремы для $n$
	\begin{enumerate}
		\item{}$T(1)$
		\begin{align*}
			\slim{x\to x_0}f(x)=\slim{x\to x_0}x=x_0\Rarr T(1)
		\end{align*}
		\item{}$T(n)\Rarr T(n+1)$
		\begin{align*}
			\slim{x\to x_0}f(x)=\slim{x\to x_0}x^{n+1}=\slim{x\to x_0}x^{n}\slim{x\to x_0}x=x_0^{n}x_0=x_0^{n+1}\Rarr T(n+1)\qed
		\end{align*}
	\end{enumerate}

	\theorem

	Многочлен степени $n$ непрерывен на $\R$

	\proof
	\begin{align*}
		 & P_{n}(x)=\sum_{k=0}^{n}a_{k}x^{k} &  & \slim{x\to x_0}P_{n}(x)=\slim{x\to x_0}\sum_{k=0}^{n}a_{k}x^{k}=
		\sum_{k=0}^{n}a_{k}\slim{x\to x_0}x^{k}=\sum_{k=0}^{n}a_{k}x_0^{k}=P_{n}(x_0)\qed
	\end{align*}

	\item{}Рациональные выражения

	\theorem

	$P_{n}/Q_{m}$ --- рациональное выражение, где $P_{n}$ --- многочлен степени $n$, $Q_{m}$ --- многочлен степени $m$

	$P_{n}/Q_{m}$ --- непрерывно на $\R\setminus\setdef{x\in\R}{Q_{m}(x)=0}$

	\proof
	\begin{align*}
		\slim{x\to x_0}\frac{P_{n}(x)}{Q_{m}(x)}=\frac{\slim{x\to x_0}P_{n}(x)}{\slim{x\to x_0}Q_{m}(x)}=\frac{P_{n}(x_0)}{Q_{m}(x_0)}\qed
	\end{align*}

	\item{}Тригонометрические функции

	\theorem

	$f(x)=\sin(x)$ ---  непрерывна на $\R$

	\proof
	\begin{align*}
		 & |\sin(x+h)-\sin(x)|=2\left|\sin\frac{h}{2}\right|\cdot\left|\cos\left(x+\frac{h}{2}\right)\right|\leq 2\left|\sin\frac{h}{2}\right|\xrightarrow{h\to 0} 0\Rarr \\
		 & \Rarr|\sin(x+h)-\sin(x)|\xrightarrow{h\to 0} 0
	\end{align*}

	\pagebreak

	Аналогично непрерывен и $\cos(x)$

	$\tg(x), \ctg(x)$ --- непрерывны на своих областях определения по алгебраическим\\
	свойствам предела

	\item{}$e^{x}$

	\theorem

	$e^{x}$ --- непрерывна на $\R$

	\proof
	\begin{align*}
		\slim{h\to 0}(e^{x+h}-e^{x})=\left(\slim{h\to 0}e^{x}e^{h}\right)-e^{x}=e^{x}\left(\slim{h\to 0}e^{h}\right)-e^{x}=0\qed
	\end{align*}

\end{enumerate}

\end{document}
