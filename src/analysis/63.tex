\documentclass{article}

\usepackage{defines}

\begin{document}

\tickettitle{63}{Дифференциалы высших порядков. Формула Лейбница для дифференциалов. Неинвариантность формы дифференциалов высших порядков.}

\define{дифференциалов высших порядков}

Дифференциал $dy$ функции $y=f(x)$ по отношению к приращению $h$ --- функция относительно $x$

Дифференциалом этой функции в $x$ по отношению к тому же приращению $h$ --- дифференциал второго порядка функции $y=f(x)$ в $x$ по отношению к приращению $h$

Обозначение: $d^{2}y$, $d^{2}f(x)$

Дифференциал функции $d^{n-1}y$ --- дифференциал $n$-го порядка и обозначается $d^{n}y$

$d^{n}y=d(d^{n-1}y)$

\theorem
\begin{align}
	(\forall n\in\N)\;d^{n}y=y^{(n)}dx^{n} \label{63:formula}
\end{align}

\proof

{\it Индукция:} $P(n)=(d^{n}y=y^{(n)}dx^{n})$
\begin{enumerate}
	\item{}$dy=y'dx\Rarr P(1)$
	\item{}$P(n-1)\Rarr P(n)$
	\begin{align*}
		d^{n}y=d(d^{n-1}y)=d(y^{(n-1)}dx^{n-1})=y^{(n)}dxdx^{n-1}=y^{(n)}dx^{n}\Rarr P(n)\qed
	\end{align*}
\end{enumerate}

{\it Замечание: }

Из теоремы следует формула $f^{(n)}(x)=\frac{d^{n}y}{dx^{n}}$

{\it Замечание: }

Форма дифференциала второго порядка неинвариантна, то есть формула $\eqref{63:formula}$ не верна при $d^{2}x\neq 0$

$y=f(x)$, $x=\varphi(t)$
\begin{align*}
	d^{2}y=d(dy)=d(f'_{x}\varphi'_{t}dt)=f''_{x^{2}}dx\varphi'_{t}dt+f'_{x}\varphi''_{t^{2}}dt^{2}=f''_{x^{2}}dx^{2}+f'_{x}d^{2}x\neq f''_{x^{2}}dx^{2}
\end{align*}

Из-за $d^{2}x\neq 0$ будут неинвариантны и формы порядков $>2$

\theorem[(формула Лейбница для дифференциалов)]
\begin{align*}
	d^{n}(uv)=\sum_{k=0}^{n}C_{n}^{k}(d^{n-k}u)(d^{k}v)\text{, где }d^{0}u=u
\end{align*}

\proof
\begin{align*}
	d^{n}(uv)=(uv)^{(n)}dx^{n}=\sum_{k=0}^{n}C_{n}^{k}u^{(n-k)}v^{(k)}dx^{n}=\sum_{k=0}^{n}C_{n}^{k}(u^{(n-k)}dx^{n-k})(v^{(k)}dx^{k})=\sum_{k=0}^{n}C_{n}^{k}(d^{n-k}u)(d^{k}v)\qed
\end{align*}

\end{document}
