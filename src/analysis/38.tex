\documentclass{article}

\usepackage{defines}

\begin{document}

\tickettitle{38}{Теорема о строгой монотонности непрерывной взаимно однозначной функции}

\theorem

$f$ --- непрерывна и взаимно однозначна на $[a;b]$ $\Rarr$ $f$ --- строго монотонна на $[a;b]$

\proof

$f$ --- взаимно однозначна на $[a;b]$ $\Rarr f(a)\neq f(b)$

\begin{enumerate}
	\item$f(a)<f(b)$

	Докажем, что если $f$ --- непрерывна и взаимно однозначна на $[a;b]$, то
	\begin{align*}
		(\forall x\in[a;b])\;f(x)\in[f(a);f(b)]
	\end{align*}

	Пойдем от противного. Пусть $x\in[a;b]$ и $f(x)<f(a)$ $\eqref{38:nleft}$ или $f(b)<f(x)$ $\eqref{38:nright}$:
	\begin{align}
		 & \llet f(x)<f(a)<f(b)\Rarr \exists x'\in(x;b):f(x')=f(a)\Rarr x'=a\text{, но }a\leq x<x'<b\label{38:nleft}   \\
		 & \llet f(a)<f(b)<f(x)\Rarr \exists x'\in(a;x):f(x')=f(b)\Rarr x'=b\text{, но }a< x'<x\leq b\label{38:nright}
	\end{align}
	\begin{align*}
		\eqref{38:nleft}\land\eqref{38:nright}\Rarr f(x)\in[f(a);f(b)]
	\end{align*}

	$f$ --- непрерывна и взаимно однозначна на $(\forall x\in[a;b])\;[x;b]\subset[a;b]$.

	Тогда из доказанного следует:
	\begin{align}
		 & (\forall x\in[a;b],x'\in[x;b])\;f(x')\in[f(x);f(b)]\notag      \\
		 & (\forall x,x'\in[a;b]:x\leq x')\;f(x)\leq f(x')\label{38:nstr} \\
		 & (\forall x,x'\in[a;b]:x\neq x')\;f(x)\neq f(x')\label{38:neq}
	\end{align}
	\begin{align*}
		\eqref{38:nstr}\land\eqref{38:neq}  \Rarr
		(\forall x,x'\in[a;b]:x<x')\;f(x)<f(x')\qed\notag
	\end{align*}

	\item$f(a)>f(b)$

	Рассмотрим $g(x):=-f(x)$:
	\begin{align*}
		f(a)>f(b)\Rarr -f(a)<-f(b)\Rarr g(a)<g(b)
	\end{align*}

	Таким образом, $g$ --- непрерывна и взаимно однозначна на $[a;b]$ $\land$ $g(a)<g(b)$.

	Тогда из п.1 следует:
	\begin{align*}
		(\forall x,x'\in[a;b])\;g(x)<g(x')\Rarr-f(x)<-f(x')\Rarr f(x)>f(x')\qed
	\end{align*}


\end{enumerate}

\end{document}
