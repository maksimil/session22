\documentclass{article}

\usepackage{defines}

\begin{document}

\tickettitle{54}{Приращение дифференцируемой функции. Теорема о дифференцировании суперпозиции.}

\theorem
\begin{align*}
	\exists f'(x)\in\R\Rarr \Delta f(x)=f'(x)h+\alpha(h)h,\;\slim{h\to 0}\alpha(h)=0
\end{align*}

\proof

Определим $\alpha(h)$:
\begin{align*}
	\alpha(h):=\frac{\Delta f(x)}{h}-f'(x)\Rarr \Delta f(x) = f'(x)h+\alpha(h)h
\end{align*}

Покажем, что $\alpha(h)\to 0$:
\begin{align*}
	\slim{h\to 0}\alpha(h)=\left(\slim{h\to 0}\frac{\Delta f(x)}{h}\right)-f'(x)=f'(x)-f'(x)=0\qed
\end{align*}

\theorem

$f(x) = g(\phi(x))$ --- суперпозиция (сложная функция)

$\phi (x)$ --- дифференцируема в $x_0$

$g(t)$ --- дифференцируема в $\phi (x_0)$
\begin{align*}
	f'(x)=g'(\phi(x))\phi'(x)
\end{align*}

\proof

$g$ --- дифференцируема в $\phi(x_0)$
\begin{align*}
	 & t=\phi(x),\;t_0=\phi(x_0)                                               \\
	 & \Delta g(t_0) = g(t_0 + \Delta t) - g(t_0) =g'(t_0)\Delta t+o(\Delta t)
\end{align*}

Рассмотрим $f'(x)$:
\begin{align*}
	f'(x) & =\slim{\Delta x\to 0}\frac{g(\phi(x_0+\Delta x))-g(\phi(x_0))}{\Delta x}=\slim{\Delta x\to 0}\frac{g(\phi(x_0)+\Delta \phi(x_0))-g(\phi(x_0))}{\Delta x}= \\
	      & =\slim{\Delta x\to 0}\frac{g'(\phi(x_0))\Delta \phi(x_0)+o(\Delta\phi(x_0))}{\Delta x}
	=\slim{\Delta x\to 0}\frac{g'(\phi(x_0))\Delta \phi(x_0)+o(\Delta\phi(x_0))}{\Delta \phi(x_0)}\frac{\Delta\phi(x_0)}{\Delta x}
\end{align*}

$\phi$ --- непрерывна в $x_0$
\begin{align*}
	 & \Delta \phi(x_0)=\phi(x_0+\Delta x)-\phi(x_0)                                                               \\
	 & \slim{\Delta x\to 0}\Delta\phi(x_0)=0\Rarr \slim{\Delta x\to 0}\frac{o(\Delta\phi(x_0))}{\Delta\phi(x_0)}=0
\end{align*}

Вернёмся к $f'(x)$:
\begin{align*}
	f'(x) & =\slim{\Delta x\to 0}\left(g'(\phi(x_0))+\frac{o(\Delta\phi(x_0))}{\Delta \phi(x_0)}\right)\slim{\Delta x\to 0}\frac{\Delta\phi(x_0)}{\Delta x}=g'(\phi(x_0))\phi'(x_0)\qed
\end{align*}

\end{document}
