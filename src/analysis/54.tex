\documentclass{article}

\usepackage{defines}

\begin{document}

\tickettitle{54}{Приращение дифференцируемой функции. Теорема о дифференцировании суперпозиции.}

\define{приращения дифференцируемой функции}

$\llet f(x)$ - дифференцируемая функция, тогда $\Delta f(x) = f(x+h) - f(x)$  --- приращение(первая разность) f(x) в $x_0$ при приращении h аргумента x.


\define{дифференцируемости}

Если приращение функции $y=f(x)$ в точке $x_0$ представимо в виде $\Delta y=A\Delta x+o(\Delta x)$, то она дифференцируема в этой точке.

$A\Delta x$, $dy$ --- дифференциал $f$ в $x_0$.

$f(x)=x\Rarr dy=\Delta x\Rarr dx=\Delta x$

\theorem
\begin{align*}
	f\text{ --- дифференцируема в $x_0$} \Lrarr \exists f'(x_0)\land dy=f'(x_0)dx
\end{align*}

\onlyif

$f$ --- дифференцируема, значит $\Delta y=A\Delta x+o(\Delta x)$
\begin{align*}
	 & \slim{\Delta x\to 0}\frac{\Delta y}{\Delta x}=\slim{\Delta x\to 0}\frac{A\Delta x+o(\Delta x)}{\Delta x}=A\Rarr \exists f'(x_0)=A \Rarr \\
	 & \Rarr \Delta y=f'(x_0)\Delta x+o(\Delta x)\Rarr dy=f'(x_0)dx\qed
\end{align*}

\enough
\begin{align*}
	 & \exists f'(x_0)\Rarr \exists\slim{\Delta x\to 0}\frac{\Delta y}{\Delta x}
	\Rarr \frac{\Delta y}{\Delta x}=f'(x_0)+\eps(\Delta x)\text{, где }\slim{\Delta x\to 0}\eps(\Delta x)=0\Rarr               \\
	 & \Rarr \Delta y=f'(x_0)\Delta x+\eps(\Delta x)\Delta x=f'(x_0)\Delta x+o(\Delta x)\Rarr f\text{ --- дифференцируема}\qed
\end{align*}

\theorem

$\llet f(x)$ имеет конечную $f'(x)$ в $x$ (иными словами дифференцируема в этой точке), тогда приращение $\Delta f = f'(x) h  +\alpha (h) h$, где $\slim{h\to 0}\alpha (h) = 0$ (1).

(1) имеет место для всех достаточно малых h.

\proof

По условию $\exists$ конечная $f'(x) \Rarr \exists$ $\slim{h\to 0}\frac{\Delta f}{h} = f'(x)$

$\llet \alpha (h) = \cfrac{\Delta f}{h} - f'(x)$, тогда $\Delta f = f'(x)h + \alpha(h) h \Rarr \slim{h\to 0}\alpha (h) = 0\qed$

\theorem

$\llet f(x) = g(\phi(x))$ --- суперпозиция(сложная функция). $\llet \phi (x)$ --- дифференцируема в $x_0$, $g(\phi)$ --- дифференцируема в $\phi (x_0)$, тогда $f(x)$ --- дифференцируема в $x_0$ : $\cfrac{d f(x)}{dx} = \cfrac{d g}{d \phi} \cdot \cfrac{d \phi}{d x}$

\proof

Т.к. $g$ и $\phi$ - непрерывны в $\phi(x_0)$ и $x_0$ соответственно,тогда запишем, чему равны приращения $\phi(x)$ и $g$ соответственно:
\begin{align*}
	 & \Delta \phi = \phi(x_0+h) - \phi(x_0)                   \\
	 & \Delta g = g(\phi + \Delta \phi) - g(\phi)              \\
	 & \slim{h\to 0}\Delta \phi = 0, \slim{h\to 0}\Delta g = 0
\end{align*}

Тогда по предыдущей теореме $\Delta g(\phi) = g'(\phi)\Delta g +\alpha (\Delta \phi) \Delta \phi$, при $&\slim{\Delta \phi\to 0}\alpha (\Delta \phi) = 0$

Рассмотрим $f'(x) = \slim{h\to 0}\frac{\Delta f}{h} = \slim{h\to 0}\frac{\Delta g(f)}{h} = \slim{h\to 0}\frac{\Delta g(f)}{\Delta \phi}\cdot\cfrac{\Delta\phi}{h} = \slim{\Delta \phi\to 0}[g'(\phi) + \alpha (\Delta \phi)] \cdot \slim{h\to 0}\frac{\Delta \phi}{h} = g'(\phi) \cdot \phi'(x) \qed$

\end{document}
