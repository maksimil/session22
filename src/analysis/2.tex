\documentclass{article}

\usepackage{defines}

\usepackage{enumitem}

\usepackage{tikz}
\usetikzlibrary{decorations.pathreplacing, tikzmark, calc}

\begin{document}

\tickettitle{2}{Множество вещественных чисел, аксиомы вещественных чисел.}

\define{Абелевой группы}

Множество $X$ и оператор $\odot$ образуют абелеву группу, если:

\begin{enumerate}[label=\Roman*.]
	\item$(\forall a, b\in X)\;a\odot b\in X$
	\item{}Ассоциативность: $(\forall a,b,c\in X)\;(a\odot b)\odot c=a\odot(b\odot c)$
	\item{}Существование нейтрального элемента: $\exists e\in X:(\forall a\in X)\;a\odot e=a$
	\item{}Существование обратного элемента: $(\forall a\in X)\;\exists a^{-1}\in X:a\odot a^{-1}=e$
	\item{}Коммутативность: $(\forall a, b\in X)\;a\odot b=b\odot a$
\end{enumerate}

\define{множества вещественных чисел 1}

Аксиомы $\R$:

\begin{enumerate}[label=\Roman*.]
	\item\tikzmark{ax1}$(\forall a, b\in\R)\;a+b\in\R$
	\item$(\forall a,b,c\in\R)\;(a+b)+c=a+(b+c)$
	\item$\exists 0\in\R:(\forall a\in\R)\;a+0=a$
	\item$(\forall a\in\R)\;\exists (-a)\in\R:a+(-a)=0$
	\item\tikzmark{ax5}$(\forall a, b\in\R)\;a+b=b+a$
	\item\tikzmark{ax6}$(\forall a, b\in \R)\;a\cdot b\in \R$
	\item$(\forall a,b,c\in \R)\;(a\cdot b)\cdot c=a\cdot(b\cdot c)$
	\item$\exists 1\in \R:(\forall a\in \R)\;a\cdot 1=a$
	\item$(\forall a\in \R:a\neq 0)\;\exists a^{-1}\in \R:a\cdot a^{-1}=1$
	\item\tikzmark{ax10}$(\forall a, b\in \R)\;a\cdot b=b\cdot a$
	\item$(\forall a,b,c\in\R)\;(a+b)\cdot c=a\cdot c+b\cdot c$ --- Дистрибутивность $"\cdot"$ на $"+"$
	\item\tikzmark{ax12}$(\forall a,b\in\R)\;(a>b)\lor (a<b)\lor (a=b)$
	\item$(\forall a,b,c\in\R)\;(a>b)\land(b>c)\Rarr (a>c)$
	\item$(\forall a,b,c\in\R)\;a>b\Rarr a+c>b+c$
	\item\tikzmark{ax15}$(\forall a,b,c\in\R \land c>0)\;a>b\Rarr a\cdot c>b\cdot c$
	\item$(\forall a\in\R)\;\exists n\in\N:n>a$ --- Аксиома Архимеда
	\item$(\forall X,Y\subset\R:(x\leq y\;\forall x\in X,y\in Y))\;
		\exists a\in\R: x\leq a\leq y\;\forall x\in X,y\in Y$ --- Аксиома полноты
\end{enumerate}

Непустое множество, соответствующее 17 аксиомам называется множеством действительных чисел.

\define{множества вещественных чисел 2}

Непустое множество, состоящее из рациональных чисел вида $\cfrac{p}{q}$, где $p,q\in\mathbb{Q}$,
и иррациональных чисел~---~чисел, представимых в виде бесконечных непериодических десятичных дробей.

\newcommand{\markax}[3]{
	\draw [decorate, decoration={brace}] ($({pic cs:#1}) + (19em, 1em)$) -- ($({pic cs:#2}) + (19em,0)$)
	node [midway, right] {#3};
}
\begin{tikzpicture}[remember picture, overlay]
	\markax{ax1}{ax5}{$\R$ образует с $"+"$ абелеву группу}
	\markax{ax6}{ax10}{$\R\setminus\{0\}$ образует с $"\cdot"$ абелеву группу}
	\markax{ax12}{ax15}{Упорядоченность $\R$}
\end{tikzpicture}

\end{document}
