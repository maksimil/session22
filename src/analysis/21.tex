\documentclass{article}

\usepackage{defines}

\begin{document}

\tickettitle{21}{Неопределённые выражения, сравнение порядков бесконечно малых и бесконечно больших величин, главная часть бесконечно большой и бесконечно малой величин. {\it Билет не проверен}}

\define{Неопределённые выражения - выражения, предел которых не может быть определён.}

Типы неопределённостей:
\begin{enumerate}
	\item{}$\frac{f(x)}{g(x)}$, где $\slim{} f(x)=\slim{}g(x)=0$
	\item{}$\frac{f(x)}{g(x)}$, где $\slim{}f(x)=\slim{}g(x)=\infty$
	\item{}$f(x)g(x)$, где $\slim{}f(x)=0$, а $\slim{}g(x)=\infty$
	\item{}$f(x)-g(x)$, где $\slim{}f(x)=\slim{}g(x)=\infty$
	\item{}\{$f(x)\}^{g(x)}$, где $\slim{}f(x)=1$, а $\slim{}g(x)=\infty$
	\item{}\{$f(x)\}^{g(x)}$, где $\slim{} f(x)=\slim{}g(x)=0$
	\item{}\{$f(x)\}^{g(x)}$, где $\slim{}f(x)=\infty$, а $\slim{}g(x)=0$
\end{enumerate}

Cравнение порядков бесконечно малых и бесконечно больших величин.

Пусть $\exists \{x_n\}$ и $\{y_n\}$ - две бесконечно малые
\begin{enumerate}
	\item{}Если $\nexists \slim{n \to \infty}\frac{x_n}{y_n}$, то последовательности $\{x_n\}$ и $\{y_n\}$ несравнимы.
	\item{}Если $\slim{n \to \infty}\frac{x_n}{y_n}=p\neq 0$, то последовательности $\{x_n\}$ и $\{y_n\}$ одного порядка малости $x_n=O(y_n);y_n=O(x_n)$
	\item{}Если $ \slim{n \to \infty}\frac{x_n}{y_n}=1$, то последовательности $\{x_n\}$ и $\{y_n\}$ эквивалентны.
	\item{}Если $ \slim{n \to \infty}\frac{x_n}{y_n}=0$, то $x_n$ величина большего порядка малости, чем $y_n$.
	\item{}Если $ \slim{n \to \infty}\frac{x_n}{y_n}=\infty$, то $y_n$ величина большего порядка малочти, чем $x_n$.

\end{enumerate}

Пусть $\exists \{x_n\}$ и $\{y_n\}$ - две бесконечно большие
\begin{enumerate}
	\item{}Если $\nexists \slim{n \to \infty}\frac{x_n}{y_n}$, то последовательности $\{x_n\}$ и $\{y_n\}$ несравнимы.
	\item{}Если $\slim{n \to \infty}\frac{x_n}{y_n}=p\neq 0$, то последовательности $\{x_n\}$ и $\{y_n\}$ одного порядка $x_n=O(y_n);y_n=O(x_n)$
	\item{}Если $ \slim{n \to \infty}\frac{x_n}{y_n}=1$, то последовательности $\{x_n\}$ и $\{y_n\}$ эквивалентны.
	\item{}Если $ \slim{n \to \infty}\frac{x_n}{y_n}=\infty$, то $x_n$ величина большего порядка малочти, чем $y_n$.
	\item{}Если $ \slim{n \to \infty}\frac{x_n}{y_n}=0$, то $y_n$ величина большего порядка малости, чем $x_n$.
\end{enumerate}

\define{}

Пусть  $\alpha(x)$ и $\beta (x)$ функции, определенные в некоторой проколотой окрестности точки $x_0$. Если функция  $\beta(x)$ представляема в виде
\begin{align*}
	\beta(x) = \alpha(x)+o(\alpha(x)), \; x \to x_0
\end{align*}

то функция  $\alpha(x)$ называется главной частью функции $\beta(x)$ при $x \to x_0$.

Если задана функция  $\beta(x)$, то ее главная часть не определяется однозначно: любая функция $\alpha(x)$ эквивалентная $\beta(x)$ является ее главной частью.

Однако, если задаваться определенным видом главной части, то главная часть указанного вида может определяться однозначно.

\end{document}
