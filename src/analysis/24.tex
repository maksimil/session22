\documentclass{article}

\usepackage{defines}

\begin{document}

\tickettitle{24}{Правосторонний и левосторонний пределы функции в точке. Свойства пределов функций}

\define{правостороннего и левостороннего пределов функции в точке}

Пусть функция $f$ определена на $(a;x_0)$

$g\in\overline\R$ называется пределом  $f$ в точке $x_0$ слева, если
\begin{align*}
	\forall\{x_{n}\}:((\forall n\in\N)\;x_{n}\in(a;x_0)\land\slimty x_{n}=x_0)\;\slimty f(x_{n})=g
\end{align*}

Обозначение: $\slim{x\to x_0-0}f(x)=g$ или $f(x_0-0)=g$

Пусть функция $f$ определена на $(x_0;a)$

$g\in\overline\R$ называется пределом  $f$ в точке $x_0$ справа, если
\begin{align*}
	\forall\{x_{n}\}:((\forall n\in\N)\;x_{n}\in(x_0;a)\land\slimty x_{n}=x_0)\;\slimty f(x_{n})=g
\end{align*}

Обозначение: $\slim{x\to x_0+0}f(x)=g$ или $f(x_0+0)=g$

\theorem

$f$ определена на $X:=(a;x_0)\cup(x_0;b)$
\begin{align*}
	\slim{x\to x_0+0}f(x)=g\land\slim{x\to x_0-0}f(x)=g\Lrarr\slim{x\to x_0}f(x)=g
\end{align*}

\onlyif

Возьмём последовательность $\{x_{n}\}\subset X$, сходящуюся к $g$

$\{x_{a_{n}}\}$ --- подпоследовательность всех $x_{n}<x_0$

$\{x_{b_{n}}\}$ --- подпоследовательность всех $x_{n}>x_0$
\begin{align*}
	 & \slim{x\to x_0-0}f(x)=g\Rarr \slimty f(x_{a_{n}})=g &  & \slim{x\to x_0+0}f(x)=g\Rarr\slimty f(x_{b_{n}})=g
\end{align*}

Иными словами:
\begin{align*}
	\forall\eps>0\;\exists k:(\forall m>k)\;|f(x_{a_{m}})-g|<\eps\land|f(x_{b_{m}})-g|<\eps\Rarr (\forall n>\max\{a_{m},b_{m}\})\;|f(x_{n})-g|<\eps\qed
\end{align*}

\enough

Определение предела по Гейне распространяется в том числе и на последовательности из $(a;x_0)\subset X$ и $(x_0;a)\subset X\qed$

\sectitle{Свойства пределов функций}

Арифметические свойства, свойства, связанные с операциями сравнения и принцип сжатой переменной доказываются через поледовательности

Аналогично эти свойства доказываются и для односторонних пределов

\end{document}

