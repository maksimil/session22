\documentclass{article}

\usepackage{defines}

\begin{document}

\tickettitle{45}{Точки разрывов монотонных функции}

\theorem

$f$ определена и монотонна на $[a,b] \Rightarrow \forall x_0 \in [a,b)\;\exists \slim{x \to x_0 + 0} f(x) \land \forall x_0\in(a;b]\;\exists \slim{x \to x_0 - 0} f(x)$

\proof

Необходимо доказать:
\begin{enumerate}
	\item{}$\forall x_0 \in [a,b)\; \exists \slim{x \to x_0 + 0} f(x)$
				\item{}$\forall x_0 \in (a;b]\; \exists \slim{x \to x_0 - 0} f(x)$
\end{enumerate}

Без ограничения общности рассмотрим только первый случай при неубывающей $f$
\begin{align*}
	 & x_0<b \Rightarrow Z:=\setdef{f(x)}{x\in(x_0;b]} \ne \eset                                                 \\
	 & f\text{ --- неубывающая} \Rarr (\forall x\in (x_0;b])\;f(x_0)\leq f(x)\Rarr Z\text{ --- ограничено снизу}
\end{align*}

Значит $\exists\inf Z=:\gamma$

Докажем, что $\gamma = \slim{x \to c + 0} f(x)$

По свойству $\inf$
\begin{align*}
	\forall \eps>0\;\exists \delta>0:\gamma \leq f(x_0 + \delta) < \gamma + \varepsilon
\end{align*}

$f$ -- неубывающая, значит
\begin{align*}
	\forall x \in (x_0; x_0+\delta) : \gamma-\eps<\gamma\leq f(x)\leq f(x_0+\delta) < \gamma + \eps\Rarr |f(x)-\gamma|<\eps\Rarr\exists\slim{x\to x_0+0}f(x)=\gamma\qed
\end{align*}

Аналогично доказывается второй случай и случаи с невозрастающей $f$

\theorem

Если $f$ определена и монотонна на $[a,b]$, то $f$ может иметь на $[a;b]$ только разрывы 1-го рода

\proof

По предыдущей теореме $\forall x_0 \in [a,b)\;\exists \slim{x \to x_0 + 0} f(x) \land \forall x_0\in(a;b]\;\exists \slim{x \to x_0 - 0} f(x)$

То есть все односторонние пределы существуют, значит разрывы могут быть только первого рода$\qed$

\pagebreak

\result

В условиях предыдущей теоремы $f$ имеет не более чем счётное множество точек разрыва

\proof

Без ограничения общности пусть $f$ --- неубывающая

Достаточно доказать, что множество точек разрыва на $(a;b)$ счётно

Обозначим это множество буквой $W$
\begin{align*}
	(\forall x\in W)\;f(x-0)<f(x+0)\Rarr\exists r(x)\in\Q:f(x-0)<r(x)<f(x+0)
\end{align*}

$f$ --- неубывающая
\begin{align*}
	 & (\forall x_1,x_2\in W:x_1<x_2)\;f(x_1-0)<r(x_1)<f(x_1+0)\leq f(x_2-0)<r(x_2)<f(x_2+0) \\
	 & \Rarr r(x_1)<r(x_2)\Rarr r\text{ --- инъекция }
\end{align*}

Таким образом $r:W\to r(W)$ --- биекция, то есть $W\sim$ подмножеству счётного множества $\Q$

Тогда $W$ --- не более чем счётное множество$\qed$

\end{document}
