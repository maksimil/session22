\documentclass{article}

\usepackage{defines}

\usepackage{mathtools}

\begin{document}

\tickettitle{5}{Типы числовых множеств. Верхняя и нижняя грани множества.}

\define{типов числовых множеств}
\begin{enumerate}
	\item Отрезок (сегмент): $[a;b] := \setdef{x\in\R}{a\leq x\leq b}$
	\item Интервал (открытый промежуток): $(a;b) := \setdef{x\in\R}{a<x<b}$
	\item Окрестность ($\eps$-окрестность) $a\in\R$:
	      $(a-\eps,a+\eps)$
	\item Числовая прямая: $(-\infty;+\infty)$
	\item Полупрямая (луч): $[a;+\infty)$; $(-\infty;a]$
	\item Полуотрезок: $[a;b)$; $(a;b]$
	\item Открытая полупрямая (луч): $(a;+\infty)$; $(-\infty;a)$
	\item Расширенная числовая прямая: $\overline{\R}:=\R\cup\{-\infty,+\infty\}$
\end{enumerate}

\define{ограниченного множества}

$A\subset\R$ ограничено сверху, если $\exists M:\forall a\in A\;a<M$

$A\subset\R$ ограничено снизу, если $\exists M:\forall a\in A\;a>M$

$A\subset\R$ ограничено, если оно ограничено и сверху, и снизу

\define{точной верхней и нижней граней множества}

$M$ --- верхняя грань $A\subset\R$, если $\forall a\in A\;a\leq M$

$M$ --- нижняя грань $A\subset\R$, если $\forall a\in A\;a\geq M$

Наименьшая (наибольшая) из всех верхних (нижних) граней множества $A\subset\R$,
называется точной верхней (нижней) гранью.

Точная верхняя грань --- $\sup A$ (супремум)

Точная нижняя грань --- $\inf A$ (инфенум)

\pagebreak

\theorem

$A \neq\eset$ $\land$ $A$ --- ограничено сверху $\Rarr \exists \sup A$

$A \neq\eset$ $\land$ $A$ --- ограничено снизу $\Rarr \exists \inf A$

\proof
\begin{enumerate}
	\item$A$ --- ограничено сверху

	Рассмотрим $U:=\setdef{u\in\R}{a\leq u\;\forall a\in A}$ --- множество
	всех верхних граней $A$.
	\begin{align*}
		 & \exists M\in\R: a<M\;\forall a\in A	\Rarr M\in U\Rarr U\neq\eset
	\end{align*}
	Если $\exists\min U$, то $\exists\sup A$ (по определению $\sup A$)

	По аксиоме полноты:
	\begin{align*}
		 & a\leq u\;\forall a\in A,u\in U\Rarr\exists m\in\R:
		a\leq m\leq u\;\forall a\in A,u\in U                    \\
		 & a\leq m\;\forall a\in A\Rarr m\in U                  \\
		 & (m\leq u\;\forall u\in U)\land(m\in U)\Rarr m=\min U
		\Rarr m=\sup A\qed
	\end{align*}

	\item$A$ --- ограничено снизу

	Рассмотрим $V:=\setdef{v\in\R}{v\leq a\;\forall a\in A}$ --- множество
	всех нижних граней $A$.
	\begin{align*}
		 & \exists M\in\R: M<a\;\forall a\in A\Rarr M\in V\Rarr V\neq\eset
	\end{align*}
	Если $\exists\max V$, то $\exists\inf A$ (по определению $\inf A$)

	По аксиоме полноты:
	\begin{align*}
		 & v\leq a\;\forall a\in A,v\in V\Rarr\exists m\in\R:
		v\leq m\leq a\;\forall a\in A,v\in V                    \\
		 & m\leq a\;\forall a\in A\Rarr m\in V                  \\
		 & (v\leq m\;\forall v\in V)\land(m\in V)\Rarr m=\max V
		\Rarr m=\inf A\qed
	\end{align*}
\end{enumerate}

\theorem
\begin{align*}
	u=\sup A\Lrarr
	 & \begin{cases}
		   x\leq u\;\forall x\in A                & (i)  \\
		   \forall\eps>0\;\exists x\in A:x>u-\eps & (ii)
	   \end{cases}  \\
	v=\inf A\Lrarr
	 & \begin{cases}
		   x\geq v\;\forall x\in A                & (iii) \\
		   \forall\eps>0\;\exists x\in A:x<v+\eps & (iv)
	   \end{cases}
\end{align*}

\proof
\begin{align*}
	(i)   & \Rarr u \text{ --- верхняя грань }                \\
	(ii)  & \Rarr u \text{ --- наименьшая верхняя грань }\qed \\
	(iii) & \Rarr v \text{ --- нижняя грань }                 \\
	(iv)  & \Rarr v \text{ --- наибольшая нижняя грань }\qed
\end{align*}

\end{document}
