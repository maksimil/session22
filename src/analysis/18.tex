\documentclass{article}

\usepackage{defines}

\begin{document}

\tickettitle{18}{Частичные пределы последовательности {\it Билет не проверен}}

\define{Частичный предел последовательности}

Предел подпоследовательности называется
частичным пределом последовательности.


Частичный предел последовательности может не быть её пределом. Например, частичными пределами последовательности
$\{(-1)^n\}$, являются числа +1 и -1, а предела у этой последовательности нет.

\theorem{}

Из каждой последовательности можно выделить подпоследовательность, имеющую предел, конечный или
бесконечный.

\proof

В самом деле, если последовательность $\{x_n\}$ не ограничена
сверху, то из неё можно выделить подпоследовательность, сходящуюся к $+\infty$. Сначала выбираем число $x_{n_1}$
такое, что $x_{n_1}>1$.
Затем, пользуясь неограниченностью сверху последовательности,
находим такой номер $n_2>n_1$, что для $x_{n_2}$
выполняется неравенство $x_{n_1}>2$ В результате получим $\slim{k \to \infty} x_{n_k}=+\infty$.

\end{document}
