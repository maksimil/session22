\documentclass{article}

\usepackage{defines}

\begin{document}

\tickettitle{67}{Приближенное вычисление функции с помощью формулы Тейлора, оценка погрешности.}

\theorem

Пусть $f(x)$ $n$ раз дифференцируема на $X$ и $x_0\in X$

$\varphi(x)$ --- приближённое значение $f(x)$:
\begin{align*}
	\varphi(x):=f(x_0)+\sum_{k=1}^{n-1}\frac{(x-x_0)^{k}}{k!}f^{(k)}(x_0)
\end{align*}

Пусть $f^{(n)}(x)$ ограничена на $X$:
\begin{align*}
	\exists M:(\forall x\in X \land k = \overline{1,n} )\;|f^{(k)}(x)|\leq M
\end{align*}

Тогда:
\begin{align*}
	|f(x)-\varphi(x)|\leq M\frac{|x-x_0|^{n}}{n!}\text{ --- универсальная оценка погрешности}
\end{align*}

\proof

По теореме Тейлора c остаточным членом в форме Лагранжа:
\begin{align*}
	 & (\forall x\in X)\;f(x)=f(x_0)+\sum_{k=1}^{n-1}\frac{(x-x_0)^{k}}{k!}f^{(k)}(x_0)+\frac{(x-x_0)^{n}}{n!}f^{(n)}(x_0+\theta(x-x_0)),\;\theta\in(0;1) \\
	 & |f(x)-\varphi(x)|=\frac{|x-x_0|^{n}}{n!}|f^{(n)}(x_0+\theta(x-x_0))|                                                                               \\
\end{align*}

Оценим $f^{(n)}(x_0+\theta(x-x_0))$:
\begin{align*}
	x_0+\theta(x-x_0)\in X \text{ ,т.к. } x, x_0 \in X\Rarr |f^{(n)}(x_0+\theta(x-x_0))|\leq M
\end{align*}

Тогда:
\begin{align*}
	 & |f(x)-\varphi(x)|=\frac{|x-x_0|^{n}}{n!}|f^{(n)}(x_0+\theta(x-x_0))|\leq M\frac{|x-x_0|^{n}}{n!}\qed
\end{align*}

{\it Замечание:}
\begin{align*}
	\slimty\frac{|x-x_0|^{n}}{n!}=0
\end{align*}

\result

Значение $f(x)$ можно заменять на $\varphi(x)$ с любой наперёд заданной точностью

{\it Замечание:}

Теорема говорит об универсальной оценке остаточного члена разложения Тейлора.($R_n = f(x)-\varphi(x)$)

Если численная оценка не важна, то используется форма Пеано и говорят об асимптотическом приближении функции.


\end{document}