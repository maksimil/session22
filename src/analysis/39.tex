\documentclass{article}

\usepackage[T2A, T1]{fontenc}
\usepackage[utf8]{inputenc}
\usepackage[russian]{babel}
\usepackage{enumitem}
\usepackage{defines}

\usepackage{fullpage}

\raggedright
\begin{document}

\tickettitle{39}{Условие непрерыности монотонной функции}

\theorem

$f$ --- монотонна (монотонно возрастающая) на промежутке $X$
\begin{align*}
	f(X)=Y\text{, где $Y$ --- промежуток}\Rarr f\text{ --- непрерывна на $X$ }
\end{align*}

Здесь под промежутком имеется в виду как открытый, полуоткрытый, так и закрытый промежуток

\proof

Без ограничения общности пусть $f$ --- монотонно возрастающая.

\begin{enumerate}
	\item{}Докажем для начала теорему для внутренних точек $X$

	Докажем, что $(\forall(a;b)\subset X, x_0\in(a;b))$ $f$ --- непрерывна слева в $x_0$:

	\begin{enumerate}
		\item{}Найдём $\slim{x\to x_0-0}f(x)$:
		\begin{align*}
			 & (\forall x\in(a;x_0))\;f(x)\leq f(x_0)\Rarr \exists\sup f((a;x_0))        \\
			 & g:=\sup f((a;x_0))
			\Rarr \forall\eps>0\;\exists\delta>0:f(x_0-\delta)>g-\eps\Rarr               \\
			 & \Rarr (\forall x\in(x_0-\delta;x_0))\;g-\eps<f(x_0-\delta)\leq f(x)\leq g
			\Rarr g-\eps<f(x)\leq g\Rarr \slim{x\to x_0-0}f(x)=g
		\end{align*}

		\item{}Докажем, что $f(x_0)=g$
		\begin{align}
			\llet f(x_0)<g & \Rarr \llet\eps=g-f(x_0)>0\;\exists x\in(a;x_0):g-\eps<f(x)\Rarr f(x_0)<f(x)\text{, но } x<x_0\label{flg} \\
			\llet f(x_0)>g & ,\; (f(x)\leq g\;\forall x\in(a;x_0))\land(f(x_0)\leq f(x)\;\forall x\in[x_0;b))\Rarr\notag               \\
			               & \Rarr (g;f(x_0))\not\subset Y\text{, но $Y$ --- промежуток}\label{fgg}
		\end{align}
		\begin{align*}
			\eqref{flg}\land\eqref{fgg}\Rarr f(x_0)=g
		\end{align*}
	\end{enumerate}

	Аналогично доказывается непрерывность справа.

	Таким образом:
	\begin{align}
		 & (\forall x_0\in(a;b))\;\slim{x\to x_0-0}f(x)=f(x_0)\label{leftcont}  \\
		 & (\forall x_0\in(a;b))\;\slim{x\to x_0+0}f(x)=f(x_0)\label{rightcont}
	\end{align}
	\begin{align*}
		\eqref{leftcont}\land\eqref{rightcont}\Rarr (\forall x_0\in(a;b))\;\slim{x\to x_0}f(x)=f(x_0)\qed
	\end{align*}

	\pagebreak
	\item{}Докажем теперь теорему для крайних точек $X$, если таковы есть

	Докажем, что $f$ --- непрерывна слева в $b:=\max X$

	\begin{enumerate}
		\item{}Найдём $\slim{x\to b-0}f(x)$:
		\begin{align*}
			 & (\forall a\in X, x\in(a;b))\;f(x)\leq f(b)\Rarr \exists\sup f((a;b)) \\
			 & g:=\sup f((a;b))
			\Rarr \forall\eps>0\;\exists\delta>0:f(b-\delta)>g-\eps\Rarr            \\
			 & \Rarr (\forall x\in(b-\delta;b))\;g-\eps<f(b-\delta)\leq f(x)\leq g
			\Rarr g-\eps<f(x)\leq g\Rarr \slim{x\to b-0}f(x)=g
		\end{align*}

		\item{}Докажем, что $f(b)=g$
		\begin{align}
			\llet f(b)<g & \Rarr \llet\eps=g-f(b)>0\;\exists x\in(a;b):g-\eps<f(x)\Rarr f(b)<f(x)\text{, но } x<b\label{flgb}      \\
			\llet f(b)>g & ,\;(f(x)\leq g\;\forall x\in(a;b))\Rarr (g;f(b))\not\subset Y\text{, но $Y$ --- промежуток}\label{fggb}
		\end{align}
		\begin{align*}
			\eqref{flgb}\land\eqref{fggb}\Rarr f(b)=g\Rarr \slim{x\to b-0}f(x)=f(b)\qed
		\end{align*}

	\end{enumerate}

	Аналогично доказывается непрерывность справа для $\min X$
\end{enumerate}

\end{document}
