\documentclass{article}

\usepackage{defines}

\begin{document}

\tickettitle{62}{Первый дифференциал функции в точке, его свойства. Инвариантность~формы~первого~дифференциала}

\define{первого дифференциала функции в точке}

Дифференциалом или первым дифференциалом $y=f(x)$ в точке $x$ по отношению к приращению $h$ называется $f'(x)h$ и обозначается как $dy$ или $df(x)$

\textbf{Геометрический смысл дифференциала}

Дифференциал в некоторой точке \textbf{---} это главная линейная часть  приращения функции

$d_x(f(x), h)$ \textbf{---} дифференциал относительно приращения $h$ и переменной $x$


Если $f(x)=x$, то $f'(x)=1$ $\Rarr$ $df(x)=dx=h=\Delta x$

$dy=f'(x)dx$ $\Rarr$ $f'(x)=\frac{dy}{dx}$


\sectitle{Свойства первого дифференциала}

\theorem

$y=f(x)$, $z=g(x)$
\begin{align*}
	d(y\pm z)=dy\pm dz
\end{align*}

\proof
\begin{align*}
	d(y\pm z)=(y\pm z)'dx =y'dx\pm z'dx=dy\pm dz\qed
\end{align*}

\theorem

$y=f(x)$, $z=g(x)$
\begin{align*}
	d\left(\frac{y}{z}\right)=\frac{z \cdot dy-y \cdot dz}{z^{2}}
\end{align*}

\proof
\begin{align*}
	d\left(\frac{y}{z}\right)=\left(\frac{y}{z}\right)'dx =\frac{y'z-yz'}{z^{2}}dx=\frac{z \cdot dy-y \cdot dz}{z^{2}}\qed
\end{align*}

\theorem

$y=f(x)$, $z=g(x)$
\begin{align*}
	d(yz)=z \cdot dy+y \cdot dz
\end{align*}

\proof
\begin{align*}
	d(yz)=(yz)'dx=(y'z+yz')dx=z \cdot dy+y \cdot dz\qed
\end{align*}

\pagebreak

\theorem

$y=f(x)$ дифференцируема в $x_0$

$dy$ --- главная часть $\Delta y$

\proof

Рассмотрим $\Delta y-dy$:
\begin{align*}
	 & \frac{\Delta y-dy}{dx}=\frac{f(x+\Delta x)-f(x)-f'(x)\Delta x}{\Delta x}=\frac{f(x+\Delta x)-f(x)}{\Delta x}-f'(x)\xrigh)tarrow{\Delta x\to 0}0 \text{ (по условию теоремы)}\Rarr \\
	 & \Rarr\Delta y-dy=o(\Delta x)\Rarr \Delta y=dy+o(\Delta x)\qed
\end{align*}

\theorem[об инвариантности формы первого дифференциала]

$z=g(y)$, $y=f(x)$
\begin{align}
	 & dz=g_{x}'dx\label{62:form_1}                \\
	 & dz=g_{y}'dy=g_{y}'f_{x}'dx\label{62:form_2}
\end{align}

$\eqref{62:form_1}$ и $\eqref{62:form_2}$ равны, те $g'_{x}dx=g'_{y}dy$ 

(Иными словами вид дифференциала всегда один и тот же: $dy=f'(x)dx$)

\proof

При условии $f_{x}'dx = dy$ рассмотрим $\eqref{62:form_1}$ :
\begin{align*}
	dz=g_{x}'dx=g(f(x))'dx=g_{y}'f_{x}'dx=g_{y}'dy=\eqref{62:form_2}\qed
\end{align*}

\end{document}