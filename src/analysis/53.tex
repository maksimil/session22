\documentclass{article}

\usepackage{defines}

\begin{document}

\tickettitle{53}{Теорема о производной константы. Теорема о производной монотонной и строго~монотонной~функции. Достаточный признак монотонности функции.}

\theorem

$f$ --- дифференцируема на промежутке $X$
\begin{align*}
	(\forall x\in X)\;f(x) = const \Leftrightarrow (\forall x\in X)\;f'(x) = 0
\end{align*}

\onlyif
\begin{align*}
	(\forall x\in X)\;f(x)=c\Rarr\slim{h\to 0}\frac{f(x+h)-f(x)}{h}=\slim{h\to 0}\frac{c-c}{h}=\slim{h\to 0}\frac{0}{h}=0\Rarr f'(x)=0\qed
\end{align*}

\enough

$f$ непрерывна и дифференцируема на промежутке $X$, применим теорему Лагранжа:
\begin{align*}
	(\forall a,b\in\R)\;\exists\gamma\in(a;b):f(b)-f(a)=f'(\gamma)(b-a)=0\Rarr f(b)=f(a)\Rarr f(x)=\const\qed
\end{align*}

\theorem

$f$ --- дифференцируема на промежутке $X$

$(\forall x\in X\;f'(x)\geq 0$ $\Lrarr$ $f$ --- монотонно возрастающая на $X$

$(\forall x\in X)\;f'(x)\leq 0$ $\Lrarr$ $f$ --- монотонно убывающая на $X$


\onlyif

Без ограничения общности пусть $f'(x)\geq 0$

Возьмём $x_1,x_2\in X$:
\begin{align*}
	\llet x_1<x_2\Rarr \exists\gamma\in(x_1,x_2):f(x_2)-f(x_1)=f'(\gamma)(x_2-x_1)\geq 0\Rarr f(x_2)\geq f(x_1)\qed
\end{align*}

\enough

Без ограничения общности пусть $f$ --- монотонно возрастающая на $X$
\begin{align*}
	 & \begin{cases}
		   f(x+h)\geq f(x) & \text{ при $h>0$} \\
		   f(x+h)\leq f(x) & \text{ при $h<0$}
	   \end{cases}
	\Rarr \frac{f(x+h)-f(x)}{h}\geq 0\Rarr                      \\
	 & \Rarr f'(x)=\slim{h\to 0}\frac{f(x+h)-f(x)}{h}\geq 0\qed
\end{align*}

\pagebreak

\theorem

$f$ --- дифференцируема на промежутке $X$

$(\forall x\in X)\;f'(x)>0$ $\Rarr$ $f$ --- строго монотонно возрастающая на $X$

$(\forall x\in X)\;f'(x)<0$ $\Rarr$ $f$ --- строго монотонно убывающая на $X$

\proof

Без ограничения общности пусть $f'(x)> 0$

Возьмём $x_1,x_2\in X$:
\begin{align*}
	\llet x_1<x_2\Rarr \exists\gamma\in(x_1,x_2):f(x_2)-f(x_1)=f'(\gamma)(x_2-x_1)> 0\Rarr f(x_2)> f(x_1)\qed
\end{align*}

\theorem

$f$ --- непрерывна на промежутке $X$

$f$ --- дифференцируема на $X\setminus A$, где $A\subset X$ --- конечное и крайние точки $X\notin A$

$(\forall x\in X\setminus A)\;f'(x)\geq 0$ $\Lrarr$ $f$ --- монотонно возрастающая на $X$

$(\forall x\in X\setminus A)\;f'(x)\leq 0$ $\Lrarr$ $f$ --- монотонно убывающая на $X$


\proof

$A$ --- конечно и $n:=|A|$, пронумеруем его элементы по возрастанию:
\begin{align*}
	 & a_{i}\in A,\;i=\overline{1,n} &  & a_{i}<a_{i+1},\;i=\overline{1,n-1}
\end{align*}

{\it Индукция:} $P(n)$ --- верность теоремы для $|A|=n$
\begin{enumerate}
	\item{}$P(0)$ --- предыдущие теоремы
	\item{}$P(n)\Rarr P(n+1)$

	Разделим $X$ элементом $a_{n+1}$:
	\begin{align*}
		 & X_{L}:=\setdef{x\in X}{x<a_{n+1}} &  & X_{R}:=\setdef{x\in X}{x> a_{n+1}} &  & X_{L}\cup\{a_{n+1}\}\cup X_{R}=X
	\end{align*}

	\onlyif

	Без ограничения общности пусть $f'(x)\geq 0$

	$f$ --- непрерывна на $X_{L}$ и дифференцируема на $X_{L}\setminus (A\setminus \{a_{n+1}\})$, тогда\\
	по $P(n)$ $f$ --- монотонно возрастающая на $X_{L}$

	$f$ --- дифференцируема на $X_{R}$ $\Rarr$ $f$ --- монотонно возрастающая на $X_{R}$

	Возьмём $x_1\in X_{L}$, $x_2\in (x_1;a_{n+1})$:
	\begin{align*}
		 & x_1<x_2\land x_1,x_2\in X_{L}\Rarr f(x_1)\leq f(x_2)
	\end{align*}

	Возьмём $\slim{x_2\to a_{n+1}-0}$ по обоим частям неравенства $f(x_1)\leq f(x_2)$, тогда по непрерывности:
	\begin{align*}
		 & \slim{x_2\to a_{n+1}-0}f(x_1)\leq \slim{x_2\to a_{n+1}-0}f(x_2)\Rarr (\forall x_1\in X_{L})\;f(x_1)\leq f(a_{n+1})
	\end{align*}

	Аналогично: $(\forall x_1\in X_{R})\;f(a_{n+1})\leq f(x_1)$

	\pagebreak

	Возьмём $x_1,x_2\in X:x_1<x_2$ и рассмотрим, где они могут лежать
	\begin{enumerate}
		\item{}$x_1,x_2\in X_{L}\Rarr f(x_1)\leq f(x_2)$
		\item{}$x_1\in X_{L},x_2=a_{n+1}\Rarr f(x_1)\leq f(x_2)$
		\item{}$x_1\in X_{L},x_2\in X_{R}\Rarr f(x_1)\leq f(a_{n+1})\leq f(x_2)\Rarr f(x_1)\leq f(x_2)$
		\item{}$x_1=a_{n+1},x_2\in X_{R}\Rarr f(x_1)\leq f(x_2)$
		\item{}$x_1,x_2\in X_{R}\Rarr f(x_1)\leq f(x_2)$
	\end{enumerate}

	Можно прийти к следущему выводу:
	\begin{align*}
		(\forall x_1,x_2\in X)\;x_1<x_2\Rarr f(x_1)\leq f(x_2)\qed
	\end{align*}

	\enough

	Без ограничения общности пусть $f$ --- монотонно возрастающая

	$f$ --- монотонно возрастающая на $X$ $\Rarr$ $f$ ---  монотонно возрастающая на $X_{L}\subset X$ и $X_{R}\subset X$

	$f$ --- непрерывна на $X_{L}$ и дифференцируема на $X_{L}\setminus (A\setminus \{a_{n+1}\})$, тогда\\
	по $P(n)$ $(\forall x\in X_{L}\setminus (A\setminus\{a_{n+1}\}))\;f'(x)\geq 0$

	$f$ --- дифференцируема на $X_{R}$ $\Rarr$ $(\forall x\in X_{R})\;f'(x)\geq 0$
	\begin{align*}
		 & X_{L}\setminus(A\setminus\{a_{n+1}\})\cup X_{R}=X_{L}\setminus A\cup X_{R}\setminus A=(X_{L}\cup X_{R})\setminus A=X\setminus a_{n+1}\setminus A=X\setminus A\Rarr \\
		 & \Rarr (\forall x\in X\setminus A)\;f'(x)\geq 0\qed
	\end{align*}

\end{enumerate}


\theorem

$f$ --- непрерывна на промежутке $X$

$f$ --- дифференцируема на $X\setminus A$, где $A\subset X$ --- конечное и крайние точки $X\notin A$

$(\forall x\in X\setminus A)\;f'(x)>0$ $\Rarr$ $f$ --- строго монотонно возрастающая на $X$

$(\forall x\in X\setminus A)\;f'(x)<0$ $\Rarr$ $f$ --- строго монотонно убывающая на $X$

\proof

Без ограничения общности пусть $f'(x)>0$

$A$ --- конечно и $n:=|A|$, пронумеруем его элементы по возрастанию:
\begin{align*}
	 & a_{i}\in A,\;i=\overline{1,n} &  & a_{i}<a_{i+1},\;i=\overline{1,n-1}
\end{align*}

{\it Индукция:} $P(n)$ --- верность теоремы для $|A|=n$
\begin{enumerate}
	\item{}$P(0)$ --- предыдущие теоремы
	\item{}$P(n)\Rarr P(n+1)$

	Разделим $X$ элементом $a_{n+1}$:
	\begin{align*}
		 & X_{L}:=\setdef{x\in X}{x<a_{n+1}} &  & X_{R}:=\setdef{x\in X}{x> a_{n+1}} &  & X_{L}\cup\{a_{n+1}\}\cup X_{R}=X
	\end{align*}

	\pagebreak

	$f$ --- непрерывна на $X_{L}$ и дифференцируема на $X_{L}\setminus (A\setminus \{a_{n+1}\})$, тогда\\
	по $P(n)$ $f$ --- строго монотонно возрастающая на $X_{L}$

	$f$ --- дифференцируема на $X_{R}$ $\Rarr$ $f$ --- строго монотонно возрастающая на $X_{R}$

	Возьмём $x_1\in X_{L}$, $x_2\in (x_1;a)$ и $x_3\in(x_2;a)$:
	\begin{align*}
		 & x_1<x_2<x_3\land x_1,x_2,x_3\in X_{L}\Rarr f(x_1)<f(x_2)<f(x_3)
	\end{align*}

	Возьмём $\slim{x_3\to a_{n+1}-0}$ по обоим частям неравенства $f(x_2)<f(x_3)$, тогда по непрерывности:
	\begin{align*}
		 & \slim{x_3\to a_{n+1}-0}f(x_2)\leq \slim{x_3\to a_{n+1}-0}f(x_{3})\Rarr f(x_2)\leq f(a_{n+1})\Rarr \\
		 & \Rarr f(x_1)<f(x_2)\leq f(a_{n+1})\Rarr (\forall x_1\in X_{L})\;f(x_1)<f(a_{n+1})
	\end{align*}

	Аналогично: $(\forall x_1\in X_{R})\;f(a_{n+1})<f(x_1)$

	Возьмём $x_1,x_2\in X:x_1<x_2$ и рассмотрим, где они могут лежать
	\begin{enumerate}
		\item{}$x_1,x_2\in X_{L}\Rarr f(x_1)<f(x_2)$
		\item{}$x_1\in X_{L},x_2=a_{n+1}\Rarr f(x_1)<f(x_2)$
		\item{}$x_1\in X_{L},x_2\in X_{R}\Rarr f(x_1)<f(a_{n+1})<f(x_2)\Rarr f(x_1)<f(x_2)$
		\item{}$x_1=a_{n+1},x_2\in X_{R}\Rarr f(x_1)<f(x_2)$
		\item{}$x_1,x_2\in X_{R}\Rarr f(x_1)<f(x_2)$
	\end{enumerate}

	Можно прийти к следущему выводу:
	\begin{align*}
		(\forall x_1,x_2\in X)\;x_1<x_2\Rarr f(x_1)<f(x_2)\qed
	\end{align*}

\end{enumerate}

\end{document}
