\documentclass{article}

\usepackage{defines}

\begin{document}

\tickettitle{53}{Теорема о производной константы. Теорема о производной монотонной и строго~монотонной~функции. Достаточный признак монотонности функции.}

\theorem
\begin{align*}
	f(x) = const \Leftrightarrow f'(x) = 0
\end{align*}

\onlyif
\begin{align*}
	f(x)=c\Rarr\slim{h\to 0}\frac{f(x+h)-f(x)}{h}=\slim{h\to 0}\frac{c-c}{h}=\slim{h\to 0}\frac{0}{h}=0\Rarr f'(x)=0\qed
\end{align*}

\enough
1) Следствие 1 из теореме Лагранжа(буквально формулировка)


\theorem

$f$ --- дифференцируема на $(a;b)$

$(\forall x\in(a;b))\;f'(x)\geq 0$ $\Lrarr$ $f$ --- монотонно возрастающая на $(a;b)$

$(\forall x\in(a;b))\;f'(x)\leq 0$ $\Lrarr$ $f$ --- монотонно убывающая на $(a;b)$


\onlyif

Без ограничения общности пусть $f'(x)\geq 0$

Возьмём $x_1,x_2\in(a;b)$:
\begin{align*}
	\llet a<x_1<x_2<b\Rarr \exists\gamma\in(x_1,x_2):f(x_2)-f(x_1)=f'(\gamma)(x_2-x_1)\geq 0\Rarr f(x_2)\geq f(x_1)\qed
\end{align*}

\enough

Без ограничения общности пусть $f$ --- монотонно возрастающая на $(a;b)$
\begin{align*}
	 & (f(x+h)\geq f(x)\text{ при $h>0$})\land (f(x+h)\leq f(x)\text{ при $h<0$})\Rarr \frac{f(x+h)-f(x)}{h}\geq 0\Rarr \\
	 & \Rarr f'(x)=\slim{h\to 0}\frac{f(x+h)-f(x)}{h}\geq 0\qed
\end{align*}

\theorem

$f$ --- дифференцируема на $(a;b)$

$(\forall x\in(a;b))\;f'(x)>0$ $\Rarr$ $f$ --- строго монотонно возрастающая на $(a;b)$

$(\forall x\in(a;b))\;f'(x)<0$ $\Rarr$ $f$ --- строго монотонно убывающая на $(a;b)$

\onlyif

Без ограничения общности пусть $f'(x)> 0$

Возьмём $x_1,x_2\in(a;b)$:
\begin{align*}
	\llet a<x_1<x_2<b\Rarr \exists\gamma\in(x_1,x_2):f(x_2)-f(x_1)=f'(\gamma)(x_2-x_1)> 0\Rarr f(x_2)> f(x_1)\qed
\end{align*}

\enough

Без ограничения общности пусть $f$ --- строго монотонно возрастающая на $(a;b)$
\begin{align*}
	 & (f(x+h) > f(x)\text{ при $h>0$})\land (f(x+h) < f(x)\text{ при $h<0$})\Rarr \frac{f(x+h)-f(x)}{h} > 0\Rarr \\
	 & \Rarr f'(x)=\slim{h\to 0}\frac{f(x+h)-f(x)}{h} > 0\qed
\end{align*}

\theorem 

$\llet$ f'(x) > 0 (< 0) везде на промежутке $X$, за исключением разве лишь конечного числа точек, тогда f(x) - строго монотонная на $X$

\proof

Доказательство является прямым следствием предшествующих теорем $\qed$
\end{document}