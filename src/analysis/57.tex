\documentclass{article}

\usepackage{defines}

\begin{document}

\tickettitle{57}{Теорема Лопиталя, применение теоремы Лопиталя для раскрытия неопределенности.}

\theorem[Лопиталя 1]

$f$, $g$ непрерывны и дифференцируемы $n$ раз в окрестности $a$

\begin{align*}
	\begin{cases}
		f^{(m)}(a)=0,m=\overline{0,n-1} &                  \\
		g^{(m)}(a)=0,m=\overline{0,n-1} & g^{(n)}(a)\neq 0
	\end{cases}\Rarr \slim{x\to a}\frac{f(x)}{g(x)}=\frac{f^{(n)}(a)}{g^{(n)}(a)}
\end{align*}

\proof
\begin{align*}
	 & f(x) = f(a) + \sum_{k=1}^{n}\frac{(x-a)^{k}}{k!}f^{(k)}(a)+o((x-a)^{n})=\frac{(x-a)^{n}}{n!}f^{(n)}(a)+o((x-a)^{n})\Rarr                     \\
	 & \Rarr\slim{x\to a}\frac{f(x)}{(x-a)^{n}}=\frac{f^{(n)}(a)}{n!}\text{, аналогично: }\slim{x\to a}\frac{g(x)}{(x-a)^{n}}=\frac{g^{(n)}(a)}{n!} \\
	 & \slim{x\to a}\frac{f(x)}{g(x)}=\slim{x\to a}\frac{f(x)}{(x-a)^{n}}\slim{x\to a}\frac{(x-a)^{n}}{g(x)}=\frac{f^{(n)}(a)}{g^{(n)}(a)}\qed
\end{align*}

\theorem[Лопиталя 2.1]

{\it Замечание: т. Лопиталя 1 рассматривает случай, когда $f$ и $g$ определены в $a$}

$f$, $g$ определены и дифференцируемы на $(a;b)$
\begin{align*}
	\begin{cases}
		\slim{x\to a+0}f(x)=0 \\
		\slim{x\to a+0}g(x)=0
	\end{cases} &  & \lor &  & \slim{x\to a+0}|g(x)|=+\infty
\end{align*}

$(\forall x\in(a;b))\;g'(x)\neq 0$
\begin{align*}
	\exists \slim{x\to a+0}\frac{f'(x)}{g'(x)}=L\in\overline{\R}\Rarr\exists\slim{x\to a+0}\frac{f(x)}{g(x)}=L
\end{align*}

\proof

Возьмём $x_2\in(a;b),\;x_1\in(a;x_2)$.

$f$ и $g$ определены на $[x_1;x_2]$ и дифференцируемы на $(x_1;x_2)$, $g'(x)\neq 0\;\forall x\in (x_1;x_2)$

Тогда по т. Коши:
\begin{align*}
	 & \frac{f'(\gamma)}{g'(\gamma)}=\frac{f(x_2)-f(x_1)}{g(x_2)-g(x_1)},\;\gamma\in(x_1;x_2)
\end{align*}

\pagebreak

Ограничим полученную дробь:
\begin{align*}
	 & m(x_2):=\inf\left\{\frac{f'(z)}{g'(z)}\;|\;z\in(a;x_2)\right\} &  & M(x_2):=\sup\left\{\frac{f'(z)}{g'(z)}\;|\;z\in(a;x_2)\right\}
\end{align*}
\begin{align}
	 & m(x_2)\leq\frac{f(x_2)-f(x_1)}{g(x_2)-g(x_1)}\leq M(x_2)\label{57:limit}
\end{align}

\begin{enumerate}
	\item{}$\slim{x\to a+0}f(x)=\slim{x\to a+0}g(x)=0$

	Возьмём $\slim{x_1\to a+0}$ по всем частям $\eqref{57:limit}$:
	\begin{align*}
		 & \slim{x_1\to a+0}m(x_2)\leq \slim{x_1\to a+0}\frac{f(x_2)-f(x_1)}{g(x_2)-g(x_1)}
		=\slim{x_1\to a+0}\frac{\frac{f(x_2)}{g(x_2)}-\frac{f(x_1)}{g(x_2)}}{1-\frac{g(x_1)}{g(x_2)}}=\frac{f(x_2)}{g(x_2)} \leq \slim{x_1\to a+0}M(x_2)\Rarr \\
		 & \Rarr m(x_2)\leq \frac{f(x_2)}{g(x_2)}\leq M(x_2)\text{ (тк $m(x_2)$ и $M(x_2)$ не зависят от $x_1$)}
	\end{align*}

	Тогда по принципу сжатой переменной:
	\begin{align*}
		 & \left(\slim{x_2\to a+0}m(x_2)=\slim{x_2\to a+0}\frac{f(x_2)}{g(x_2)}=L\right)\land\left(\slim{x_2\to a+0}M(x_2)=L\right)
		\Rarr\exists\slim{x_2\to a+0}\frac{f'(x_2)}{g'(x_2)}=L\qed
	\end{align*}

	\item{}$\slim{x\to a+0}|g(x)|=+\infty$

	Возьмём $\sliminf{x_1\to a+0}$ по всем частям $\eqref{57:limit}$:
	\begin{align*}
		 & \sliminf{x_1\to a+0}m(x_2)\leq \sliminf{x_1\to a+0}\frac{f(x_1)-f(x_2)}{g(x_1)-g(x_2)}
		 =\sliminf{x_1\to a+0}\frac{\frac{f(x_1)}{g(x_1)}-\frac{f(x_2)}{g(x_1)}}{1-\frac{g(x_2)}{g(x_1)}}=\sliminf{x_1\to a+0}\frac{f(x_1)}{g(x_1)} 
		 \leq \sliminf{x_1\to a+0}M(x_2)\Rarr \\
		 & \Rarr m(x_2)\leq \sliminf{x_1\to a+0}\frac{f(x_1)}{g(x_1)}\leq M(x_2)\text{ (тк $m(x_2)$ и $M(x_2)$ не зависят от $x_1$)}
	\end{align*} 

	Аналогично для $\slimsup{x_1\to a+0}$:
	\begin{align}
		m(x_2)\leq \sliminf{x_1\to a+0}\frac{f(x_1)}{g(x_1)}\leq\slimsup{x_1\to a+0}\frac{f(x_1)}{g(x_1)}\leq M(x_2)\label{57:inflimit}
	\end{align}

	Возьмём $\slim{x_2\to a+0}$ по всем частям $\eqref{57:inflimit}$:
	\begin{align*}
		&\slim{x_2\to a+0}m(x_2)\leq\slim{x_2\to a+0}\left(\sliminf{x_1\to a+0}\frac{f(x_1)}{g(x_1)}\right)
		\leq\slim{x_2\to a+0}\left(\slimsup{x_1\to a+0}\frac{f(x_1)}{g(x_1)}\right)\leq \slim{x_2\to a+0}M(x_2)\Rarr\\
		&\Rarr L\leq\sliminf{x_1\to a+0}\frac{f(x_1)}{g(x_1)}\leq\slimsup{x_1\to a+0}\frac{f(x_1)}{g(x_1)}\leq L\Rarr
		 \sliminf{x_1\to a+0}\frac{f(x_1)}{g(x_1)}=\slimsup{x_1\to a+0}\frac{f(x_1)}{g(x_1)}=L\Rarr
		\exists\slim{x_1\to a+0}\frac{f(x_1)}{g(x_1)}=L\qed
	\end{align*}
\end{enumerate}

\pagebreak

\theorem[Лопиталя 2.2]

$f$, $g$ определены и дифференцируемы на $(b;a)$
\begin{align*}
	\begin{cases}
		\slim{x\to a-0}f(x)=0 \\
		\slim{x\to a-0}g(x)=0
	\end{cases} &  & \lor &  & \slim{x\to a-0}|g(x)|=+\infty
\end{align*}

$(\forall x\in(b;a))\;g'(x)\neq 0$
\begin{align*}
	\exists \slim{x\to a-0}\frac{f'(x)}{g'(x)}\in\overline{\R}\Rarr\exists\slim{x\to a-0}\frac{f(x)}{g(x)}=\slim{x\to a-0}\frac{f'(x)}{g'(x)}
\end{align*}

\proof

\newcommand\ft{\widetilde{f}}
\newcommand\gt{\widetilde{g}}

Произведём замену, чтобы свести случай к т. Лопиталя 2.1:
\begin{align*}
	 & x=-t                     & \ft(t) & =f(t) & \ft'(t) & =-f'(-t) \\
	 & x\to a-0\Lrarr t\to -a+0 & \gt(t) & =g(t) & \gt'(t) & =-g'(-t)
\end{align*}

Перепишем теперь условие через $t$:
\begin{align*}
	 & \slim{x\to a-0}f(x)=0\Rarr\slim{t\to -a+0}f(t)=0\Rarr\slim{t\to -a+0}\ft(t)=0    \\
	 & \slim{x\to a-0}g(x)=0\Rarr\slim{t\to -a+0}g(t)=0\Rarr\slim{t\to -a+0}\gt(t)=0    \\
	 & (\forall x\in(b;a))\;g'(x)\neq 0\Rarr (\forall t\in(-a;-b))\;\gt'(t)=-g'(-t)\neq 0
\end{align*}

Тогда по т. Лопиталя 2.1:
\begin{align*}
	 & \exists\slim{x\to a-0}\frac{f'(x)}{g'(x)}=\slim{t\to -a+0}\frac{\ft'(t)}{\gt'(t)}\Rarr
	\exists\slim{t\to -a+0}\frac{\ft(t)}{\gt(t)}=\slim{t\to -a+0}\frac{\ft'(t)}{\gt'(t)}\Rarr                                                                            \\
	 & \Rarr\exists\slim{x\to a-0}\frac{f(x)}{g(x)}=\slim{t\to -a+0}\frac{\ft(t)}{\gt(t)}=\slim{t\to -a+0}\frac{\ft'(t)}{\gt'(t)}=\slim{x\to a-0}\frac{f'(x)}{g'(x)}\qed
\end{align*}

\theorem[Лопиталя 2.3]

$f$, $g$ определены и дифференцируемы на $(b;a)\cup(a;c)$
\begin{align*}
	\begin{cases}
		\slim{x\to a}f(x)=0 \\
		\slim{x\to a}g(x)=0
	\end{cases} &  & \lor &  & \slim{x\to a}|g(x)|=+\infty
\end{align*}

$(\forall x\in(b;a)\cup(a;c))\;g'(x)\neq 0$
\begin{align*}
	\exists \slim{x\to a}\frac{f'(x)}{g'(x)}\in\overline{\R}\Rarr\exists\slim{x\to a}\frac{f(x)}{g(x)}=\slim{x\to a}\frac{f'(x)}{g'(x)}
\end{align*}

\proof
\begin{align*}
	\exists\slim{x\to a}\frac{f'(x)}{g'(x)}\Rarr
	\begin{cases}
		\exists\slim{x\to a-0}\cfrac{f(x)}{g(x)}=\slim{x\to a-0}\cfrac{f'(x)}{g'(x)}=\slim{x\to a}\cfrac{f'(x)}{g'(x)} \\[1.5em]
		\exists\slim{x\to a+0}\cfrac{f(x)}{g(x)}=\slim{x\to a+0}\cfrac{f'(x)}{g'(x)}=\slim{x\to a}\cfrac{f'(x)}{g'(x)}
	\end{cases}\Rarr\exists\slim{x\to a}\frac{f(x)}{g(x)}=\slim{x\to a}\frac{f'(x)}{g'(x)}
\end{align*}

\pagebreak

\theorem[Лопиталя 3.1]

$f$, $g$ определены и дифференцируемы на $(a;+\infty)$, $a>0$
\begin{align*}
	\begin{cases}
		\slim{x\to +\infty}f(x)=0 \\
		\slim{x\to +\infty}g(x)=0
	\end{cases} &  & \lor &  & \slim{x\to +\infty}|g(x)|=+\infty
\end{align*}

$(\forall x\in(a;+\infty))\;g'(x)\neq 0$
\begin{align*}
	\exists \slim{x\to +\infty}\frac{f'(x)}{g'(x)}\in\overline{\R}\Rarr\exists\slim{x\to +\infty}\frac{f(x)}{g(x)}=\slim{x\to +\infty}\frac{f'(x)}{g'(x)}
\end{align*}

\proof

Произведём замену, чтобы свести случай к т. Лопиталя 2.1:
\begin{align*}
	 & x=\frac{1}{t}               & \ft(t) & =f\left(\frac{1}{t}\right) & \ft'(t) & =-\frac{1}{t^{2}}f'\left(\frac{1}{t}\right) \\
	 & x\to +\infty\Lrarr t\to 0+0 & \gt(t) & =g\left(\frac{1}{t}\right) & \gt'(t) & =-\frac{1}{t^{2}}g'\left(\frac{1}{t}\right)
\end{align*}

Перепишем теперь условие через $t$:
\begin{align*}
	 & \slim{x\to +\infty}f(x)=0\Rarr\slim{t\to 0+0}f\left(\frac{1}{t}\right)=0\Rarr\slim{t\to 0+0}\ft(t)=0 \\
	 & \slim{x\to +\infty}g(x)=0\Rarr\slim{t\to 0+0}g\left(\frac{1}{t}\right)=0\Rarr\slim{t\to 0+0}\gt(t)=0 \\
	 & (\forall x\in(a;+\infty))\;g'(x)\neq 0\Rarr
	\left(\forall t\in\left(0;\frac{1}{a}\right)\right)\;\gt'(t)=-\frac{1}{t^{2}}g'\left(\frac{1}{t}\right)\neq 0
\end{align*}

Тогда по т. Лопиталя 2.1:
\begin{align*}
	 & \exists\slim{x\to +\infty}\frac{f'(x)}{g'(x)}=\slim{t\to 0+0}\frac{\ft'(t)}{\gt'(t)}\Rarr
	\exists\slim{t\to 0+0}\frac{\ft(t)}{\gt(t)}=\slim{t\to 0+0}\frac{\ft'(t)}{\gt'(t)}\Rarr  \\
	 & \Rarr\exists\slim{x\to +\infty}\frac{f(x)}{g(x)}=\slim{t\to 0+0}\frac{\ft(t)}{\gt(t)}
	=\slim{t\to 0+0}\frac{\ft'(t)}{\gt'(t)}=\slim{x\to+\infty}\frac{f'(x)}{g'(x)}\qed
\end{align*}

\theorem[Лопиталя 3.2 и 3.3]

Формулируются и доказываются аналогично с т. Лопиталя 2.2 и 2.3
\end{document}
