\documentclass{article}

\usepackage{defines}

\begin{document}

\tickettitle{19}{Бесконечно малые величины, действия над ними, классификация бесконечно малых величин {\it Билет не проверен}}

\define

Последовательность $\{a_n\}$ называется бесконечно малой последовательностью, если $\slim{n \to \infty}a_n=0$

Отметим несколько свойств бесконечно малых последовательностей:
\begin{enumerate}
	\item{}Алгебраическая сумма конечного числа бесконечно малых последовательностей есть бесконечно малая последовательность.

	Пусть $\{a_n\}$ и $\{b_n\}$ бесконечно малые последовательности. Покажем, что и последовательности $\{a_n+b_n\}$ и $\{a_n-b_n\}$ являются также бесконечно малыми.
	Зададим  $\epsilon>0$, тогда существует такой номер $n_\epsilon$, что $|a_n|<\frac{\epsilon}2$  и $|b_n|<\frac{\epsilon}2$ для всех $n \geq n_\epsilon$. Поэтому для $n \geq n_\epsilon$  имеем
	\begin{align*}
		|\alpha_n \pm \beta_n |\leq |\alpha_n|+|\beta_n|<\frac{\epsilon}2+\frac{\epsilon}2=\epsilon
	\end{align*}
	что и означает, что $\slim{n \to \infty}(\alpha_n\pm \beta_n)=0$

	Соответствующее утверждение для любого
	конечного числа слагаемых следует из указанного по индукции.
	\item{} Произведение бесконечно малой последовательности на ограниченную последовательность является бесконечно малой последовательностью.

	Пусть ${a_n}$ бесконечно малая последовательность, a ${x_n}$ ограниченная
	последовательность, т. е. существует такое число $b > 0$, что $|x_n| \leq b$ для всех номеров $n = 1, 2,...$ . Зададим  $\epsilon > 0$; в силу определения бесконечно малой последовательности существует такой номер $n_\epsilon$, что $| a_n| <\frac{\epsilon}b$ для всех $n \geq n_\epsilon$.
	Поэтому для всех $n \geq n_\epsilon$ имеем
	\begin{align*}
		|a_n b_n |= |a_n|+|b_n|<\frac{\epsilon}bb=\epsilon
	\end{align*}
	что и означает, что последовательность ${a_nx_n}$ бесконечно малая.
\end{enumerate}

\result

Произведение конечного числа бесконечно малых последовательностей является бесконечно малой последовательностью.

\proof

Это сразу следует по индукции из свойства 2, если заметить, что бесконечно малая
последовательность, как и всякая последовательность, имеющая предел, ограничена.

Классификация бесконечно малых величин:

Пусть $\exists \{x_n\}$ и $\{y_n\}$ - две бесконечно малые
\begin{enumerate}
	\item{}Если $\nexists \slim{n \to \infty}\frac{x_n}{y_n}$, то последовательности $\{x_n\}$ и $\{y_n\}$ несравнимы.
	\item{}Если $\slim{n \to \infty}\frac{x_n}{y_n}=p\neq 0$, то последовательности $\{x_n\}$ и $\{y_n\}$ одного порядка малости $x_n=O(y_n);y_n=O(x_n)$
	\item{}Если $ \slim{n \to \infty}\frac{x_n}{y_n}=1$, то последовательности $\{x_n\}$ и $\{y_n\}$ эквивалентны.
	\item{}Если $ \slim{n \to \infty}\frac{x_n}{y_n}=0$, то $x_n$ величина большего порядка малости, чем $y_n$.
	\item{}Если $ \slim{n \to \infty}\frac{x_n}{y_n}=\infty$, то $y_n$ величина большего порядка малочти, чем $x_n$.

\end{enumerate}

\end{document}
