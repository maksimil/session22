\documentclass{article}

\usepackage{defines}

\begin{document}

\tickettitle{25}{Предел суперпозиции функции}

\define{суперпозиции}

Пусть $f:X\to Y$ и $g:Y\to Z$

$h:X\to Z$, $h(x)=g(f(x))$ --- суперпозиция (сложная функция) функций $g$ и $f$

Обозначение: $h=g \circ f$

\theorem

$\exists\slim{x\to x_0}g(x)=y_0\land\exists\slim{y\to y_0}f(y)=L$

$y_0$ --- точка сгущения области определения $f$

$g(x)\neq y_0$ в $\dot{S}$ --- проколотой окрестности $x_0$

Тогда $\exists\slim{x\to x_0}f(g(x))=g$

\proof

Возьмём произвольную $\{x_{n}\}\subset\dot{S}$, сходящуюся к $x_0$, по определению предела
\begin{align*}
	 & \slimty g(x_{n})=y_0 &  & (\forall n\in\N)\;x_{n}\in\dot{S}\Rarr g(x_{n})\neq y_0
\end{align*}

Тогда последовательность $\{g(x_{n})\}$ тоже соответсвует определению предела
\begin{align*}
	 & \slimty g(x_{n})=y_0\land(\forall n\in\N)\;g(x_{n})\neq y_0\Rarr \slimty f(g(x_{n}))=L\Rarr                               \\
	 & \Rarr\forall\{x_{n}\}((\forall n\in\N)\;x_{n}\in\dot{S}\land x_{n}\to x_0)f(g(x_{n}))=L\Rarr \slim{x\to x_0}f(g(x))=L\qed
\end{align*}

\end{document}
