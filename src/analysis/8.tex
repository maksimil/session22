\documentclass{article}

\usepackage{defines}

\begin{document}

\tickettitle{8}{Предел числовой последовательности, его единственность. Ограниченность числовой последовательности.}

\define{предела числовой послевательности}

Число $g \in\R$ называется пределом последовательности $\{a_n\}$, если
\begin{align*}
	\forall\eps>0\;\exists k\in\N:(\forall n>k)\;|a_{n}-g|<\eps
\end{align*}

Обозначение: $g=\lim\limits_{n\to\infty}{a_n}$ или $a_n \rightarrow {g}$ при $n \rightarrow {\infty}$

{\it Альтернативные определения:}

\begin{enumerate}
	\item{}$g=\lim\limits_{n\to\infty}{a_n}$, если $\forall\eps>0$ неравенство $|a_n-g|<\eps$ выполняется при достаточно больших $n$.
	\item{}$g=\slimty a_{n}$, если $\forall\eps>0\;\exists k\in\N:(\forall n>k)\;a_{n}\in(g-\eps,g+\eps)$
\end{enumerate}

\define{сходящейся и расходящейся числовой послевательности}

Если последовательность имеет предел, то она сходится.

Если последовательность не сходится, то она расходится.

\theorem

Если последовательность имеет предел, то он единственный.

\proof

Возьмём сходящуюся последовательность $\{a_{n}\}$.

Пусть $g=\slimty a_{n}\land g'=\slimty a_{n}\land g\neq g'$.

Возьмём $\eps=\frac{|g-g'|}{2}>0$ и воспользуемся определением предела:
\begin{align*}
	 & \llet \eps=\frac{|g-g'|}{2}>0\;\exists k_1\in\N:(\forall n>k_1)\;|a_{n}-g|<\eps  \\
	 & \llet \eps=\frac{|g-g'|}{2}>0\;\exists k_2\in\N:(\forall n>k_2)\;|a_{n}-g'|<\eps
\end{align*}

Рассмотрим $\{a_{n}\}$ при $n>k:=\max\{k_1,k_2\}$:
\begin{align*}
	 & |a_{n}-g|<\eps\land|a_{n}-g'|<\eps\Rarr|a_{n}-g|+|a_{n}-g'|<2\eps \\
	 & 2\eps=|g-g'|=|(a_{n}-g')-(a_{n}-g)|\leq|a_{n}-g'|+|a_{n}-g|<2\eps
\end{align*}

Пришли к противоречию, значит $g=g'\qed$

\pagebreak

\define{ограниченной последовательности}

Последовательность ограничена сверху, если множество её значений ограничено сверху:

Последовательность ограничена снизу, если множество её значений ограничено снизу.

Последовательность называется ограниченной, если она ограничена и сверху, и снизу.

\theorem

Если последовательность имеет предел, то она ограничена.

\proof

Пусть $a_{n}$ --- сходящаяся последовательность, а $g$ --- её предел:
\begin{align*}
	\llet\eps=1\;\exists k:(\forall n>k)\;g-1<a_{n}<g+1
\end{align*}

Введём $m$ и $M$:
\begin{align*}
	 & m:=\min\{a_1,a_2,...,a_{k},g-1\} &  & M:=\max\{a_1,a_2,...,a_{n},g+1\}
\end{align*}
Тогда:
\begin{align*}
	\begin{cases}
		m\leq a_{n}\leq M         & \forall n\leq k \\
		m\leq g-1<a_{n}<g+1\leq M & \forall n>k
	\end{cases}\Rarr (\forall n\in\N)\;m\leq a_{n}\leq M\qed
\end{align*}

{\it Замечание:}

Из ограниченности последовательность не следует ее сходимость.

Пример: $a_n=(-1)^n$ ограничена, но не сходится.

\end{document}
