\documentclass{article}

\usepackage{defines}

\begin{document}

\tickettitle{20}{Бесконечно большие величины, действия над ними, классификация бесконечно больших величин {\it Билет не проверен}}

\define

Если $\slim{n \to \infty} x_n=+\infty$ или $\slim{n \to \infty} x_n=-\infty$, то последовательность $\{x_n\}$ является бесконечно большой.


Отметим несколько свойств бесконечно больших последовательностей:
\begin{enumerate}
	\item{} Сумма.


	А) Если одно из слагаемых имеет конечный предел ($\slim{x \to a}x_n=A$ ), другое – бесконечный ( $\slim{y \to a}y_n=\infty$), тогда по свойству бесконечно больших функций предел их алгебраической суммы равен бесконечности. При помощи символов это можно записать следующим образом: $A\pm \infty=\pm infty$ .

	Б) Если оба слагаемых имеют бесконечные пределы, то важно знать какого знака эти пределы. Если обе функции имеют бесконечные пределы одинаковых знаков ($\slim{y \to a}x_n=+\infty$;$\slim{y \to a}y_n=+\infty$  или $\slim{y \to a}x_n=-\infty$;$\slim{y \to a}y_n=-\infty$ ), то по свойству бесконечно больших функций предел их суммы будет равен бесконечности того же знака: $+\infty+\infty=+infty$ или $(-\infty)+(-\infty)=-infty$. Если обе функции являются бесконечно большими разных знаков ( $\slim{y \to a}x_n=-\infty$;$\slim{y \to a}y_n=+\infty$ ), то ничего определенного о пределе их суммы сказать нельзя (величина предела зависит от конкретного примера), $\infty - \infty$ - неопределенность.

	\item{} Произведение.

	А) Если один из множителей имеет конечный предел, отличный от нуля, а другой множитель стремится к бесконечности ($\slim{x \to a}x_n=A\neq 0$; $\slim{y \to a}y_n=\infty$), то по свойству бесконечно больших функций предел произведения равен бесконечности: $A*\infty=\infty$.

	Б) Если оба множителя стремятся к бесконечности ( $\slim{y \to a}x_n=\infty$;$\slim{y \to a}y_n=\infty$), то по свойству бесконечно больших функций предел произведения равен бесконечности, при чем со знаком плюс (минус), если эти функции являются бесконечно большими одного знака (разного знака):  $\infty*\infty=\infty$.

	В) Если же предел одного множителя равен нулю, у другого бесконечности ($\slim{y \to a}x_n=\infty$;$\slim{y \to a}y_n=0$ ), то ничего определенного о пределе их произведения сказать нельзя:  $\infty*0$ - неопределенность.

	\item Частное.

	      А) Если $\slim{x \to a}x_n=A\neq 0$; $\slim{y \to a}y_n=0$, то предел их частного по теореме о связи между бесконечно малыми и бесконечно большими функциями будет равен бесконечности: $\frac{A}{0}=\infty$, так как  $\frac{A}{0}=A*\frac{1}{0}=A*\infty=\infty$.

	      Б) Если $\slim{x \to a}x_n=A$; $\slim{y \to a}y_n=\infty$ , то предел их частного по теореме о связи между бесконечно малыми и бесконечно большими функциями будет равен нулю: $\frac{A}{\infty}=0$.

	      Действительно, $\frac{A}{\infty}=A*\frac{1}{\infty}=A*0=0$.

	      С) Если наоборот  $\slim{x \to a}x_n=\infty$; $\slim{y \to a}y_n=B$, то предел их частного равен бесконечности: $\frac{\infty}B=\infty$ .

	      Д) Если же обе функции стремятся к  бесконечности ($\slim{y \to a}x_n=\infty$;$\slim{y \to a}y_n=\infty$), то ничего определенного о пределе их частного сказать нельзя: $\frac{\infty}\infty$ - неопределенности.
\end{enumerate}


Классификация бесконечно больших величин:

Пусть $\exists \{x_n\}$ и $\{y_n\}$ - две бесконечно большие
\begin{enumerate}
	\item{}Если $\nexists \slim{n \to \infty}\frac{x_n}{y_n}$, то последовательности $\{x_n\}$ и $\{y_n\}$ несравнимы.
	\item{}Если $\slim{n \to \infty}\frac{x_n}{y_n}=p\neq 0$, то последовательности $\{x_n\}$ и $\{y_n\}$ одного порядка $x_n=O(y_n);y_n=O(x_n)$
	\item{}Если $ \slim{n \to \infty}\frac{x_n}{y_n}=1$, то последовательности $\{x_n\}$ и $\{y_n\}$ эквивалентны.
	\item{}Если $ \slim{n \to \infty}\frac{x_n}{y_n}=\infty$, то $x_n$ величина большего порядка малочти, чем $y_n$.
	\item{}Если $ \slim{n \to \infty}\frac{x_n}{y_n}=0$, то $y_n$ величина большего порядка малости, чем $x_n$.

\end{enumerate}

\end{document}
