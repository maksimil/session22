\documentclass{article}

\usepackage{defines}

\begin{document}

\tickettitle{73}{Интегрирование рациональных выражений и правильных дробей.}

\define{рационального выражения}

Пусть $P$, $Q$ --- многочлены с вещественными коэфицентами, тогда выражение $\frac{P(x)}{Q(x)}$ называется рациональной дробью с вещественными коэфицентами

Если $\deg P<\deg Q$, то дробь --- правильная, иначе --- неправильная

Неправильную дробь можно разложить на сумму многочлена и правильной дроби

\define{приводимого полинома}

Полином $Q$ приводим, если:
\begin{align*}
	\exists P,S:Q=PS\land \deg P\geq 1\land \deg S\geq 1
\end{align*}

Где $P,S,Q$ --- полиномы с вещественными коэфицентами

Из основной теоремы алгебры следует, что любой полином $\deg>2$ приводим

\define{простых дробей}

Простыми дробями называются дроби вида:
\begin{align*}
	 & \frac{A}{(x-p)^{k}},\;k\in\N &  & \frac{Ax+B}{(x^{2}+px+q)^{m}},\;m\in\N
\end{align*}

Где $A,B,p,q\in\R$

При этом $x^{2}+px+q$ не имеет вещественных корней

Другими словами: $f/g^{k}$ --- простая дробь, если $\deg f<\deg g$ и $g$ --- неприводимый со старшим коэфицентом $=1$

\theorem

Любая правильная рациональная дробь $\frac{P(x)}{Q(x)}$, где старший коэфицент $Q=1$ может быть представлена в виде суммы простых дробей

\proof

Разложим $Q$ на произведение $\mu$ неприводимых полиномов:
\begin{align*}
	 & Q=\prod_{k=1}^{\mu}g_{i}^{k_{i}},\;k_{i}\in\N &  & i\neq j\Rarr g_{i}\neq g_{j} &  & g_{i}\text{ --- неприводимый}
\end{align*}

Старший коэфицент $Q=1$ $\Rarr$ старшие коэфиценты $g_{i}=1$

\newcommand\Qp{\widetilde{Q}}
Возьмём $g_{1}^{k_1}$:
\begin{align*}
	 & \Qp:=\prod_{k=2}^{\mu}g_{i}^{k_{i}} &  & Q=\Qp g_{1}^{k_1}
\end{align*}

\pagebreak

Применим следущий алгоритм к $\Qp$ и $g_1^{k_1}$ (алгоритм Евклида):
\begin{align*}
	\text{Делим $\Qp$ на $g_1^{k}$ с остатком: }   & \Qp=Q_0g_1^{k_1}+R_1       &  & \deg R_1<\deg g_1^{k_1} \\
	\text{Делим $g_1^{k_1}$ на $R_1$ с остатком: } & g_1^{k_1}=Q_1R_1+R_2       &  & \deg R_2<\deg R_1       \\
	                                               & R_1=Q_2R_2+R_3             &  & \deg R_3<\deg R_2       \\
	                                               & ...                                                     \\
	                                               & R_{k-1}=Q_{k}R_{k}+R_{k+1} &  & \deg R_{k+1}<\deg R_{k} \\
	                                               & R_{k}=Q_{k+1}R_{k+1}       &  & R_{k+1}\setminus R_{k}
\end{align*}

Алгоритм заканчивается, когда $R_{k+1}$ делит $R_{k}$ без остатка ($R_{k+1}\setminus R_{k}$)

Алгоритм конечный, тк $\deg R_{k+1}$ на каждом шаге уменьшается, значит в конце концов $\deg R_{k+1}$ достигнет $0$ $\Rarr R_{k+1}=\const\Rarr R_{k+1}\setminus R_{k}$
\begin{align*}
	 & R_{k+1}\setminus R_{k}\land R_{k-1}=Q_{k}R_{k}+R_{k+1}\Rarr R_{k+1}\setminus R_{k-1}\Rarr R_{k+1}\setminus R_{k-2}\Rarr...                            \\
	 & ...\Rarr R_{k+1}\setminus \Qp\land R_{k+1}\setminus g_1^{k_1}\Rarr R_{k+1}=:c=\const\text{ (по построению $\Qp$ и $g_1^{k_1}$ не имеют общих корней)}
\end{align*}

$R_{k+1}$ по $k+1$ шагу можно выразить через $R_{k-1}$ и $R_{k}$.

Аналогично можно выразить $R_{k}$,  $R_{k-1}$ и тд...

Можно прийти к следущему выводу:
\begin{align*}
	 & \exists U,V:c=R_{k+1}=cU\Qp+cVg_1^{k_1}\Rarr U\Qp+Vg_1^{k_1}=1                         \\
	 & \frac{P}{Q}=\frac{PU\Qp+PVg_1^{k_1}}{g_1^{k_1}\Qp}=\frac{PV}{\Qp}+\frac{PU}{g_1^{k_1}}
\end{align*}

Обозначим $PU$ как $f_1$:
\begin{align*}
	 & f_1:=PU &  & \frac{P}{Q}=\frac{PV}{\Qp}+\frac{f_1}{g_1^{k_1}}
\end{align*}

Применив эти шаги к $PV/\Qp$ и $g_2^{k_2}$, $g_3^{k_3}$, ..., $g_{\mu-1}^{k_{\mu-1}}$ получим:
\begin{align*}
	\frac{P}{Q}=\sum_{i=1}^{\mu}\frac{f_{i}}{g_{i}^{k_{i}}}
\end{align*}

Применим следущий алгоритм к $f_{i}$
\begin{align*}
	 & f_{i}=D_{1}g_{i}+A_{i0}=(D_2g_{i}+A_{i1})g_{i}+A_{i0}=D_2g_{i}^{2}+A_{i1}g_{i}+A_{i0}=\sum_{j=0}^{k}A_{ij}g_{i}^{j} &  & \deg A_{ij}<\deg g_{i}
\end{align*}

Алгоритм конечен, тк состоит только из $k$ шагов
\begin{align*}
	 & \frac{P}{Q}=\sum_{i=1}^{\mu}\frac{f_{i}}{g_{i}^{k_{i}}}=\sum_{i=1}^{\mu}\sum_{j=0}^{k_{i}}\frac{A_{ij}g_{i}^{j}}{g_{i}^{k_{i}}}
	=\sum_{i=0}^{\mu}\sum_{j=0}^{k_{i}}\frac{A_{ij}}{g_{i}^{k_{i}-j}}=\left(\sum_{i=0}^{\mu}A_{ik_{i}}\right)+\sum_{i=0}^{\mu}\sum_{j=0}^{k_{i}-1}\frac{A_{ij}}{g_{i}^{k_{i}-j}} \\
	 & S:=\sum_{i=0}^{\mu}A_{ik_{i}}
\end{align*}

\pagebreak

Пусть $S\neq 0$:
\begin{align*}
	\frac{P}{Q}=S+...\Rarr P=QS+...\Rarr \deg P=\deg Q+\deg S\Rarr \deg P\geq \deg Q\text{, но }\deg P<\deg Q\Rarr S=0
\end{align*}

$\deg A_{ij}<\deg g_{i}$ и $g_{i}$ --- неприводимый со старшим коэфицентом $=1$ $\Rarr$ $A_{ij}/g_{i}^{k_{i}-j}$ --- простая дробь

Тогда:
\begin{align*}
	\frac{P}{Q}=\sum_{i=0}^{\mu}\sum_{j=0}^{k_{i}-1}\frac{A_{ij}}{g_{i}^{k_{i}-j}}\qed
\end{align*}

\theorem

Если интеграл можно свести к интегралу от рациональной дроби, то его можно вычислить через элементарные функции

\proof

Рациональную дробь можно выразить как сумму простых дробей, а простые дроби интегрируемы в элементарных функциях (материал 74 билета)$\qed$

\hd{Формула Остроградского}
\begin{align*}
	\int\frac{P(x)}{Q(x)}dx=\int\frac{P_1(x)}{Q_1(x)}dx+\int\frac{P_2(x)}{Q_2(x)}dx
\end{align*}

$P_1(x)/Q_1(x)$ --- рациональная часть интеграла

$Q=Q_1Q_2$, $\deg P<\deg Q$
\end{document}
