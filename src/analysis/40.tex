\documentclass{article}

\usepackage{defines}

\begin{document}

\tickettitle{40}{Свойства функции, обратной строго монотонной непрерывной функции. Теорема о пределе корня.}

\theorem

$f$ --- непрерывна и строго монотонна на $[a;b]$ $\Rarr$
$f^{-1}$ --- непрерывна и строго монотонна на $f([a;b])$

\proof

\begin{enumerate}
	\item{}Докажем, что $f$ взаимно-однозначна на $[a;b]$
	\begin{align}
		 & \llet f(x_1)=f(x_2)\notag                                                               \\
		 & x_1<x_2\Rarr f(x_1)<f(x_2)\Rarr f(x_1)\neq f(x_2)\text{, но }f(x_1)=f(x_2)\label{40:lt} \\
		 & x_1>x_2\Rarr f(x_1)>f(x_2)\Rarr f(x_1)\neq f(x_2)\text{, но }f(x_1)=f(x_2)\label{40:gt}
	\end{align}
	\begin{align*}
		\eqref{40:lt}\land\eqref{40:gt}\Rarr x_1=x_2
	\end{align*}
	\item{}Докажем теорему, используя материал предыдущих билетов:

	По т. Дарбу(Билет 36): $f([a;b])=[c;d]$

	$f$ рассматривается на $[a;b]$, то есть $f$ --- непрерывна означает, что она непрерывна на $[a;b]$.

	Аналогично сокращаются утверждения про $f^{-1}$ на $[c;d]$

	По теореме о непрерывности обратной функции (Билет 37): 
 
        $f$ --- взаимно-однозначная $\land$ $f$ --- непрерывна $\Rarr$ $f^{-1}$ --- непрерывна

	По теореме о строгой монотонности непрерывной взаимно-однозначной функции (Билет 38): 
 
 $f^{-1}$ --- взаимно-однозначная $\land$ $f^{-1}$ --- непрерывна $\Rarr$ $f^{-1}$ --- строго монотонна$\qed$
\end{enumerate}


\theorem

$f(x)=\sqrt[n]{x},\;n\in\N$ --- непрерывна на $[0;+\infty)$ при чётных $n$ и непрерывна на $\R$ при нечётных $n$

\proof

\begin{enumerate}
	\item{}$n$ --- чётно

	$g(x)=x^{n}$ ---  непрерывна и строго монотонна на $[0;+\infty)$ $\Rarr$

	$\Rarr$ $g^{-1}(x)=f(x)=\sqrt[n](x)$ --- непрерывна и строго монотонна на $g([0;+\infty))=[0;+\infty)$

	\item{}$n$ --- нечётно

	$g(x)=x^{n}$ ---  непрерывна и строго монотонна на $\R$ $\Rarr$

	$\Rarr$ $g^{-1}(x)=f(x)=\sqrt[n](x)$ --- непрерывна и строго монотонна на $g(\R)=\R$$\qed$
        \item{} В обоих случаях $\lim_{k\to \infty} \sqrt[n]{x_{k}}=\sqrt[n]{x_{0}}$, если $lim_{k\to \infty} x_{k}=x_{0}$ ($x_{0} \geq 0$ при $n$ - четном и $x_{0}b\in R$ при $n$ - нечетном)
\end{enumerate}

\end{document}