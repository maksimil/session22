\documentclass{article}

\usepackage{defines}

\begin{document}

\tickettitle{22}{Вещественная функция одного вещественного аргумента, её график. Примеры. Убывающие и возрастающие функции. {\it Билет не проверен}}

\define

Если область значений $Y$ функции $f
	(x)$ есть числовая ось $\R$ (расширенная числовая ось $(-\infty,+\infty)$), то $f(x)$ называют числовой функцией или функцией вещественногО аргумента.

\define

Графиком функции $f(x)$ с областью определения $X$ и областью значений $Y$ назовем подмножество прямого произведения $P(x,y)$, состоящее из тех пар $(x,y)$, для которых $y=f(x)$, то есть $P(x,f(x))$.

Существует несколько способов задания функции:
\begin{enumerate}
	\item{}аналитический (формулой)
	\item{}графический (задаются специальные функции. Например, функция Дирихле)
	\item табличный (задаёт функцию таблицей, содержащей значения аргумента х и соответствующие значения функции , например, таблица логарифмов)
	\item графический (состоит в изображении графика функции – множества точек (x,y))
\end{enumerate}

\define

Функция возрастает, если значение $f(x)$ увеличивается с ростом значения $x$.
Функция убывает, если значение $f(x)$ уменьшается с ростом значения $x$

\define

Функция $f$, определенная на множестве $E$, называется строго возрастающей (строго убывающей), если для любых двух чисел $x_1 \in E$, $x_2 \in E$ таких, что $x_1 < x_2$, выполняется неравенство $f(x_1) < f(x_2)$ (соответственно $f(x_1) > f(x_2)$). Функция, строго возрастающая или строго убывающая, называется строго монотонной.

\define

Функции: линейная $y=C$ ($C$ постоянная), степенная $y=x^a$, где $a \in \R$, показательная $x=a^x$, $a > 0$, логарифмическая $y = log_a x, a > 0, a \geq 1,$ тригонометрические $y = \sin x$, $y = \cos x$, $y = \tg x$, $y = \ctg x$ и обратные тригонометрические функции $y = \arcsin x$, $y = \arccos x$, $y = \arctg x$, $y = \arcctg x$ называются основными элементарными функциями.

Всякая функция $f$, которая может быть задана с помощью формулы $y = f(x)$, содержащий лишь конечное число арифметических операций над основными элементарными функциями и композиций, называется элементарной
функцией.

В множестве элементарных функций выделяются следующие классы:
\begin{enumerate}
	\item{}Полиномы (многочлены) - функции вида
	\begin{align*}
		P(x) = a_nx^n + . . . + a_1x + a_0
	\end{align*}
	Если $a_n \geq 0$, то целое неотрицательное число $n$ называется степенью многочлена $P(x)$.
	Функция, тождественно равная нулю, является в силу данного определения многочленом, ей будем (это не общепринято) приписывать степень ноль.
	\item{}Рациональные функции - функции $f(x)$ представимые в виде
	\begin{align*}
		f(x) = \frac{P(x)}{Q(x)},
	\end{align*}
	где P и Q - многочлены (Q - ненулевой многочлен).
	Функция $f$ определена
	во всех точках R, кроме тех, в которых знаменатель Q обращается в ноль.
	\item{} Иррациональные функции, т. е. такие функции, не являющиеся рациональными, которые могут быть заданы композицией конечного числа рациональных
	функций, степенных функций с рациональными показателями и четырех арифметических действий. Например, функция
	\begin{align*}
		f(x) = \sqrt[5]{\frac{x-1}{x^2+\sqrt{x}}}
	\end{align*}
	является иррациональной функцией.
	\item{}Трансцендентные функции - элементарные функции не являющиеся рациональными или иррациональными.
	Все прямые и обратные тригонометрические функции являются трансцендентными.
\end{enumerate}

\end{document}
