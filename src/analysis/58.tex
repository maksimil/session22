\documentclass{article}

\usepackage{defines}

\begin{document}

\tickettitle{58}{Построение графиков функций с помощью дифференциального исчисления: асимптоты {\it Билет не просмотрен Ксюшей, но проверен Артёмом}}

Построить график функции означает изучить свойства функции и изобразить их графически. Свойства, которые можно изобразить графически:
\begin{enumerate}
	\item нули функции $\ \  f(x) = 0$
	\item непрерывность
	\item дифференцируемость
	\item точки разрывов функции $f(x)$ и $f'(x)$
	\item периодичность
	\item чётность $\backslash$ нечётность
	\item монотонность
	\item локальные extr-мы $f(x)$
	\item поведение $f(x)$ при $x \to \pm \infty$
	\item ассимптота
	\item выпуклость $\backslash$ вогнутость функции
	\item точки перегиба функции
\end{enumerate}

Под кривой или $y = f(x)$ мы будем понимать множество всех точек графика этой функции. В частности, прямая линия -- это график линейной функции $y = a \cdot x + b$. \\
Мы будем говорить, что точки кривой $y = f(x)$ лежат над прямой $y = \varphi (x)$ (или под прямой), если выполняется $f(x) > \varphi (x) \ (f(x) < \varphi (x))$ \\

Будем говорить, что точки некоторой кривой лежат по разные стороны от прямой $y = \varphi (x)$, если выполняется следующие условия: \\
$x \ne z$ \\
$(f(x) > \varphi (x)) \wedge (f(z) < \varphi (z)) \quad$ \\
$(f(x) < \varphi (x)) \wedge (f(z) > \varphi (z))$ \\

\define{} \\
Прямая $y = a \cdot x + b$ называется наклонной асимптотой кривой $y = f(x)$, если $\lim_{x \to \pm \infty} [f(x) - a \cdot x - b] = 0$.
$\Rightarrow \ \lim_{x \to \pm \infty} \frac{f(x)}{x} = a$, $\lim_{x \to \pm \infty} [f(x) - a \cdot x] = b$ \\


Из определения следует, что асимптота занимает то предельное положение, к которому стремится касательная $kk'$ кривой $y = f(x)$ в $(\cdot) (x, f(x))$ при $x \to \pm \infty$ \\


\define{} \\
Прямая $x = c,$ параллельная оси $Oy$, --- вертикальная асимптота кривой $y = f(x)$, если $\lim_{x \to c-0} f'(x) = \pm \infty = \lim_{x \to c-0} f(x) \vee \lim_{x \to c+0} f'(x) = \pm \infty = \lim_{x \to c+0} f(x)$ \\

\define{} \\
Прямая $y = c,$ параллельная оси $Ox$, --- горизонтальная асимптота кривой $y = f(x)$, если $\lim_{x \to +\infty} f(x) = \lim_{x \to -\infty} f(x) = b$ \\

\textit{Пример 1:} \\
$y = \frac{1}{x}, \ x \in \R$ \\
$Ox$ - горизонтальная асимптота \\

$Oy$ - вертикальная асимптота\\
\end{document}
