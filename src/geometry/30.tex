\documentclass{article}

\usepackage{defines}

\begin{document}

\tickettitle{30}{Понятие эллипса. Понятия эксцентриситета, фокального радиуса, директрисы. Теорема,~связывающая~фокальный~радиус~и~расстояние~до~директрисы. Уравнение~эллипса~в~полярных~координатах}

\define{эксцентриситета}

Эксцентриситет эллипса $\varepsilon$ --- отношение половины фокусного расстояния к длине большой полуоси: $$\varepsilon := \frac{c}{a}$$
\begin{align*}
	 & 2a>2c\Rarr\eps\in [0;1) &  & \varepsilon=\sqrt{1-\frac{b^2}{a^2}} &  & \frac{b}{a}=\sqrt{1-\varepsilon^2}
\end{align*}


\begin{minipage}{0.6\linewidth}
	\define{фокальных радиусов}

	Фокальные радиусы эллипса $r_1$ и $r_2$ --- отрезки, соединяющие любую точку эллипса $M (x, y)$ с его фокусами.
	\begin{align*}
		r_1 & =a+\eps x & r_2 & =a-\eps x
	\end{align*}
\end{minipage}%
\begin{minipage}{0.4\linewidth}
	\centering
	\begin{tikzpicture}[>=stealth]
		\coordinate (M) at (1.2, 0.81) node[point, anchor=south] at (M) {$M(x,y)$};
		\draw [->, thin] (-2.5,0) -- (2.5,0) node[above] {$x$};
		\draw [->, thin] (0,-1.5) -- (0,1.5) node[left] {$y$};
		\draw [dashed] ellipse (2.0 and 1.0);
		\draw [thin, red, text=black] (-1.5, 0) node[below]{$F_1$} node[point]{} -- node[above, text=red]{$r_1$} (M);
		\draw [thin, red, text=black] (1.5, 0) node[below]{$F_2$} node[point]{}  -- node[right, text=red]{$r_2$} (M);
		\draw (0, 0) node[below right]{$O$} node[point]{};
		\draw (M) node[point]{};
	\end{tikzpicture}
	\captionof{figure}{Фокальные радиусы}
\end{minipage}

\vspace{2em}

\begin{minipage}{0.6\linewidth}
	\define{директрис}

	Директрисы --- прямые $x=\pm \frac{a}{\varepsilon}$

	\define{фокального параметра}

	Фокальный параметр $p$ --- расстояние от фокуса до соответствующей ему дирекстриссы.
\end{minipage}%
\begin{minipage}{0.4\linewidth}
	\centering
	\begin{tikzpicture}[>=stealth]
		\draw [->, thin] (-3.0,0) -- (3.0,0) node[above] {$x$};
		\draw[->, thin] (0,-1.5) -- (0,1.5) node[left] {$y$};
		\draw[dashed] ellipse (2.0 and 1.0);
		\draw[thick, blue](-2.2, 0) -- node[above, xshift=0.3em]{$p$} (-1.5, 0);
		\draw[thick, blue](1.5, 0) -- node[above, xshift=-0.3em]{$p$} (2.2, 0);
		\draw (-1.5, 0) node[below]{$F_1$};
		\draw (1.5, 0) node[below]{$F_2$};
		\draw[thin, red] (-2.2, -1.5) -- (-2.2, 1.5) node[left, black]{$x=-\frac{a}{\eps}$};
		\draw[thin, red] (2.2, -1.5) -- (2.2, 1.5) node[right, black]{$x=\frac{a}{\eps}$};
		\draw  (1.5, 0) node[point]{};
		\draw (-1.5, 0) node[point]{};
		\draw (0, 0) node[below right]{$O$} node[point]{};
	\end{tikzpicture}
	\captionof{figure}{Директрисы}
\end{minipage}

\pagebreak

\theorem
\begin{align*}
	\varepsilon=\frac{r}{d}
\end{align*}

$\varepsilon$ --- эксцентриситет эллипса

$r$ --- расстояние от точки до фокуса

$d$ --- расстояние от точки до директрисы, соответствующей фокусу

\proof

\begin{minipage}{0.6\linewidth}
	\begin{align*}
		\frac{r}{d}=\frac{a-\eps x}{|x-\frac{a}{\eps}|}=\eps\frac{a-\eps x}{|a-\eps x|} = \eps\qed
	\end{align*}
\end{minipage}%
\begin{minipage}{0.4\linewidth}
	\centering
	\begin{tikzpicture}[>=stealth]
		\draw [->, thin] (-3.0,0) -- (3.0,0) node[above] {$x$};
		\draw[->, thin] (0,-1.5) -- (0,1.5) node[left] {$y$};
		\draw[dashed] ellipse (2.0 and 1.0);
		\draw (-1.5, 0) node[below]{$F_1$};
		\draw [thin, red, text=black](1.2, 0.81) node[above]{$M$}-- node[above, text=red]{$d$} (2.2, 0.81);
		\draw [thin, blue, text=black] (1.2, 0.81) node[point]{} -- node[right, text=blue]{$r$} (1.5, 0);
		\draw (1.5, 0) node[below]{$F_2$};
		\draw[thin] (2.2, -1.5) -- (2.2, 1.5) node[right, black]{$x=\frac{a}{\eps}$};
		\draw  (1.5, 0) node[point]{};
		\draw (-1.5, 0) node[point]{};
		\draw (0, 0) node[below right]{$O$} node[point]{};
	\end{tikzpicture}
	\captionof{figure}{}
\end{minipage}

\define{эллипса 2}

Эллипс --- геометрическое	место	точек, для каждой	из которых отношение расстояния $r$ до некоторой фиксированной точки $F_1$ (фокуса) к расстоянию $d$ до некоторой фиксированной	прямой (директрисы)	есть величина постоянная и меньше единицы: $$\varepsilon=\frac{r}{d}<1$$

\sectitle{Уравнение эллипса в полярных координатах}

\vspace{1em}

\begin{minipage}{0.6\linewidth}
	\begin{enumerate}
		\item Полюс в фокусе $F_1$
		\item Полярная ось $\uparrow \uparrow \overrightarrow{F_1F_2}$
	\end{enumerate}
\end{minipage}%
\begin{minipage}{0.4\linewidth}
	\centering
	\begin{tikzpicture}[>=stealth]
		\draw [->, thin] (-3.0,0) -- (3.0,0) node[above] {$x$};
		\draw [->, thin] (0,-1.5) -- (0,1.5) node[left] {$y$};
		\draw [dashed] ellipse (2.0 and 1.0);
		\draw [thin, red, text=black] (-1.5, 0) node[below]{$F_1$} node[point]{} -- node[left, yshift=0.3em, text=red]{$\rho$} (1.2, 0.81);
		\draw [thin] (2.2, -1.5) -- (2.2, 1.5);
		\draw [thin] (-2.2, -1.5) -- (-2.2, 1.5);
		\draw [thin, blue, text=black] (-2.2, 0.81) node[left]{$M_y$} node[point]{} -- node[above, text=blue, xshift=-2em] {$d$} (1.2, 0.81) node[above]{$M(\rho, \varphi)$};
		\draw [dashed] (1.2, 0.81) node[point]{} -- (1.2, 0) node[below]{$M_x$} node[point]{};
		\draw (-2.2, 0) node[below left]{$J_1$};
		\draw (1.5, 0) node[point]{} node[above]{$F_2$};
		\draw (0, 0) node[below right]{$O$} node[point]{};
	\end{tikzpicture}
	\captionof{figure}{Эллипс в полярных координатах}
\end{minipage}

Рассмотрим $M(\rho, \varphi)$ на эллипсе:
\begin{align*}
	r    & =|\overrightarrow{F_1M}|=\rho                                                              \\
	d    & =|\overrightarrow{M_yM}|=\mes_{Ox}\lvec{M_{y}M}=\mes_{Ox}(\lvec{J_1F_1}+\lvec{F_1M_{x}})=  \\
	     & =\mes_{Ox}\lvec{J_1F_1}+\mes_{Ox}\lvec{F_1M_{x}}=p+\rho\cos\varphi                         \\
	\eps & =\frac{r}{d}=\frac{\rho}{p+\rho \cos\varphi} \Rarr \rho=\frac{p\eps}{1-\eps \cos{\varphi}}
\end{align*}

\end{document}
