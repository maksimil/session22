\documentclass{article}

\usepackage{defines}

\usepackage{pgfplots}

\usepackage{tikz}

\usepackage{amsmath}

\usepackage{enumerate}

\begin{document}

\tickettitle{30}{Понятие эллипса. Понятия эксцентриситета, фокального радиуса, директрисы. Теорема, связывающая фокальный радиус и расстояние до директрисы. Уравнение эллипса в полярных
координатах}

\tikzset{
  outer dot/.style = {scale=4.5*sqrt(\pgflinewidth)},
  inner dot/.style = {scale=sqrt(\pgflinewidth),#1},inner dot/.default={black},
  point/.style={insert path={ node[outer dot]{.} node[inner dot=#1]{.}}}
}

\define{эллипса}

Эллипс --- это	геометрическое	место	точек,	для	каждой	из	которых	отношение	расстояния $r$ до некоторой фиксированной точки	$F_1$ (фокуса) к расстоянию	$d$ до некоторой фиксированной	прямой (директрисы)	есть	величина постоянная	и меньше единицы: $$\varepsilon=\frac{r}{d}<1$$

\define{эксцентриситета}

Эксцентриситет эллипса $\varepsilon$ --- отношение половины фокусного расстояния к длине большой полуоси: $$\varepsilon = \frac{c}{a}$$

$0 \leq \varepsilon <1$, т. к. $2a>2c$
\begin{align*} 
    & \varepsilon=\sqrt{1-\frac{b^2}{a^2}}, \ \ \frac{b}{a}=\sqrt{1-\varepsilon^2}
\end{align*}

\define{фокальных радиусов}

Фокальные радиусы эллипса $r_1$ и $r_2$ --- отрезки, соединяющие любую точку эллипса $M (x, y)$ с его фокусами.

$r_1=d(F_1, M)=a+\frac{xc}{a}=a+\varepsilon x$

$r_2=d(F_2, M)=a-\frac{xc}{a}=a-\varepsilon x$
\begin{flushright}
    \begin{tikzpicture}[>=stealth]
        \draw [->, thin] (-2.5,0) -- (2.5,0);
        \draw[->, thin] (0,-1.5) -- (0,1.5); 
        \draw[dashed] ellipse (2.0 and 1.0);
        \draw[thin, red, text=black] (-1.5, 0) node[below]{$F_1$} node[point]{} -- node[above]{$r_1$} (1.2, 0.81) node[above]{$M(x, y)$};
        \draw[thin, red, text=black] (1.5, 0) node[below]{$F_2$} node[point]{}  -- node[right]{$r_2$} (1.2, 0.81) node[point]{} ;
        \draw (0, 0) node[below right]{$O$} node[point]{};
    \end{tikzpicture}
\end{flushright}

\define{директрис}

Директрисы --- прямые $x=\pm \frac{a}{\varepsilon}$
\begin{flushright}
    \begin{tikzpicture}[>=stealth]
            \draw [->, thin] (-3.0,0) -- (3.0,0);
            \draw[->, thin] (0,-1.5) -- (0,1.5); 
            \draw[dashed] ellipse (2.0 and 1.0);
            \draw[thin, green, text=black](-2.2, 0) -- node[above]{$p$} (-1.5, 0);
            \draw[thin, green, text=black](1.5, 0) -- node[above]{$p$} (2.2, 0);
            \draw (-1.5, 0) node[below]{$F_1$};
            \draw (1.5, 0) node[below]{$F_2$};
            \draw[thin, red] (-2.2, -1.5) -- (-2.2, 1.5) node[left, black]{$x=-\frac{a}{c}$};
            \draw[thin, red] (2.2, -1.5) -- (2.2, 1.5) node[right, black]{$x=\frac{a}{c}$};
            \draw  (1.5, 0) node[point]{};
            \draw (-1.5, 0) node[point]{};
            \draw (0, 0) node[below right]{$O$} node[point]{};
    \end{tikzpicture}
\end{flushright}

\define{фокального параметра}

Фокальный параметр $p$ --- расстояние от фокуса до соответствующей ему дирекстриссы.

\theorem 

$$\varepsilon=\frac{r}{d}$$
$\varepsilon$ --- эксцентриситет эллипса

$r$ --- расстояние от точки до фокуса

$d$ --- расстояние от точки до директрисы, соответствующей фокусу

\proof 

\begin{align*}
    \frac{r}{d}=\frac{a-\varepsilon x}{|x-\frac{a}{\varepsilon}|}=\frac{a-\varepsilon x}{\frac{a-\varepsilon x}{\varepsilon}} \stackrel{\text{т. к. директрисса левее точки M}} = \frac{a-\varepsilon x}{a-\varepsilon x}*\varepsilon
\end{align*}

\begin{flushright}
  
    \begin{tikzpicture}[>=stealth]
            \draw [->, thin] (-3.0,0) -- (3.0,0);
            \draw[->, thin] (0,-1.5) -- (0,1.5); 
            \draw[dashed] ellipse (2.0 and 1.0);
            \draw (-1.5, 0) node[below]{$F_1$};
            \draw [thin, red, text=black](1.2, 0.81) node[above]{$M$}-- node[above]{$d$} (2.2, 0.81);
            \draw [thin, red, text=black] (1.2, 0.81) node[point]{} -- node[right]{$r$} (1.5, 0);
            \draw (1.5, 0) node[below]{$F_2$ (c, 0)};
            \draw[thin] (2.2, -1.5) -- (2.2, 1.5) node[right, black]{$x=\frac{a}{c}$};
            \draw  (1.5, 0) node[point]{};
            \draw (-1.5, 0) node[point]{};
            \draw (0, 0) node[below right]{$O$} node[point]{};
    \end{tikzpicture}   
\end{flushright} 
\begin{center}
  Уравнение эллипса в полярных координатах
\end{center}
\begin{enumerate}[$\bullet$]
    \item Полюс в фокусе $F_1$
    \item Полярная ось $\uparrow \uparrow \overrightarrow{F_1F_2}$
\end{enumerate}
\begin{flushright}
    \begin{tikzpicture}[>=stealth]
            \draw [->, thin, purple] (-3.0,0) -- (3.0,0);
            \draw[->, thin] (0,-1.5) -- (0,1.5); 
            \draw[dashed] ellipse (2.0 and 1.0);
            \draw[thin, red] (-1.5, 0) node[below, black]{$F_1$} node[point]{} -- node[above left]{$\rho$} (1.2, 0.81);
            \draw[thin] (2.2, -1.5) -- (2.2, 1.5);
            \draw[thin] (-2.2, -1.5) -- (-2.2, 1.5);
            \draw[thin, blue, text=black](-2.2, 0.81) node[left]{$M_y$} node[point]-- (1.2, 0.81) node[above]{$M(\rho, \varphi)$};
            \draw[dashed](1.2, 0.81) node[point]{} -- (1.2, 0) node[below]{$M_x$} node[point]{};
            \draw  (-2.2, 0) node[below left]{$J_1$};
            \draw  (1.5, 0) node[point]{} node[above]{$F_2$};
            \draw (0, 0) node[below right]{$O$} node[point]{};
    \end{tikzpicture}
\end{flushright}
$\llet M(\rho, \varphi)$
\begin{align*}
    & r=|\overrightarrow{F_1M}|=\rho \\    & d=|\overrightarrow{M_yM}|=|\overrightarrow{J_1F_1}+\overrightarrow{F_1M_x}|=|\overrightarrow{J_1F_1}|+\overrightarrow{F_1M_x}=p+\rho \cos\varphi \\
    & \varepsilon=\frac{r}{d}=\frac{\rho}{p+\rho \cos\varphi} \Rarr \rho=\frac{p\varepsilon}{1-\varepsilon \cos{\varphi}}
\end{align*}

\end{document}