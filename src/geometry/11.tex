\documentclass{article}

\usepackage{defines}

\begin{document}

\tickettitle{11}{Понятие скалярного произведения векторов. Теорема об ортогональности. Теорема о координатах вектора в ДПСК. Теорема о направляющих косинусах.}

\define{скалярного произведения векторов}

см. билет 10

\define{ортогональных векторов}

Ненулевые векторы $\vec{a}$ и $\vec{b}$ ортогональны, если $(\vec{a},\vec{b})=0$

\theorem

$\vec{a}\neq \vec{0}$, $\vec{b}\neq \vec{0}$
\begin{align*}
	(\vec{a},\vec{b})=0\Lrarr \vec{a}\perp\vec{b}
\end{align*}

\proof
\begin{align*}
	(\vec{a},\vec{b})=0\Lrarr |\vec{a}||\vec{b}|\cos\angle(\vec{a},\vec{b})=0\Lrarr \cos\angle(\vec{a},\vec{b})=0\Lrarr \angle(\vec{a},\vec{b})=\frac{\pi}{2}\Lrarr \vec{a}\perp\vec{b}\qed
\end{align*}

\theorem

$O\vec{e}_{1}\vec{e}_{2}...\vec{e}_{n}$ --- ДПСК $\Rarr$ $(\vec{a},\vec{e}_{i})$ --- $i$-я координата вектора $\vec{a}$

\proof

Разложим $\vec{a}$ по базису:
\begin{align*}
	 & \vec{a}=\sum_{j=1}^{n}\alpha_{j}\vec{e}_{j}                                                       \\
	 & (\vec{a},\vec{e}_{i})=\left(\sum_{j=1}^{n}\alpha_{j}\vec{e}_{j},\vec{e}_{i}\right)=\alpha_{i}\qed
\end{align*}

\define{направляющих векторов}

Направляющие косинусы вектора $\vec{a}$ --- $\cos\angle(\vec{a},\vec{i})$, $\cos\angle(\vec{a},\vec{j})$, $\cos\angle(\vec{a},\vec{k})$

\theorem[о направляющих косинусах]

Направляющие косинусы $\vec{a}$ --- координаты орта $\vec{a}$ в ДПСК

\proof

Найдём координаты орта $\vec{a}$:
\begin{align*}
	 & (\ort\vec{a})_x=(\ort\vec{a},\vec{i})=|\ort\vec{a}||\vec{i}|\cos\angle(\ort\vec{a},\vec{i})=\cos\angle(\vec{a},\vec{i})
\end{align*}
\begin{align*}
	 & (\ort\vec{a})_y=\cos\angle(\vec{a},\vec{j}) &  & (\ort\vec{a})_z=\cos\angle(\vec{a},\vec{j})\qed
\end{align*}

\end{document}
