\documentclass{article}

\usepackage{defines}

\begin{document}

\tickettitle{12}{Понятие Евклидового пространства. Теорема о неравенстве Коши-Буняковского. Теорема о нормированном пространстве.}

\define{евклидого пространства}

Линеал $\E^{n}$ называется евклидовым, если на нём задана функция $(\cdot):\E\times\E\to\R$ (скалярное произведение), причём выполняются аксиомы:
\begin{enumerate}[label=\Roman*.]
	\item{}\tikzmark{12:ax_1}$(\forall \vec{a},\vec{b}\in \E)\;\vec{a}+\vec{b}=\vec{b}+\vec{a}$
	\item{}$(\forall \vec{a},\vec{b},\vec{c}\in \E)\;(\vec{a}+\vec{b})+\vec{c}=\vec{a}+(\vec{b}+\vec{c})$
	\item{}$\exists \vec{0}\in \E:(\forall \vec{a}\in\E)\;\vec{a}+\vec{0}=\vec{a}$
	\item{}$(\forall \vec{a}\in \E)\;\exists \vec{b}\in\E:\vec{a}+\vec{b}=\vec{0}$
	\item{}$(\forall \vec{a}\in \E,\lambda, \mu\in\R)\;\lambda(\mu\vec{a})=(\lambda\mu)\vec{a}$
	\item{}$(\forall \vec{a}\in \E,\lambda,\mu\in\R)\;(\lambda+\mu)\vec{a}=\lambda\vec{a}+\mu\vec{a}$
	\item{}$(\forall \vec{a},\vec{b}\in \E,\lambda\in\R)\;\lambda(\vec{a}+\vec{b})=\lambda\vec{a}+\lambda\vec{b}$
	\item{}\tikzmark{12:ax_8}$(\forall \vec{a}\in \E)\;1\cdot\vec{a}=\vec{a}$

	\item{}\tikzmark{12:ax_9}$\E$ --- конечномерен
	\item{}$(\forall A\in\A,\vec{v}\in \E)\;\exists B\in\A:\lvec{AB}=\vec{v}$
	\item{}\tikzmark{12:ax_11}$(\forall A,B,C\in\A)\;\lvec{AB}+\lvec{BC}=\lvec{AC}$

	\item{}\tikzmark{12:ax_12}$(\forall \vec{a},\vec{b}\in\E)\;(\vec{a},\vec{b})=(\vec{b},\vec{a})$
	\item{}$(\forall \lambda\in\R,\vec{a},\vec{b}\in\E)\;(\lambda\vec{a},\vec{b})=\lambda(\vec{a},\vec{b})$
	\item{}$(\forall \vec{a},\vec{b},\vec{c}\in\E)\;(\vec{a},\vec{b}+\vec{c})=(\vec{a},\vec{b})+(\vec{a},\vec{c})$
	\item{}\tikzmark{12:ax_15}$(\forall \vec{a}\in\E)\;(\vec{a},\vec{a})>0\text{, если }\vec{a}\neq \vec{0}\land(\vec{a},\vec{a})=0\text{, если }\vec{a}=\vec{0}$
\end{enumerate}

\begin{tikzpicture}[remember picture, overlay]
	\def\offset{18em}
	\draw [decorate, decoration={brace}] ($({pic cs:12:ax_1}) + (\offset, 1em)$) -- ($({pic cs:12:ax_8}) + (\offset,0)$)
	node [midway, right] {Акисомы линеала $\E$};
	\draw [decorate, decoration={brace}] ($({pic cs:12:ax_9}) + (\offset, 1em)$) -- ($({pic cs:12:ax_11}) + (\offset,0)$)
	node [midway, right] {Акисомы аффинного пространства $\A$};
	\def\scaloffset{25em}
	\draw [decorate, decoration={brace}] ($({pic cs:12:ax_12}) + (\scaloffset, 1em)$) -- ($({pic cs:12:ax_15}) + (\scaloffset,0)$)
	node [midway, right] {Акисомы скалярного произведения};
\end{tikzpicture}

\theorem[(неравенство Коши-Буняковского)]
\begin{align*}
	(\vec{a},\vec{b})^{2}\leq (\vec{a},\vec{a})(\vec{b},\vec{b})
\end{align*}

\proof

Рассмотрим вектор $\lambda\vec{a}+\vec{b}$. По 15й аксиоме:
\begin{align}
	\label{12:deg_2}
	(\lambda\vec{a}+\vec{b},\lambda\vec{a}+\vec{b})\geq 0\Rarr(\vec{a},\vec{a})\lambda^{2}+2(\vec{a},\vec{b})\lambda+(\vec{b},\vec{b})\geq 0\;\forall\lambda\in\R
\end{align}

\begin{enumerate}
	\item{}$\vec{a}=\vec{0}$
	\begin{align*}
		(0,\vec{b})^{2}=(\vec{0},\vec{0})(\vec{b},\vec{b})\qed
	\end{align*}
	\item{}$\vec{a}\neq \vec{0}\Rarr(\vec{a},\vec{a})>0$

	Тогда $\eqref{12:deg_2}$ выполняется для $\forall\lambda\in\R$ при $D\leq 0$:
	\begin{align*}
		D=4(\vec{a},\vec{b})^{2}-4(\vec{a},\vec{a})(\vec{b},\vec{b})\leq 0\Rarr(\vec{a},\vec{b})^{2}\leq(\vec{a},\vec{a})(\vec{b},\vec{b})\qed
	\end{align*}
\end{enumerate}

\pagebreak

\define{нормированного пространства}

Линеал $L^{n}$ называется нормированным, если задано правило $x\mapsto||x||:L\to\R$, называемое нормой и выполняются аксиомы:
\begin{enumerate}[label=\Roman*.]
	\item{}$(\forall x\in L)\;||x||>0\text{, если } x>0;\;||x||=0\text{, если }x=0$
	\item{}$(\forall \lambda\in\R,x\in L)\;||\lambda x||=|\lambda|\cdot||x||$
	\item{}$(\forall x,y\in L)\;||x+y||\leq ||x||+||y||$
\end{enumerate}

\theorem

Всякое евклидово пространство $\E$ можно рассматривать как нормированное, определив норму:
\begin{align*}
	||\vec{a}||=\sqrt{(\vec{a},\vec{a})}
\end{align*}

\proof

Проверим аксиомы нормы:
\begin{enumerate}[label=\Roman*.]
	\item{}$(\forall x\in L)\;||x||>0\text{, если } x>0;\;||x||=0\text{, если }x=0$
	\begin{align*}
		 & \vec{a}=\vec{0}\Rarr (\vec{a},\vec{a})=0\Rarr||\vec{a}||=0 \\
		 & \vec{a}\neq 0\Rarr (\vec{a},\vec{a})>0\Rarr ||\vec{a}||>0
	\end{align*}
	\item{}$(\forall \lambda\in\R,x\in L)\;||\lambda x||=|\lambda|\cdot||x||$
	\begin{align*}
		||\lambda\vec{a}||=\sqrt{(\lambda\vec{a},\lambda\vec{a})}=\sqrt{\lambda^{2}(\vec{a},\vec{a})}=|\lambda|\sqrt{(\vec{a},\vec{a})}=|\lambda|\cdot||\vec{a}||
	\end{align*}
	\item{}$(\forall x,y\in L)\;||x+y||\leq ||x||+||y||$

	По неравенству Коши-Буняковского:
	\begin{align*}
		 & (\vec{a},\vec{b})^{2}\leq(\vec{a},\vec{a})(\vec{b},\vec{b})\Lrarr (\vec{a},\vec{b})\leq\sqrt{(\vec{a},\vec{a})(\vec{b},\vec{b})}
	\end{align*}
	\begin{align*}
		||\vec{a}+\vec{b}||
		 & =\sqrt{(\vec{a}+\vec{b},\vec{a}+\vec{b})}=
		\sqrt{(\vec{a},\vec{a})+2(\vec{a},\vec{b})+(\vec{b},\vec{b})}\leq
		\sqrt{\sqrt{(\vec{a},\vec{a})}^{2}+2\sqrt{(\vec{a},\vec{a})}\sqrt{(\vec{b},\vec{b})}+\sqrt{(\vec{b},\vec{b})}^{2}}            \\
		 & =\sqrt{||\vec{a}||^{2}+2||\vec{a}||\cdot||\vec{b}||+||\vec{b}||^{2}}=|||\vec{a}||+||\vec{b}|||=||\vec{a}||+||\vec{b}||\qed
	\end{align*}
\end{enumerate}

\end{document}
