\documentclass{article}

\usepackage{defines}

\begin{document}

\tickettitle{2}{Понятие линейного пространства. Линейная зависимость и независимость векторов. Теорема~о~линейной~зависимости. Теорема о нулевом векторе и линейной зависимости.}

\define{линейного пространства (линеала)}

Линеал --- множество $L$ обладающее следущими свойствами:
\begin{enumerate}
	\item{}Аддитивность

	Задано правило ($+$): $(\forall x,y\in L)\;x+y\in L$

	\item{}Произведение на число

	Задано правило ($\cdot$): $(\forall x\in L,\lambda\in\R)\;\lambda\cdot x\in L$

	\item{}Аксиомы:
	\begin{enumerate}[label=\Roman*.]
		\item{}$(\forall x,y\in L)\;x+y=y+x$
		\item{}$(\forall x,y,z\in L)\;(x+y)+z=x+(y+z)$
		\item{}$\exists 0\in L:(\forall x\in L)\;x+0=x$
		\item{}$(\forall x\in L)\;\exists y\in L:x+y=0$
		\item{}$(\forall x\in L,\lambda, \mu\in\R)\;\lambda(\mu x)=(\lambda\mu)x$
		\item{}$(\forall x\in L,\lambda,\mu\in\R)\;(\lambda+\mu)x=\lambda x+\mu x$
		\item{}$(\forall x,y\in L,\lambda\in\R)\;\lambda(x+y)=\lambda x+\lambda y$
		\item{}$(\forall x\in L)\;1\cdot x=x$
	\end{enumerate}
\end{enumerate}

Примеры линеалов:
\begin{enumerate}
	\item{}$V^{1}$ --- множество всех свободных векторов на прямой
	\item{}$V^{2}$ --- множество всех свободных векторов на плоскости
	\item{}$V^{3}$ --- множество всех свободных векторов в пространстве
	\item{}$\{0\}$ --- простейший линеал
	\item{}$\R^{n}$ --- множество всех упорядоченных множеств из $n$ чисел:
	\begin{enumerate}
		\item{}$(a_1,a_2,...,a_n)+(b_1,b_2,...,b_n)=(a_1+b_1,a_2+b_2,...,a_n+b_n)$
		\item{}$\lambda(a_1,a_2,...,a_n)=(\lambda a_1,\lambda a_2,...,\lambda a_n)$
	\end{enumerate}
\end{enumerate}

\define{линейной комбинации}
\begin{align}
	\label{2:lincomb}
	\sum_{i=1}^{n}a_iv_i,\;a_i\in\R,v_i\in L,i=\overline{1,n}
\end{align}

Выражение вида $\eqref{2:lincomb}$ называется линейной комбинацией элементов $v_i$,\\
а $a_i$ --- коэфицентами комбинации.

Тривиальная линейная комбинация: $a_i=0,\;i=\overline{1,n}$

\pagebreak

\define{разложения}

Разложением $p\in L$ по $e_i\in L$ называется линейная комбинация $e_i\in L$:
\begin{align*}
	p=\sum_{i=1}^{n}a_ie_i,\;a_i\in\R,i=\overline{1,n}
\end{align*}

\define{линейной зависимости}

Элементы $v_i\in L$ линейно зависимы, если $\exists a_i\in\R$:
\begin{align*}
	 & \sum_{i=1}^{n}a_iv_i=0 &  & \land &  & \sum_{i=1}^{n}a_i^{2}\neq 0
\end{align*}

Элементы $v_i\in L$ линейно независимы, если они не линейно зависимы.

лз --- линейно зависимы, лнз --- линейно независимы.

\theorem

Если среди $\setdef{v_i\in L}{i=\overline{1,n}}$ существует лз подсистема, то вся система лз.

Система --- синоним слова множество.

\proof

Упорядочим $v_i$ таким образом, чтобы лз подсистема была на первых $k$ местах:
\begin{align*}
	 & \exists a_i\in\R:(\sum_{i=1}^{k}a_iv_i=0)\land(\sum_{i=1}^{k}a_i^{2}\neq 0)
\end{align*}

\newcommand\ap{\widetilde{a}}
Определим $\ap_{i}$:
\begin{align*}
	 & \ap_{i}=a_{i},\quad i=\overline{1,k} &  & \ap_{i}=0,\quad i=\overline{k+1,n}
\end{align*}
\begin{align*}
	\sum_{i=1}^{n}\ap_{i}^{2}=\sum_{i=1}^{k}a_{i}^{2}+\sum_{i=k+1}^{n}0^{2}=\sum_{i=1}^{k}a_{i}\neq 0
\end{align*}

Тогда получим линейную комбинацию:
\begin{align*}
	 & \sum_{i=1}^{n}\ap_{i}v_{i}=\sum_{i=1}^{k}a_{i}v_{i}+\sum_{i=k+1}^{n}0\cdot v_{i}=0 &  & \sum_{i=1}^{n}\ap_{i}^{2}\neq 0
\end{align*}

Из её существования следует, что $\setdef{v_{i}\in L}{i=\overline{1,n}}$ --- лз система$\qed$

\result[1]

Если среди $v_{i}$ есть $0$ то система лз, тк $\{0\}$ --- лз система.

\pagebreak

\result[2]

Любая подсистема лнз системы тоже лнз, тк $(\exists \text{лз подсистемы} \Rarr \text{система лз})$ означает $(\text{система лнз} \Rarr \forall\text{подсистема лнз})$.

\theorem
\begin{align*}
	\{v_1,v_2,...,v_n\}\text{ --- лз}\Lrarr\exists v_{i}\text{, который можно выразить через остальные}
\end{align*}

\onlyif
\begin{align*}
	\sum_{i=1}^{n}\lambda_{i}v_{i}=0\land\exists i:\lambda_{i}\neq 0
\end{align*}

Без ограничения общности пусть $\lambda_{1}\neq 0$, тогда:
\begin{align*}
	\sum_{i=1}^{n}\lambda_{i}v_{i}=\lambda_1v_1+\sum_{i=2}^{n}\lambda_{i}v_{i}=0\Rarr\lambda_1v_1=-\sum_{i=2}^{n}\lambda_{i}v_{i}\Rarr
	v_1=\sum_{i=2}^{n}\left(-\frac{\lambda_{i}}{\lambda_1}\right)v_{i}\qed
\end{align*}

\enough

Без ограничения общности пусть $v_1$ можно выразить через остальные, тогда:
\begin{align*}
	v_1=\sum_{i=2}^{n}\lambda_{i}v_{i}
\end{align*}

\newcommand\lp{\widetilde{\lambda}}
Определим $\lp_{i}$:
\begin{align*}
	 & \lp_1=-1 &  & \lp_{i}=\lambda_{i},\quad i=\overline{2,n}
\end{align*}
\begin{align*}
	\sum_{i=1}^{n}\lp_{i}^{2}=1+\sum_{i=2}^{n}\lp_{i}^{2}\neq0
\end{align*}

Тогда получим линейную комбинацию:
\begin{align*}
	 & \sum_{i=1}^{n}\lp_{i}v_{i}=-v_1+\sum_{i=2}^{n}\lambda_{i}v_{i}=0 &  &
	\sum_{i=1}^{n}\lp_{i}^{2}\neq0
\end{align*}

Из её существования следует, что $\{v_1,v_2,...,v_n\}$ --- лз$\qed$

\pagebreak

\theorem
\begin{align*}
	\begin{cases}
		\{v_1,v_2,...,v_{n-1}\} & \text{ --- лнз} \\
		\{v_1,v_2,...,v_{n}\}   & \text{ --- лз}
	\end{cases}\Rarr\exists\alpha_{i}\in\R:v_{n}=\sum_{i=1}^{n-1}\alpha_{i}v_{i}
\end{align*}

\proof
\begin{align*}
	\{v_1,v_2,...,v_{n}\}\text{ --- лз}\Rarr\exists\alpha_{i}\in\R:\sum_{i=1}^{n}\alpha_{i}v_{i}=0\land\sum_{i=1}^{n}\alpha_{i}^{2}\neq 0
\end{align*}

Допустим, что $\alpha_{n}=0$:
\begin{align*}
	\begin{cases}
		\displaystyle
		\sum_{i=1}^{n}\alpha_{i}v_{i}=\sum_{i=1}^{n-1}\alpha_{i}v_{i}+0\cdot v_{n}=\sum_{i=1}^{n-1}\alpha_{i}v_{i}=0 \\[1.5em]
		\displaystyle
		\sum_{i=1}^{n}\alpha_{i}^{2}=\sum_{i=1}^{n-1}\alpha_{i}^{2}+0^{2}=\sum_{i=1}^{n-1}\alpha_{i}^{2}\neq 0
	\end{cases}\Rarr\{v_1,v_2,...,v_{n-1}\}\text{ --- лз}\text{, что противоречит условию}
\end{align*}

Учитывая, что $\alpha_{n}\neq 0$:
\begin{align*}
	\sum_{i=1}^{n}\alpha_{i}v_{i}=0\Rarr\sum_{i=1}^{n-1}\alpha_{i}v_{i}+\alpha_{n}v_{n}=0\Rarr\alpha_{n}v_{n}=-\sum_{i=1}^{n-1}\alpha_{i}v_{i}
	\Rarr v_{n}=\sum_{i=1}^{n-1}\left(-\frac{\alpha_{i}}{\alpha_{n}}\right)v_{i}\qed
\end{align*}

\end{document}
