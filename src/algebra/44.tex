\documentclass{article}

\usepackage{defines}

\begin{document}

\tickettitle{44}{Критерий Эйзенштейна. Поиск целочисленных и рациональных корней.}

\theorem[(критерий Эйзенштейна)]

Пусть $P(x)=a_n x^n+a_{n-1} x^{n-1}+...+a_1 x+a_0\in\Z[x]$

Если $\exists$простое $p$:
\begin{enumerate}
	\item{} $\lnot p\setminus a_n$
	\item{} $p\setminus a_i,\;i=\overline{0, n-1}$
	\item{} $\lnot p^2\setminus a_0$
\end{enumerate}

то $P$ неприводим над $\Q$

\proof

Пусть $P\in\Z[x]$ приводим над $\Q$ $\Rarr$ $P$ --- приводим над $\Z$:
\begin{align*}
	 & (\exists S,Q\in\Z[x])\;P=SQ\land d:=deg~S\geq 1\land l:=deg~Q\geq 1 \\
	 & S(x)=b_{d}x^{d}+...+b_0                                             \\
	 & Q(x)=e_{l}x^{l}+...+e_0
\end{align*}

Без ограничения общности пусть $d\geq l$

Рассмотрим коэфиценты $P=SQ$:
\begin{align*}
	 & a_0=b_0 e_0                                       \\
	 & a_1=b_0 e_1+b_1 e_0                               \\
	 & a_2=b_0 e_2+b_1 e_1+b_2 e_0                       \\
	 & ...                                               \\
	 & a_l=b_0 e_l+b_1 e_{l-1}+...+b_{l}e_0              \\
	 & a_{l+1}=b_1 e_{l}+b_2 e_{l-1}+...+b_{l+1}e_0      \\
	 & ...                                               \\
	 & a_{d}=b_{d-l} e_{l}+b_{d-l+1}e_{l-1}+...+b_{d}e_0
\end{align*}
\begin{align*}
	p\setminus a_0\land\lnot p^{2}\setminus a_0\land p\text{ --- простое}\Rarr (p\setminus b_0 \land \lnot p\setminus e_0)\lor (\lnot p\setminus b_0\land p\setminus e_0)
\end{align*}

\begin{enumerate}
	\item{}$p\setminus b_0\land\lnot p\setminus e_0$
	\begin{align*}
		 & p\setminus a_1\Rarr p\setminus b_0e_1+b_1e_0\Rarr p\setminus b_1 &  & p\setminus a_2\Rarr p\setminus b_2 &  & ... &  & p\setminus a_{d} \Rarr p\setminus b_{d}
	\end{align*}
	\begin{align*}
		p\setminus b_{d}\land a_{n}=b_{d}e_{l}\Rarr p\setminus a_{n}\text{, что противоречит п.1}\Rarr P\text{ --- неприводим над $\Q\qed$}
	\end{align*}

	\item{}$\lnot p\setminus b_0\land p\setminus e_0$
	\begin{align*}
		 & p\setminus a_1\Rarr p\setminus b_0e_1+b_1e_0\Rarr p\setminus e_1 &  & p\setminus a_2\Rarr p\setminus e_2 &  & ... &  & p\setminus a_{l} \Rarr p\setminus e_{l}
	\end{align*}
	\begin{align*}
		p\setminus e_{l}\land a_{n}=b_{d}e_{l}\Rarr p\setminus a_{n}\text{, что противоречит п.1}\Rarr P\text{ --- неприводим над $\Q\qed$}
	\end{align*}
\end{enumerate}

\sectitle{Поиск целочисленных и рациональных корней}

\theorem
\begin{align*}
	\alpha\in\mathbb{Z}\land S\in\mathbb{Z}[x]\land S(\alpha )=0\Rarr\alpha\setminus S(0)\land Q(x)=S(x)/(x-\alpha )\in\Z[x]
\end{align*}

\proof

Поделим $S(x)=a_n x^n+...+a_0$ на $x-\alpha$ по схеме Горнера, получив частное $Q(x)=b_{n-1} x^{n-1}+...+b_0$

Приравниваем коэффициенты при одинаковых степенях в представлении $S(x)=(x-\alpha)Q(x)$:
\begin{align*}
	 & a_{n}=b_{n-1}                  &  & b_{n-1}=a_{n}                  \\
	 & a_{n-1}=b_{n-2}-\alpha b_{n-1} &  & b_{n-2}=a_{n-1}+\alpha b_{n-1} \\
	 & ...                            &  & ...                            \\
	 & a_{1}=b_{0}-\alpha b_1         &  & b_0=a_1+\alpha b_1             \\
	 & a_0=-\alpha b_0
\end{align*}

Из схемы видно, что $(\forall i=\overline{0,n-1})\;b_i\in\Z$, тк $(\forall i=\overline{1,n})\;a_i\in\Z$ и $\alpha\in\Z$ $\Rarr$ $Q\in\Z[x]$

В частности, $b_0\in\Z\land a_0 = -\alpha b_0\in\Z \Rarr \alpha\setminus a_0\qed$

\algorithm{отсечений}

{\it Задача:} найти целые корни $S\in\Z[x]$

\begin{enumerate}
	\item{}Вычислим $S(1)$ и $S(-1)$
	\begin{align*}
		S(\alpha)=0\land\alpha\in\Z\Rarr S(x)=(\alpha-x)Q(x)\land Q\in\Z[x]\Rarr (\forall x\in\Z:S(x)\neq 0)\;(\alpha-x)\setminus S(x)
	\end{align*}

	На данном шаге имеем следущие отсечения:
	\begin{align}
		S(\alpha)=0\Rarr\alpha\setminus S(0)\land(\alpha-1)\setminus S(1)\land (\alpha+1)\setminus S(-1)\label{44:sec}
	\end{align}
	\item{}$\forall\gamma\in\Z:\gamma\setminus S(0)$
	\begin{enumerate}[label*=\arabic*.]
		\item{}Если в правой части $\eqref{44:sec}$ $\exists x:\lnot(\gamma-x)\setminus S(x)\Rarr S(\gamma)\neq 0$, перейдём к следущему\\
		делителю $S(0)$ и вернёмся к этому шагу

		Иначе --- перейдём к следущему шагу
		\item{}Вычислим $S(\gamma)$:
		\begin{enumerate}[label=(\alph*)]
			\item{}$S(\gamma)\neq 0$, тогда добавим отсечение $(\alpha-\gamma)\setminus S(\gamma)$ к перечню $\eqref{44:sec}$
			\item{}$S(\gamma)= 0$, тогда добавим $\gamma$ к найденным корням и:
			\begin{enumerate}[label=(\roman*)]
				\item{}Если вычисление $S(\gamma)$ происходило по схеме Горнера, то далее можно искать целые корни $S'(x):=S(x)/(x-\gamma)$

				\item{}Если значение $S(\gamma)$ было вычислено не по схеме Горнера, то можно найти $S'(x):=S(x)/(x-\gamma)$ другим способом и продолжить (i),
				но можно ограничиться делением $S(0)$ на $\gamma$ и проверять делители $S(0)/\gamma$
			\end{enumerate}

			Проверяются делители, не отвергнутые ранее, все отсечения сохраняются.

			$\gamma$ проверяется ещё раз, тк он может быть кратным корнем.

		\end{enumerate}
	\end{enumerate}
\end{enumerate}

\theorem

$S\in\Z[x]$ и старший коэфицент $S$ $=1$
\begin{align*}
	S(\alpha)=0\land\alpha\in\Q\Rarr\alpha\in\Z
\end{align*}

\proof

Пусть несократимая дробь $b/c\in\Q$ является корнем полинома $S$:
\begin{align*}
	S\left(\frac{b}{c}\right)=\left(\frac{b}{c}\right)^n+a_{n-1} \left(\frac{b}{c}\right)^{n-1}+\ldots +a_0=0
\end{align*}

Умножим это равенство на $c^{n-1}$:
\begin{align*}
	 & 0=\frac{b^n}{c}+a_{n-1} b^{n-1}+a_{n-2} b^{n-2} c+...+a_0 c^{n-1}\Rarr \frac{b^{n}}{c}\in\Z\Rarr c\setminus b^{n} \\
	 & (b,c)=1\Rarr(b^{n},c)=1\land c\setminus b^{n}\Rarr c=1\Rarr b/c\in\Z\qed
\end{align*}

\theorem
\begin{align*}
	 & S(x)=a_n x^n+\ldots +a_0\in\mathbb{Z} [x]                           \\
	 & T(y)=y^n+a_{n-1} y^{n-1}+a_n a_{n-2} y^{n-2}+\ldots +a^{n-1}_n a_0.
\end{align*}
\begin{align*}
	T(y)=0\Lrarr S(y/a^{n})=0
\end{align*}

\proof

Корни $S$ совпадают с корнями $a_{n}^{n-1}S$:
\begin{align*}
	a^{n-1}_n S(x)=a^n_n x^n+a^{n-1}_n a_{n-1} x^{n-1}+\ldots +a^{n-1}_n a_0.
\end{align*}

Сделаем замену переменной: $y: = a_n x$, $T(y):=a_{n}^{n-1}S(y/a_{n})$:
\begin{align*}
	T(y):=a_{n}^{n-1}S\left(\frac{y}{a_{n}}\right) & =a_{n}^{n}\left(\frac{y}{a_{n}}\right)^{n}+a_{n}^{n-1}a_{n-1}\left(\frac{y}{a_{n}}\right)^{n-1}+...+a_{n}^{n-1}a_0= \\
	                                               & =y^{n}+a_{n-1}y^{n-1}+a_{n}a_{n-2}y^{n-2}+...+a^{n-1}a_0
\end{align*}

Тогда из определения $T$:
\begin{align*}
	T(y)=0\Lrarr S(y/a_{n})=0\qed
\end{align*}

{\it Замечание:}

Для нахождения всех рациональных корней полинома $S$ можно найти все целые корни $T$ по алгоритму отсечений

\end{document}
