\documentclass{article}

\usepackage{defines}

\begin{document}

\tickettitle{44}{Критерий Эйзенштейна. Поиск целочисленных и рациональных корней.}

\theorem[6.34. Критерий Эйзенштейна]

Пусть имеется полином $f(x)=a_n x^n+a_{n-1} x^{n-1}+\ldots +a_1 x+a_0$ над кольцом целых чисел $\mathbb{Z} [x]$.
Если существует простое p такое, что

\begin{enumerate}
	\item{} $\rceil p\diagdown a_n$;
	\item{} $p\backslash a_i~~i=\overline{0, n-1}$;
	\item{} $\rceil p^2\diagdown a_0$,
\end{enumerate}

то $f$ неприводим над полем рациональных чисел $\R$.

\proof

От противного. Пусть $f$ приводим над $\R$. Согласно предыдущей
теореме он приводим и над кольцом целых чисел $\mathbb{Z}$, т.е.

\begin{align*}
	(\exists S,Q\in Z[x])~f=SQ~\land ~d:=deg~S\geq 1~\land ~l:=deg~Q\geq 1.
\end{align*}

Пусть $S(x)=b_d x^d +\ldots+b_0$ и $Q(x)=e_l x^l+\ldots +e_0$.
Приравниваем коэффициенты при одинаковых степенях в равенстве $f=SQ$, получим

\begin{align*}
	 & a_0=b_0 e_0                    \\
	 & a_1=b_0 e_1+b_1 e_0            \\
	 & a_2=b_0 e_2+b_1 e_1+b_2 e_0    \\
	 & \ldots                         \\
	 & a_l=b_0 e_l+b_1 e_{l-1}+\ldots
\end{align*}

Рассмотрим первое равенство. Так как $p\diagdown a_0$, но $\rceil p^2\diagdown a_0$, то однo из двух чисел $b_0$ либо $e_0$ делится на $p$, а другое нет (простота).
Пусть $\rceil p\diagdown b_0$. Из второго равенства получаем $p\diagdown e_1$, из третьего: $p\diagdown e_2$ и так $l$ раз, т.е. $p\diagdown e_l$.
Но поскольку $a_n=b_d e_l$, это дает $p\diagdown a_n$.  Противоречие с п.1 отвергает приводимость $f$.\newpage

\textbf{Поиск целочисленных и рациональных корней} \newline

Если многочлен над кольцом целых чисел или над полем рациональных, то особый
интерес представляют целочисленные и рациональные корни. Оказывается, узнать об
их существовании и найти их можно конечным алгоритмом. Его основание составляют
следующие три теоремы.

\theorem[6.35]

Если $\alpha\in\mathbb{Z}$, $S\in\mathbb{Z} [x]$ и $S(\alpha )=0$, то $\alpha $ --- делитель свободного члена многочлена $S$, т.е.
$\alpha\diagdown S(0)$, и частное $Q(x)=S(x)/(x-\alpha )$ является целочисленным многочленом.

\proof

Делим $S(x)=a_n x^n+\ldots +a_0$ на $x-\alpha $ по схеме Горнера; получаем частное $Q(x)=b_{n-1} x^{n-1}+\ldots +b_0$.
Приравниваем коэффициенты при одинаковых степенях в представлении $S(x)=(x-\alpha )Q(x)$, получаем

\begin{align*}
	 & b_{n-1}=a_n                    \\
	 & b_{n-2}=a_{n-1}+\alpha b_{n-1} \\
	 & b_{n-3}=a_{n-2}+\alpha b_{n-2} \\
	 & \ldots
\end{align*}

Из этой схемы видно, что все $b_i,~~i=\overline{0, n-1}$, будут целыми, ибо $a_1,\ldots , a_n$ и $\alpha $ --- целые.
В частности $b_0$ --- целое. Но $a_0 = -\alpha b_0 \Rarr \alpha\diagdown a_0~~\qed$ \newline

\textbf{З а м е ч а н и е 11.1.} \newline

Согласно теореме Виета, $a_0$ есть произведение корней, в которое входит, в частности, $\alpha $, но пока не доказано, что $b_0$ --- целое, это ничего не дает. \newline

Теорема 6.35 открывает возможность найти все целые корни полинома путем проверки всех делителей свободного члена, т.е. за конечное число шагов. Однако при большом
числе делителей свободного члена таких шагов, состоящих в вычислении значения многочлена, может оказаться чересчур много.

Уменьшить вычислительные затраты позволяют следующие отсечения. Вначале вычисляют $S(1)$ и $S(-1)$, которые должны быть вычислены и без применения отсечений,
так как $1$ и $-1$ --- делители любого свободного члена. Если $\alpha $ --- корень, то он должен удовлетворять разложению: $S(x) = (x-\alpha )Q(x)$ и $Q$, согласно теореме 6.35, целочисленный.
Следовательно, $Q(1)$ --- целое число. А это значит, что $(1-\alpha)\diagdown S(1)$.

$Q(-1)$ --- тоже целое. Поэтому $(1 + \alpha )\diagdown S(1)$. Поэтому корнем может быть только такое $\alpha $, для которого верно

\begin{align}
	\frac{S(0)}{\alpha }\in\mathbb{Z}~~\land ~~\frac{S(1)}{1-\alpha  }\in\mathbb{Z}~~\land ~~\frac{S(-1)}{1+\alpha }\in\mathbb{Z} \label{40:lt}
\end{align}

Вычислительные затраты на проверку $\eqref{40:lt}$ незначительны.
Это $n$ сложений целых коэффициентов и одно деление суммы. Значительно выше затраты на вычисление значения
полинома от аргументов, отличных от $0$ и $\pm 1$.

Отсечение типа $\eqref{40:lt}$ может пополняться. Пусть, например, некоторое $\alpha_1$ удовлетворило $\eqref{40:lt}$, но проверка показала, что $S(\alpha_1)\neq 0$. Тогда добавляем отсечение $\frac{S(\alpha_1)}{\alpha_1-\alpha }\in\mathbb{Z}$. И т.д.

Если же корень найден: $S(\alpha_1)=0$, то дальнейший поиск целых корней может идти различно.

\begin{enumerate}
	\item Если вычисление значения $S(\alpha_1)$ происходило по схеме Горнера, то далее оптимальным будет искать целые корни полинома $S_1(x):=S(x)\diagup (x-\alpha_1)$, коэффициенты которого уже вычислены в схеме Горнера.
	      Проверяются, конечно, только те делители свободного члена полинома $S_1$, которые не были отвергнуты раньше. \\
	      Если $\alpha_1$ продолжает быть делителем нового свободного
	      члена, то он тоже проверяется, так как он может оказаться кратным корнем.
	\item Если значение $S(\alpha_1)$ было вычислено не по схеме Горнера, то можно разделить $S(x)$ на $x-\alpha_1$ каким-то способом и далее поступать как в п.1. Но можно ограничиться
	      делением свободного члена $s_0$ на $\alpha_1$ и проверять ранее не отвергнутые делители $s_0\diagup\alpha_1$
	      подстановкой их в $S$. \\
	      В обоих вариантах все предыдущие отсечения остаются в силе $\qed$
\end{enumerate}

\theorem[6.36.]

Все рациональные корни полинома

\begin{align*}
	S(x)=x^n+a_{n-1}x^{n-1}+\ldots +a_0,~~\{a_i\}^{n-1}_0\in\mathbb{Z},
\end{align*}

целые.

\proof

Пусть несократимая дробь $b/c$ является корнем полинома $S$.

Тогда

\begin{align*}
	0=\left(\frac{b}{c}\right)^n+a_{n-1} \left(\frac{b}{c}\right)^{n-1}+\ldots +a_0.
\end{align*}

Умножим это равенство на $c^{n-1}$:

\begin{align*}
	0=\frac{b^n}{c}+a_{n-1} b^{n-1}+a_{n-2} b^{n-2} c+\ldots +a_0 c^{n-1}.
\end{align*}

Отсюда видно, что число $b^n/c$ --- целое. Но $(b, c) = 1 \Rarr (b_n, c) = 1$.

Это совместно с  $c\diagdown b^n$ только когда $c = 1~~\qed$ \newline

Если коэффициент при старшей степени полинома отличен от 1, то могут быть не целые корни, но их тоже можно найти за конечное число шагов. \newpage

\theorem[6.37.]

Все рациональные корни полинома

\begin{align*}
	S(x)=a_n x^n+\ldots +a_0\in\mathbb{Z} [x],
\end{align*}

(как дробные, так и целые) являются целыми корнями полинома

\begin{align*}
	T(y)=y^n+a_{n-1} y^{n-1}+a_n a_{n-2} y^{n-2}+\ldots +a^{n-1}_n a_0.
\end{align*}

разделенными на $a_n$.

\proof

Ясно, что корни полинома $S$ совпадают с корнями полинома

\begin{align*}
	a^{n-1}_n S(x)=a^n_n x^n+a^{n-1}_n a_{n-1} x^{n-1}+\ldots +a^{n-1}_n a_0.
\end{align*}

Сделаем замену переменной $y = a_n x$. $T(y)=a^{n-1}_n S(x)$.
Отсюда видно, что каждому корню $\beta_i$ полинома $T$ соответствует корень $\alpha_i=\beta_i/a_n$ полинома $S~~\qed$

\end{document}
