\documentclass{article}

\usepackage{defines}

\begin{document}

\tickettitle{43}{Связь приводимости целочисленного многочлена в кольце целых чисел с его приводимостью над полем рациональных чисел}

\theorem[6.30]

Приводимость сопутствующего полинома $Q$ над кольцом целых чисел
или в поле рациональных влечет приводимость исходного полинома $P$ в поле рациональных чисел.

\proof

Приводимость полинома $Q$ означает существование полиномов
$Q_1, Q_2\in\mathbb{Z} [x]$, таких, что $deg~Q_1\geq 1,~deg~Q_2\geq 1$ . И, согласно определению 6.18, существуют $d, z\in\mathbb{Z}$, такие, что

\begin{align*}
	P=\frac{d}{z} Q=\frac{d}{z} (Q_1 Q_2)=(\frac{d}{z} Q_1) Q_2.
\end{align*}

В силу отсутствия делителей нуля в поле рациональных чисел, степень полинома
$\frac{d}{z} Q_1$ не менее 1 $\qed$ \newline

Теория приводимости имеет место не только для полиномов над кольцом целых чисел.
В частности, кольцо может удовлетворять импликации

\begin{align}
	(a, m)=1\land (b, m)=1 \Rarr (ab, m)=1\label{40:gt}
\end{align}

\theorem[6.32]

Полином $P(x)$ с целочисленными коэффициентами, приводимый над
полем рациональных чисел, приводим также и над кольцом целых чисел.

\proof

Пусть $P=SQ$, где $S$ и $Q$ --- полиномы над полем рациональных
чисел степени не меньше 1.
Пусть коэффициенты полинома $S$ являются несократимыми
дробями, и $C$ --- НОК их знаменателей.
Тогда все коэффициенты $CS$ целые. Пусть $c$ их НОД.\newline

Очевидно, $f(x):=\frac{C}{c} S(x)$ ---  примитивный полином над кольцом целых чисел. \newline

С полиномом $Q$ поступим также: $Q(x)=\frac{b}{B} g(x0$, где $g(x)$ --- примитивен. \newline

Тогда $P(x)=\frac{c}{C} \frac{b}{B} f(x)g(x)$. Приведем дробь $\frac{c}{C} \frac{b}{B}$ к несократимой $\frac{a}{A}$.
Если $A>1$, то существует $\alpha$ --- простой делитель $A$ (возможно и $\alpha =A$). \newline

По лемме Гаусса $f(x)g(x)=:a_n x^n+\ldots +a_0$ --- примитивный многочлен.
Следовательно, $\alpha$ не делит все его коэффициенты. Пусть $\rceil\alpha\backslash a_k$, тогда в силу простоты $\alpha$ верно $(\alpha , a_k)=1$.
С другой стороны, $(\alpha , a)=1$, и по теореме-следствию из линейного представления НОД $(\alpha , a a_k)=1$, что влечет $\rceil\alpha\backslash a a_k$. Но $\alpha$ должно делить $a a_k$, поскольку $a a_k/\alpha$ является коэффициентом целочисленного полинома $P$ при степени $k$.
Источник противоречия --- предположение $A > 1$. \newline

Отвергая его, получаем $P(x)=a f(x) g(x)$, где $a$ --- целое число, $f$ и $g$ целочисленные полиномы степени не менее 1. Полагая $h(x) = a f(x)$, имеем $P = hg$, т.е. приводимость $P$ в кольце $\mathbb{Z} [x]~\qed$

Этот же результат другими словами. \newpage

\theorem[6.33]

Если полином $P$ из $\mathbb{Z} [x]$ неприводим над кольцом целых чисел $\mathbb{Z}$, то он не приводим и над полем рациональных чисел.\newline

Как уже говорилось, целочисленный полином сколь угодно большой степени может
не иметь рациональных корней. Аналогично и с приводимостью. Можно указать целочисленные полиномы сколь угодно больших степеней, которые не приводимы над кольцом
целых чисел, а в силу только что доказанной теоремы, и над полем рациональных чисел.

\end{document}
