\documentclass{article}

\usepackage{defines}

\begin{document}

\tickettitle{43}{Связь приводимости целочисленного многочлена в кольце целых чисел с его приводимостью над полем рациональных чисел}

\theorem

$P\in\Z[x]$ --- приводим над $\Q$ $\Rarr$ $P$ --- приводим над $\Z$

\proof
\newcommand\Sp{\widetilde{S}}
\newcommand\Qp{\widetilde{Q}}

$P$ --- приводим над $\Q$:
\begin{align*}
	\exists S,Q\in\Q[x]:\deg S\geq 1\land \deg Q\geq 1\land P=SQ
\end{align*}

Пусть $\Sp$ --- сопутстующий $S$, а $\Qp$ --- $Q$, те $\Sp$, $\Qp$ --- примитивны и выполняется:
\begin{align*}
	 & S=\frac{c}{C}\Sp,\;c,C\in\Z,\Sp\in\Z[x] &  & Q=\frac{b}{B}\Qp,\;b,B\in\Z,\Qp\in\Z[x]
\end{align*}
\begin{align*}
	P(x)=S(x)Q(x)=\frac{c}{C}\frac{b}{B}\Sp(x)\Qp(x)=\frac{a}{A}\Sp(x)\Qp(x)
\end{align*}

Причём $a/A$ --- сокращённая дробь $bc/BC$, те $(a,A)=1$

Пусть $A>1$ $\Rarr$ $\exists\text{простое }d:d\setminus A\Rarr\lnot d\setminus a\Rarr (d,a)=1\text{ (по простоте $d$)}$

По лемме Гаусса $\Sp(x)\Qp(x)=a_{n}x^{n}+...+a_{0}$ --- примитивный многочлен $\Rarr$ $\exists k:\lnot d\setminus a_{k}$
\begin{align*}
	\lnot d\setminus a_{k} & \Rarr (d, a_{k})=1\land (d, a)=1\Rarr (d, aa_{k})=1\Rarr \frac{aa_{k}}{d}\notin\Z \Rarr       \\
	                       & \Rarr \frac{aa_{k}}{A}\notin\Z\text{, но }\frac{aa_{k}}{A} \text{ --- коэфицент $P\in\Z[x]$ }
\end{align*}

Источник противоречия --- предположение $A > 1$ $\Rarr$ $A=1$

Тогда:
\begin{align*}
	(a\Sp),\Qp\in\Z[x]\land\deg (a\Sp)\geq 1\land\deg\Qp\geq 1\land P=(a\Sp)\Qp\Rarr P\text{ --- приводим над $\Z\qed$}
\end{align*}

\theorem

$P\in\Z[x]$ --- неприводим над $\Z$ $\Rarr$ $P$ --- неприводим над $\Q$

\proof

Утверждение получается обращением предыдущей теоремы$\qed$

{\it Замечание:}

Как уже говорилось, полином в $\Z[x]$ сколь угодно большой степени может не иметь корней в $\Q$

Аналогично с приводимостью. Можно указать полиномы в $\Z[x]$ сколь угодно больших степеней, которые не приводимы над $\Z$ и, как следствие над $\Q$

\end{document}
