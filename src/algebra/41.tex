\documentclass{article}

\usepackage{defines}

\begin{document}

\tickettitle{41}{Приводимость полиномов}

\define

Полином $P$ называется приводимым над кольцом $K$, если:
\begin{align*}
	\exists S,Q\in K[x]:\deg S\geq 1\land \deg Q\geq 1\land P=SQ
\end{align*}

Приводимость над $\Cx$ любого полинома при $\deg>1$ следует из ОТА

Над $\R$ приводим любой полином $\deg>2$ и могут быть неприводимы полиномы $\deg=2$ вида $ax^2 + bx +c$, когда $b^2 - 4ac < 0$

Над $\Q$ могут быть неприводимы полиномы сколь угодно высоких степеней

\theorem
\begin{align*}
	d, n\in\N\land\sqrt[d]{n}\notin \N \Rarr \nexists p, q\in \N:(p,q) = 1\land \sqrt[d]{n} = p/q
\end{align*}

\proof

Пусть $\sqrt[d]{n} = p/q$, $p,q\in\N$:
\begin{align*}
	n=\frac{p^{d}}{q^{d}}\in\Z & \Rarr q^{d}\setminus p^{d}\Rarr q\setminus p^{d} \Rarr q\setminus p^{d-1}p\land (p,q)=1 \\
	                           & \Rarr q\setminus p^{d-1}\land(p,q)=1\Rarr...\Rarr q\setminus p\land(p,q)=1\Rarr q=1
\end{align*}

Тогда $p/q\in\N$, что противоречит условию, значит, таких $p$ и $q$ не существует$\qed$
\end{document}
