\documentclass{article}

\usepackage{defines}

\begin{document}

\tickettitle{2}{Классификация отображений и бинарные отношения}

\define{отображения}

Пусть объявлено отображение $f$ записью $A \overset{f}{\to} B$ \ или $f: A\to B$

Читается: (имеется) отображение $f$ из (множества) $A$ в (множество) $B$

Множество $A$ --- область задания отображения $f$

Множество $B$ --- область значений отображения $f$

Конкретному $x \in A$ отображение $f$ сопоставляет конкретный $y \in B$: $y=f(x)$,\\
где $y$ --- образ элемента $x$, а $x$ --- прообраз элемента $y$.

\define{естественного расширения отображения}

Естественное расширение отображения $f: A\to B$, это отображение $\widetilde{f}$, заданное на множестве подмножеств множества $A$ формулой
\begin{align*}
	\widetilde{f}(\alpha):=\setdef{y\in B}{\exists x\in \alpha:y=f(x)},\;\alpha \subset A
\end{align*}

Обозначение: $\widetilde{f}: 2^A \to 2^B$, где $2^{A}:=\setdef{\alpha}{\alpha\subset A}$

Если $z \in 2^A$, то $\widetilde{f}(z)$ --- образ множества $z$, а $z$ --- прообраз множества $\widetilde{f}(z)$

Обычно тильду опускают и $\widetilde{f}$ тоже обозначают через $f$

\sectitle{Классификация отображений}

\define{инъекции}

Инъекция --- это “взаимно однозначное отображение в ...”
\begin{align*}
	f:A\to B \text{--- инъекция} \Lrarr (\forall x,y \in A)\;x\neq y\Rarr f(x)\neq f(y)
\end{align*}

{\it Пример:} $f$ --- инъекция
\begin{align*}
	 & f:\left[-\frac{\pi}{2}, \frac{\pi}{2}\right]\to\R &  & f(x)=\sin(x)
\end{align*}

\define{сюръекции}

Сюръекция --- это “отображение на ...”.
\begin{align*}
	f:A\to B \text{ --- сюръекция} \Lrarr (\forall y \in B)\;\exists x \in A: y=f(x)
\end{align*}

Из определения следует, что $B=f(A)$

{\it Пример:} $f$ --- сюръекция
\begin{align*}
	 & f:\R\to[-1;1] &  & f(x)=\sin(x)
\end{align*}

\pagebreak

\define{биекции}

Биекция --- (сюръективная инъекция, инъективная сюръекция) — это “взаимно однозначное отображение на...”.
\begin{align*}
	f:A\to B \text{ --- биекция} \Lrarr (\forall y \in B) \; \exists ! x \in A: y=f(x)
\end{align*}

{\it Пример:} $f$ --- биекция
\begin{align*}
	 & f:\left(-\frac{\pi}{2};\frac{\pi}{2}\right)\to(-1;1) &  & f(x)=\sin(x)
\end{align*}

{\it Замечание:}

Существуют отображения, не являющиеся  ни инъекцией, ни сюръекцией, ни биекцией

{\it Пример:} $f$ --- ни инъекция, ни сюръекция, ни биекция
\begin{align*}
	 & f:\R\to\R &  & f(x)=\sin(x)
\end{align*}

\define{бинарных отношений}

Отображение декартова квадрата $M \times M$ в двоичное множество $\Lambda$ называют бинарным отношением:
\begin{align*}
	R : M\times M \to \Lambda
\end{align*}

Обозначение: $aRb$, или $R(a, b)$, или $(a,b)\in R$, тк часто отношение рассматривается\\
как множество пар, для которых верно $aRb$

Булева алгебра - частный случай бинарных отношений: $M=\Lambda$

\define{свойств бинарных отношений}
\begin{enumerate}
	\item рефлексивность: $(\forall a\in M)\;aRa\text{ --- истина}$
	\item антирефлексивность: $(\forall a\in M)\;aRa\text{ --- ложь}$
	\item симметрия: $(\forall a,b\in M)\;aRb \Lrarr bRa$
	\item антисимметрия: $(\forall a,b\in M:a\neq b)\;aRb \Lrarr \lnot bRa$;
	\item ассиметрия: антисимметрия $\land$ антирефлексивность
	\item транзитивность: $(\forall a,b,c\in M)\;aRb \land bRc \Rarr aRc$.
\end{enumerate}

{\it Примеры:}
\begin{enumerate}
	\item Отношение строгого порядка: транзитивность $\land$ антисимметрия $\land$ антирефлексивность\\
	      Обозначения: $<, >, \succ, \prec$
	\item Отношение нестрогого порядка: транзитивность $\land$ антисимметрия $\land$ рефлексивность\\
	      Обозначения: $\ge, \geqslant, \le, \leqslant, \succeq, \preceq$
	\item Отношение эквивалентности: транзитивность $\land$ симметрия $\land$ рефлексивность\\
	      Обозначения: $\Lrarr, \sim, \equiv, =$
\end{enumerate}

\pagebreak

\define{произведения бинарных отношений}

Произведение бинарных отношений $R,S:M\times M\to\Lambda$ обозначается $RS$ и определяется:
\begin{align*}
	RS:=\setdef{(a, c)\in M\times M}{\exists b \in M: (a,b)\in R \land (b,c)\in S}
\end{align*}

Или же не на языке множества пар:
\begin{align*}
	(\forall a,c\in M)\;a(RS)c\Lrarr \exists b\in M:aRb\land bSc
\end{align*}

\theorem

Произведение бинарных отношений ассоциативно:
\begin{align*}
	(QR)S=Q(RS)
\end{align*}

\proof

Рассмотрим $(QR)S$ по определению:
\begin{align*}
	(QR)S
	 & =\setdef{(a,c)\in M\times M}{\exists b\in M:[(a,b)\in QR]\land (b,c)\in S}=                               \\
	 & =\setdef{(a,c)\in M\times M}{\exists b\in M:[\exists d\in M:(a,d)\in Q\land (d,b)\in R]\land (b,c)\in S}= \\
	 & =\setdef{(a,c)\in M\times M}{\exists b,d\in M:(a,d)\in Q\land (d,b)\in R\land (b,c)\in S}=                \\
	 & =\setdef{(a,c)\in M\times M}{\exists d\in M:(a,d)\in Q\land [\exists b\in M:(d,b)\in R\land (b,c)\in S]}= \\
	 & =\setdef{(a,c)\in M\times M}{\exists d\in M:(a,d)\in Q\land (d,c)\in RS}=Q(RS)\qed
\end{align*}

\end{document}
