\documentclass{article}

\usepackage{defines}

\usepackage{enumitem}

\usepackage{enumerate}

\usepackage{amsfonts}

\begin{document}

\tickettitle{2}{Классификация отображений и бинарные отношения}

\define{отображения}

$\llet$ объявлено отображение $f$ записью $A \ \overset{f}{\mapsto} B$ \ или $f: A\mapsto B$ (читается: (имеется) отображение f из (множества) A в (множество) B.) Множество A --- область задания отображения $f$. Множество
B --- область значений отображения $f$. Конкретному $x \in A$ отображение $f$ сопоставляет конкретный $y \in B$: $y=f(x)$, где $y$ --- образ элемента $x$, а $x$ --- прообраз элемента $y$.

\define{естественного расширения отображения}

Естественное расширение отображения $f: A\mapsto B$, это отображение $\widetilde{f}$, заданное на множестве подмножеств множества A формулой $$\widetilde{f}(\alpha):=\setdef{y}{y=f(x), x \in \alpha},$$ где $\alpha \subseteq A$.

Обозначение: $\widetilde{f}: 2^A \mapsto 2^B$.

Если $z \in 2^a$, то $\widetilde{f}(z)$ --- образ множества $z$, а $z$ --- прообраз множества $\widetilde{f}(z)$.

$$\textbf{Разновидности отображений}$$
\begin{enumerate}
    \item Инъекция --- это “взаимно однозначное отображение в ...”. $$f \text{--- инъекция} \Lrarr (\forall x \in A) \ y \in A\backslash \{x\} \Rarr f(x)\neq f(y).$$

    П:
    \begin{align*}  
         A & =\left[-\frac{\pi}{2}, \frac{\pi}{2}\right] & A & =\left[-\frac{\pi}{2}, \frac{\pi}{2}\right]
        \\
        \sin{A} & \mapsto \mathbb{R}' - \text{инъекция} & \sin{A} & =[-1, 1] - \text{расширение}
    \end{align*}
    \item Сюръекция --- это “отображение на ...”. $$f \text{ --- сюръекция} \Lrarr (\forall y \in B) \ (\exists x \in A) \ y=f(x)$$
    Область задания: $A$
    
    Область значений: $B':=f(A)$.
    
    П:
    \begin{align*}
        & A=[-\pi, \pi] \\
        & B=[-1, 1] \\
        & \sin[-\pi, \pi]=[-1, 1]
    \end{align*}

    \item Биекция --- (сюръективная инъекция, инъективная сюръекция) — это “взаимно однозначное отображение на...”. $$f \text{ --- биекция} \Lrarr (\forall y \in B) \ (\exists ! x \in A) \ y=f(x)$$

    П:
    \begin{align*}
        & A=\left(-\frac{\pi}{2}, \frac{\pi}{2}\right) \\
        & B=(-1, 1) \\
        & \sin \left(-\frac{\pi}{2}, \frac{\pi}{2}\right)=(-1, 1)
    \end{align*}

    \item Не является  ни инъекцией, ни сюръекцией, ни биекцией.
    
    П: $sin: \mathbb{R}' \mapsto \mathbb{R}'$
    
\define{бинарных отношений}

Отображение декартова квадрата $M \times M$ в двоичное множество $\Lambda$ называют бинарным отношением: $$\beta : M\times M \mapsto \Lambda$$

Обозначение: $aRb$ или $R(a, b)$

Булева алгебра - частный случай бинарных отношений: $M=\Lambda$

$$\text{Бинарные отношения могут иметь свойства:}$$
\begin{enumerate}[1)]
    \item рефлексивность: $aRa$ --- истина $\forall a$;
    \item антирефлексивность: $aRa$ --- ложь $\forall a$;
    \item симметрия: $aRb \Lrarr bRa \ \forall a, b$;
    \item ассиметрия $aRb \Lrarr \rceil bRa \ \forall a, b$
    \item антисимметрия: $a\neq b \Rarr \ (aRb \Lrarr \rceil bRa)$;
    \item транзитивность: $(aRb \land bRc \Rarr aRc) \ \forall a, b, c$.   
\end{enumerate}

\begin{enumerate}[1]
    \item Отношение строгого порядка: транзитивность $\land$ асимметричность. \\ Обозначения: $<, >, \succ, \prec$.
    \item Отношение нестрогого порядка: транзитивность $\land$ антисимметричность $\land$ рефлексивность. \\ Обозначения: $\ge, \geqslant, \le, \leqslant, \succeq, \preceq$.
    \item Отношение эквивалентности: симметрия $\land$ транзитивность $\land$ рефлексивность. \\ Обозначения: $\Rarr, \sim, \equiv, =$.
\end{enumerate}

Произведение бинарных отношений: $$RS:=\setdef{(a, c)}{(\exists b \in M): aRb \land bRc}$$

\theorem 

Произведение бинарных отношений обладае ассоциативностью: $$(QR)S=Q(RS)$$




    
\end{enumerate}

\end{document}