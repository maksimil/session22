\documentclass{article}

\usepackage{defines}

\begin{document}

\tickettitle{42}{Многочлены над полем рациональных чисел. Лемма Гаусса.}

\define{6.18}

Пусть имеется полином $P(x)=a_n x^n+\ldots +a_0$ над полем рациональных чисел. Пусть $z$ --- наименьшее общее кратное всех знаменателей его коэффициентов.
Тогда $zP(x)=:Q(x)$ --- полином с целыми коэффициентами: $\tilde{Q} (x)\in \mathbb{Z} [x]$.
Если $d$ ---  НОД коэффициентов полинома $\tilde{Q}$, то имеет место разложение $\tilde{Q} = dQ$. Полином $Q$  называется сопутствующим полиному $P$.\newline

Согласно определению 6.18, если $Q$ ---  сопутствующий полином, то $P = \frac{d}{z}Q$, где $d, z\in \mathbb{Z}$.\newline

Понятие сопутствующего полинома обобщается аналогичным образом на произвольные поля, являющиеся пополнением обратными элементами колец с единицами и без делителей нуля.\\
Далее --- рациональные поля. Обозначать их будем с помощью "шляпки"над символом порождающего кольца. В частности, множество рациональных чисел получает обозначение $\widehat{\mathbb{Z}}$.\\
Отличным от целых чисел примером колец, порождающих рациональные поля, могут служить кольца полиномов от произвольного количества переменных $x_1,...,x_n$ над полями или кольцами с единицей и без делителей нуля.
(Переменная $x$ исходного полинома — другая.)

\define{6.19}

Пусть $K$ --- кольцо с единицей. Если все коэффициенты полинома $P\in K[x]$
не имеют общего делителя отличного от единицы, то говорят, что этот полином
примитивен.\newline

Согласно определению 6.19, сопутствующий (как по определению 6.18, так
и по его обобщению) полином примитивен.

\theorem[6.29]

Сопутствующий полином единствен с точностью до знака.

\proof

Пусть $P$ --- полином над полем рациональных чисел, и имеются два его разложения:

\begin{align*}
	(P =\frac{a_1}{b_1} Q_1 = \frac{a_2}{b_2} Q_2)\land (Q_1, Q_2\in \mathbb{Z} [x])\land (a_1, a_2, b_1, b_2\in \mathbb{Z}),
\end{align*}

причем дроби $a_1/b_1, a_2/b_2$ несократимы, а $Q_1, Q_2$ --- примитивны.
Тогда справедливо

\begin{align*}
	a_1 b_2 Q_1 = a_2 b_1 Q_2 =:\tilde{Q}\in\mathbb{Z} [x],
\end{align*}

т.е. $a_1 b_2$ и $a_2 b_1$ в силу примитивности $Q_1$ и, соответственно, $Q_2$,
---наибольшие общие делители всех коэффициентов целочисленного полинома
$\tilde{Q}$. В силу единственности НОД, с точностью до знака $\sigma =\pm$ верно

\begin{align*}
	a_1 b_2 =\sigma a_2 b_1 \Lrarr \frac{a_1}{b_1} = \sigma\frac{a_2}{b_2} \Rarr Q_1 = \sigma Q_2~\qed
\end{align*} \newpage

\theorem[6.30]

Приводимость сопутствующего полинома $Q$ над кольцом целых чисел
или в поле рациональных влечет приводимость исходного полинома $P$ в поле рациональных чисел.

\proof

Приводимость полинома $Q$ означает существование полиномов
$Q_1, Q_2\in\mathbb{Z} [x]$, таких, что $deg~Q_1\geq 1,~deg~Q_2\geq 1$ . И, согласно определению 6.18, существуют $d, z\in\mathbb{Z}$, такие, что

\begin{align*}
	P=\frac{d}{z} Q=\frac{d}{z} (Q_1 Q_2)=(\frac{d}{z} Q_1) Q_2.
\end{align*}

В силу отсутствия делителей нуля в поле рациональных чисел, степень полинома
$\frac{d}{z} Q_1$ не менее 1 $\qed$ \newline

Теория приводимости имеет место не только для полиномов над кольцом целых чисел.
В частности, кольцо может удовлетворять импликации

\begin{align}
	(a, m)=1\land (b, m)=1 \Rarr (ab, m)=1\label{40:gt}
\end{align}

\theorem[6.31 (Лемма Гаусса)]

Пусть в некотором кольце выполняется (6.53), тогда
произведение примитивных полиномов над этим кольцом есть примитивный полином.

\proof

Пусть существуют два примитивных многочлена

\begin{align*}
	Q(x):=b_k x^k+\ldots +b_1 x+b_0,~~~
	S(x)=c_l x^l+\ldots +c_1 x+c_0,~~~
	b_k, c_l\neq 0.
\end{align*}

Рассмотрим их произведение $P(x)=(b_k x^k+\ldots +b_0)(c_l x^l+\ldots +c_0)=:a_n x^n+\ldots +a_0.$

Предположим, что $P$ не примитивен, тогда существует такое $\alpha > 1$, что $\alpha\backslash a_i,~i=0,\ldots, n$.
Если $\alpha$ не простое число, то выберем какой-нибудь его простой делитель $d$.
Так как $S$ и $Q$ примитивны, то не все их коэффициенты делятся на $d$.
Пусть какое-то количество коэффициентов при подряд идущих высоких степенях делится на $d$.

\begin{align*}
	 & k>r \Rarr d\backslash b_i,~~i=\overline{k, r+1},~~\land~~
	\rceil d\backslash b_r,
	\\
	 & l>t \Rarr d\backslash c_i,~~i=\overline{l, t+1},~~\land~~
	\rceil d\backslash c_t,
\end{align*}

где $r, t\geq 0$. Или может быть, что $d$ не делит уже и старший коэффициент:
$\rceil d\backslash b_k$, или $\rceil d\backslash c_l$.
Коэффициент полинома $P$ при степени $r+t$  таков

\begin{align}
	a_{r+t}=b_r c_t+(b_{r-1} c_{t+1}+b_{r+1} c_{t-1})+(b_{r-2} c_{t+2}+b_{r+2} c_{t-2})+\ldots\label{40:lt}
\end{align}

(Если индекс коэффициента полинома выходит за допустимые границы, то коэффициент
с таким индексом полагается нулем.)
По предположению $d\backslash b_{r+1}~\land ~d\backslash c_{t+1}$, значит и
первая скобка в $\eqref{40:lt}$ делится на $d$, аналогично со второй скобкой там, и так далее.

По построению, $b_r$ и $c_t$ не делятся на $d$. Поэтому в силу простоты $d$ имеем

\begin{align*}
	(b_r, d) = 1~\land~(c_t, d) = 1.
\end{align*}

Воспользовавшись $\eqref{40:gt}$, получаем, что $b_r c_t$ не делится на $d$,
а прочие слагаемые в $\eqref{40:lt}$ делятся на $d$.
Следовательно, вся правая часть $\eqref{40:lt}$ не делится на $d$, значит и $a_{r+t}$ не делится на $d$. Противоречие с определением числа $d$ доказывает лемму Гаусса.

\end{document}
