\documentclass{article}

\usepackage{defines}

\begin{document}

\tickettitle{42}{Многочлены над полем рациональных чисел. Лемма Гаусса.}

\define{сопутстующего полинома}
\newcommand\Qt{\widetilde{Q}}

Пусть имеется полином $P(x)\in\Q[x]$.

Пусть $z$ --- НОК всех знаменателей его коэффициентов. Тогда $\Qt(x):=zP(x)\in\Z[x]$.

Если $d$ ---  НОД коэффициентов полинома $\Qt$, то имеет место разложение $\Qt=dQ$.

Тогда полином $Q$  называется сопутствующим полиному $P$.

$Q$ --- сопутствующий полином $\Rarr$ $P=\frac{d}{z}Q,\;d,z\in\Z$

Понятие сопутствующего полинома обобщается аналогичным образом на произвольные поля, являющиеся пополнением обратными элементами колец с единицами и без делителей нуля.\\
Далее --- рациональные поля. Обозначать их будем с помощью "шляпки" над символом порождающего кольца.

В частности, множество рациональных чисел получает обозначение $\widehat{\mathbb{Z}}$.

\define{примитивного полинома}

Пусть $K$ --- кольцо с единицей. Если все коэффициенты полинома $P\in K[x]$
не имеют общего делителя отличного от единицы, то говорят, что этот полином
примитивен.

Сопутствующий полином примитивен по определению.

\theorem

Сопутствующий полином единствен с точностью до знака.

\proof

Пусть $P\in\Q[x]$, и имеются два его разложения:
\begin{align*}
	(P =\frac{a_1}{b_1} Q_1 = \frac{a_2}{b_2} Q_2)\land (Q_1, Q_2\in \mathbb{Z} [x])\land (a_1, a_2, b_1, b_2\in \mathbb{Z})
\end{align*}

где $a_1/b_1, a_2/b_2$ --- несократимы, а $Q_1, Q_2$ --- примитивны. Тогда справедливо:
\begin{align*}
	a_1 b_2 Q_1 = a_2 b_1 Q_2 =:\Qt\in\mathbb{Z} [x],
\end{align*}

$Q_1,Q_2$ --- примитивны $\Rarr$ $a_1b_2$ и $a_2b_1$ --- НОД коэфицентов $\Qt$

НОД единственен с точностью до знака, значит:
\begin{align*}
	\sigma\in\{1,-1\},\;a_1 b_2 =\sigma a_2 b_1 \Lrarr \frac{a_1}{b_1} = \sigma\frac{a_2}{b_2} \Rarr Q_1 = \sigma Q_2~\qed
\end{align*}

\pagebreak

\theorem

$Q$ --- полином, сопутствующий $P$

$Q$ --- приводим над $\Q$ или над $\Z$ $\Rarr$ $P$ --- приводим над $\Q$

\proof

\begin{enumerate}
	\item{}$Q$ --- приводим над $\Q$:
	\begin{align*}
		\exists Q_1,Q_2\in\Z[x]:\deg Q_1\geq 1\land\deg Q_2\geq 1\land Q=Q_1Q_2
	\end{align*}

	По определению сопутствующего полинома $\exists d,z\in\Z$:
	\begin{align*}
		P=\frac{d}{z}Q=\frac{d}{z}(Q_1Q_2)=\left(\frac{d}{z}Q_1\right)Q_2
	\end{align*}

	В $\Q$ не существует делителей $0$ $\Rarr$ $\deg\frac{d}{z}Q_1=\deg Q_1$

	Таким образом:
	\begin{align*}
		\left(\frac{d}{z}Q_1\right),Q_2\in\Z[x]\land\deg\left(\frac{d}{z}Q_1\right)\geq 1\land\deg Q_2\geq 1\land P=\left(\frac{d}{z}Q_1\right)Q_2\Rarr P\text{ --- приводим в $\Q[x]$}\qed
	\end{align*}

	\item{}$Q$ --- приводим над $\Z$ $\Rarr$ $Q$ --- приводим над $\Q$ (тк $\Z[x]\subset\Q[x]$) $\Rarr$ $P$ --- приводим над $\Q\qed$

\end{enumerate}

\lemma[Гаусса]

Пусть в кольце $K$ выполняется:
\begin{align}
	(\forall a,b,m\in K)\;(a,m)=1\land(b,m)=1\Rarr(ab,m)=1\label{42:coprime}
\end{align}

Тогда:
\begin{align*}
	(\forall A,B\in K)\;A,B\text{ --- примитивны}\Rarr AB\text{ --- примитивен}
\end{align*}

\proof

Возьмём два примитивных полинома из $K[x]$: \begin{align*}
	 & Q(x):=b_k x^k+...+b_1 x+b_0 &  & S(x):=c_l x^l+...+c_1 x+c_0 &  & b_k, c_l\neq 0
\end{align*}

Рассмотрим их произведение:
\begin{align*}
	P(x):=(b_k x^k+\ldots +b_0)(c_l x^l+\ldots +c_0)=a_n x^n+\ldots +a_0.
\end{align*}

Пусть $P$ --- не примитивен $\Rarr$ $\exists \text{простое } d>1:(\forall i=\overline{0,n})\;d\setminus a_{i}$

$Q$, $S$ --- примитивны $\Rarr$ не все их коэффициенты делятся на $d$, тогда:
\begin{align*}
	 & (\forall i=\overline{r+1,k})\;d\setminus b_{i}\land\lnot d\setminus b_{r} &  & (\forall i=\overline{t+1,l})\;d\setminus c_{i}\land\lnot d\setminus c_{t}
\end{align*}

Другими словами, первые $k-r$ коэфицентов $Q$ делятся на $d$, первые $l-t$ коэфицентов $S$ делятся на $d$

Тогда рассмотрим $a_{r+t}$:
\begin{align*}
	a_{r+t}=b_r c_t+(b_{r-1} c_{t+1}+b_{r+1} c_{t-1})+(b_{r-2} c_{t+2}+b_{r+2} c_{t-2})+\ldots
\end{align*}

Причём если индекс коэффициента полинома выходит за допустимые границы, то коэффициент
с таким индексом полагается нулем.

По предположению $d\setminus b_{r+1}\land d\setminus c_{t+1}$ $\Rarr$ первая скобка делится на $d$

Аналогично на $d$ делятся и остальные скобки.

По построению, $\lnot d\setminus b_{r}\land \lnot d\setminus c_{t}\land d\text{ --- простое}$:
\begin{align*}
	 & (b_r, d) = 1\land(c_t, d) = 1\Rarr (b_{r}c_{t},d)=1 \text{ (по $\eqref{42:coprime}$)}\Rarr \lnot d\setminus b_{r}c_{t}        \\
	 & \lnot d\setminus b_{r}c_{t}\land (\forall i\in\N)\; d\setminus (b_{r-i}c_{t+i}+b_{r+i}c_{t-i}) \Rarr \lnot d\setminus a_{r+t}
\end{align*}

Что противоречит определению $d$ $\Rarr$ $P$ --- примитивен$\qed$

\end{document}
